\section{Lecture 1 ---  3rd January --- Baby Stuff}

\subsection{References for the Course}
\begin{itemize}
    \item De Carmo -- Curves and Surfaces
    \item Tu -- Introduction to Smooth Manifolds
    \item Lee -- Introduction to Smooth Manifolds
\end{itemize}

\subsection{Geometry of curves in three dimensions}
\begin{definition}[smooth]
    A real function of real variable is \textit{smooth} if it has, at all points, derivatives of all orders.
    \label{def:smooth}
\end{definition}
\begin{definition} [parameterized curve]    
Let $I \subset \R$ be an open interval. A \textit{parameterized curve} in $\R ^{n}$ is a smooth map $\gamma : I \to \mathbb R ^{n}$.
    \label{def:parameterized curve}
\end{definition}

\begin{example}
    Some of examples of curves are:
    \begin{enumerate}
	\item $\gamma : \R \to \R ^{2}$, $t \stackrel{\gamma}{\mapsto} \left( \cos t, \sin t \right)$
	\item $\gamma : \R \to \R ^{2}$, $t \stackrel{\gamma}{\mapsto} \left( t, mt \right)$ for any real number $m$.
    \end{enumerate}
\end{example}

\subsection{Brief Review of Inverse Function Theorem \& Reparameterization of Curves}

Let us recall what the Inverse Function Theorem\footnote{See Denlinger, Elements of Real Analysis: Theorem 6.2.4} says:
\begin{theorem}
    Suppose $f$ is $1-1$ and continuous on an open interval $I$. If $f$ is differentiable at a point $x_{0} \in I$ and $f' \left( x_0 \right) \ne 0$, then $f^{-1}$ is differentiable at $f\left( x_{0} \right)$, and 
    \begin{equation*}
	\left( f^{-1} \right) \left( f\left( x_0 \right) \right) = \frac{1}{f'\left( x_{0} \right)}
    \end{equation*}
    \label{thm:inv-fn-thm-in-1D}
\end{theorem}
 
It follows from this theorem \ref{thm:inv-fn-thm-in-1D} that:

\begin{theorem}[Inverse Function Theorem in $1$-D]
    Let $U$ be an open subset of $\R$, $\varphi : U \to \R$ be a smooth map and $u$ be some point in $U$ such that $\varphi ' \left( u \right) \ne 0$ for some $u\in U$. Then there is an open nbhd $V \subset U$ containing $u$ such that $\varphi \mid _{V} : V \to \varphi \left( V \right)$ is a diffeomorphism.
    \label{def:inv-fn-thm-in-1D-gen-style}
\end{theorem}
\begin{proof}[Proof (sketch).]
    Let $u$ be some point in $U$ such that $\varphi ' \left( u \right) \ne 0$. Since $\varphi$ is smooth, we have that $\varphi ^{'}$ is continuous. Hence by the continuity of $\varphi '$, we have that there is an open interval $V \subset U$ such that $\varphi ' \left( y \right) \ne 0$ for all $y\in V$. Applying the meean value theorem on $\varphi \mid_{V}$, we have that $\varphi \mid _{V}$ is $1-1$ and the theorem follows from the previous theorem.
\end{proof}

\begin{example}[Inverse function theorem does not imply that the function is a diffeomorphism!] Let $U= \R \setminus \left\{ 0 \right\}$.
    Consider the map $\varphi : U \to \R$, $x \stackrel{\varphi}{\mapsto} x^{2}$.  Observe that $\varphi ' \left( u \right) \ne 0$ for all $u \in U$ but $\varphi$ is not a diffeomorphism since $\varphi$ is not injective.
\end{example}

\begin{observation}
    If $\varphi : I \to \R$ is a $\mathcal C ^{1}$ function such that $\varphi ' (u)\ne 0$ for any $u\in I$ then $\varphi : I \to \varphi (I)$ is a diffeomorphism and $\varphi ' (u) \ne 0$ for any $u \in I$. Consequently, if $\varphi$ is smooth then $\varphi ^{\left( k \right)}$ is nonzero on $I$ for any $k\in \mathbb Z_{\ge 0}$.
    \label{obs:smooth-and-non-zero}
\end{observation}
\begin{proof}
    It follows immediately from mean value theorem that $\varphi$ is injective and it follows from Theorem \ref{thm:inv-fn-thm-in-1D} that it is differentiable and cannot be zero anywhere.
\end{proof}

\begin{remark}
    
Consider $I$ to be an open interval and $\varphi : I \to \R$ be a smooth map such that $\varphi ' \left( u \right) \ne 0$ for all $u \in I$. Hence, it follows that $\varphi '$ is injective. Thus, $\varphi : I \to \varphi (I)$ is a diffeomorphism from Theorem \ref{thm:inv-fn-thm-in-1D}.
    \label{rem:a-remark-1}
\end{remark}


\begin{definition}[Reparametrization]
    Let $\gamma : I \to \R ^{n}$ be a smooth curve and $\varphi : J \to I$ be a diffeomorphism where $J$ is an open interval. Define $\beta = \gamma \circ \varphi : J \to \R ^{n}$. Then $\beta$ is a smooth curve (by the Chain Rule) and $\beta$ is called the \textit{reparametrization} of $\gamma$. Note that since $\gamma = \beta \circ \varphi ^{-1}$, we call $\beta$ and $\gamma$ \textit{reparameterizations of each other}.
    \label{def:reparametrization}
\end{definition}

The proof of the following proposition is so easy that it is skipped:
\begin{proposition}
    If $\beta$ and $\gamma$ are reparameterization of each other then $\im \left( \beta \right) = \im \left( \gamma \right)$.
    \label{prop:im-of-reparamterization}
\end{proposition}

\begin{definition}[regular curve]
    Let $\gamma : I \to \R ^{n}$ be a smooth curve. Then $\gamma ' \left( t_0 \right)$ is called the \textit{tangent} of $\gamma$ at $t_{0} \in I$. If $\gamma ' \left( t \right) \ne 0$ for every $t\in I$, we say that $\gamma$ is \textit{regular}.
    \label{def:regular}
\end{definition}

Now, suppose that $\gamma : I \to \R ^{n}$ be a regular curve. We want to a find $\beta : J \to \R^{n}$, reparameterization of $\gamma$ such that $\lVert \beta ' \left( t \right) \rVert = 1$ for all $t\in I$. To achieve this, we define\ldots

\subsection{Arc Length Parameterization}

\begin{definition}
    Let $\gamma : I \to \R ^{n}$ be a regular curve, the \textit{arc length} between $t_1$ and $t_2$ in $I$ is
    \begin{equation*}
	L_{\gamma} \left( t_{1}, t_{2} \right) = \int_{t_{1}}^{t_{2}} \norm{\gamma ' \left( t \right)} \, \mathrm{dt}
    \end{equation*}
    Let us fix $t_{0} \in I$. Define $L_{\gamma} : I \to \R$ by $L_{\gamma} \left( t \right) = \int_{t_{0}}^{t} \norm{\gamma ' \left( x \right)}\, \text{dx}$ for every $t\in I$.
    \label{def:arc-length}
\end{definition}

Now observe that $L_{\gamma} \left( t \right) = \norm{\gamma ^{'} \left( t \right)}$ for every $t\in I$. Since $\gamma$ is regular, we have that $L_{\gamma}'$ is nonzero in $I$ (by Observation \ref{obs:smooth-and-non-zero}). Hence $L_{\gamma}$ is smooth (\textcolor{red}{why?}).

Hence, $L_{\gamma} : I \to L_{\gamma} \left( I \right)$ is a diffeomorphism. Hence, $L_{\gamma} ^{-1} : J \to I$ is smooth where $J := L_{\gamma} \left( I \right)$. Now, $\beta = \gamma \circ L_{\gamma} ^{-1} : J \to \R ^{n}$ is a reparametrization of $\gamma$. Let $S_{\gamma} = L_{\gamma}^{-1}$. Thus for all $s \in J$,
\begin{align*}
    \beta ^{'} \left( s \right) = \gamma ' \left( S_{\gamma} (s) \right) \cdot S_{\gamma} '(s)
\end{align*}
Hence if $s \in S$ then $L_{\gamma} (t) =s$ for some $t\in I$ and hence,
\begin{align*}
    S_{\gamma} ' (s) &= S_{\gamma} ' \left( L_{\gamma} (t) \right)  & \\
    &= \frac{1}{L_{\gamma} ' (t)} & (\text{by Theorem \ref{thm:inv-fn-thm-in-1D}}) \\
    &= \frac{1}{\norm{\gamma ' (t)}} \\
    &= \frac{1}{\gamma ' \left( S_{\gamma} \left( s \right) \right)}
\end{align*}
Hence, we have that 
\begin{equation*}
    \beta ' (s) = \frac{\gamma ' \left( S_{\gamma} (s) \right)}{\norm{\gamma ' \left( S_{\gamma} \left( s \right) \right)}}
\end{equation*}
and
\begin{equation*}
    \norm{\beta ' (s)} =1
\end{equation*}
for all $s \in J$.

This proves the following theorem:
\begin{theorem}
    Let $\gamma$ be a regular curve then there is a parameterization $S_{\gamma} : J \to I$ scuch that 
    \begin{equation*}
	\norm{\beta ' (s)} =1
    \end{equation*}
    for all $s\in J$ where $\beta = \gamma \circ S_{\gamma}$.
    \label{thm:arc-length-parameterization}
\end{theorem}

\begin{definition}
    The parameterization in Theorem \ref{thm:arc-length-parameterization} is called \textit{arc length parameterization}.
    \label{def:arc-length-prmtrztn}
\end{definition}

Now, with the aforemention definition and theorem, we can assume that all regular curves are \textit{unit speed parametrization}.

Let $\gamma : I \to \R ^{3}$ be a regular curve, $\gamma ' (t) \ne 0$ for each $t\in I$. Since$\gamma$ is a unit speed paramterization, we have that $\gamma ' (t) \cdot \gamma ' (t) =0$ for each $t\in I$. By \href{https://math.stackexchange.com/questions/96265/differentiating-an-inner-product}{differentiating} we have that $\gamma ' (t) \cdot \gamma '' (t) = 0$ for each $t\in I$.

Hence, $\gamma '' (t)$ is perpendicular to $\gamma ' (t)$ for each $t\in I$. This begs us to make the following definition:

\begin{definition}[Normal Vector at a point $t$]
    Let $\gamma$ be as in the previous paragraph.
    The unit vector $\hat{\eta} (t)$ be the unit vector in the direction of $\gamma '' (t)$. We call $\hat{\eta} (t)$ is called the \textit{normal vector} at $t$.
    \label{def:normal-vector-at-a-point}
\end{definition}

\begin{proposition}
    The norm function on $\R ^{n}$ is smooth.
    \label{prop:norm-smooth}
\end{proposition}
\begin{proof}
    We first show that $f: \R ^{n} \to \R _{\ge 0}$ given by $f(x)=\langle x , x \rangle$ is smooth. This is easy to see: the projection functions are smooth. Now, note that the result follows immediately from Theorem 1.3.1 of Differential Geometry of Manifolds by Lovett. Now the square root function $\sqrt{\cdot} : \R _{> 0} \to \R _{>0}$ is smooth. Since composition of smooth functions is smooth, we have that $f$ is smooth.
\end{proof}

Now, let $\gamma$ be as in previous paragraph. Then there exist a map $K_{\gamma} : I \to \R$ such that $\gamma '' (t) = K_{\gamma} (t) \hat{\eta} (t)$ for each $t\in I$. We call this function $K_{\gamma}$ as the curvature function of $\gamma$. Observe that $|K_{\gamma} (t)|= \norm{\gamma '' (t)}$ is smooth by Proposition \ref{prop:norm-smooth}.

Observe that if $K_{\gamma}$ is the zero function, then $\gamma '' (t) = 0$ for all $t \in I$ and hence by the Mean value theorem, $\gamma$ must be straight line. 

Now, let $\gamma ' (t) = \hat{t} (t)$. The plane defined by $\left( \hat{t} (t), \hat{\eta} (t) \right)$ is called the \textit{oscillating plane} of $t$.

Let $\hat{b} (t) = \hat{t} (t) \times \hat{\eta} (t)$. Then
\begin{align*}
    \hat{b} '(t) &= \hat{t} ' (t) \times \hat{\eta} (t) + \hat{t} (t) \times \hat{\eta} ' (t) \\
    &= \gamma '' (t) \times \hat{\eta} (t) + \hat{t} (t) \times \hat{\eta} ' (t) \\
    &= 0 +\hat{t} (t) \times \hat{\eta} ' (t)
\end{align*}

Thus, $\hat{b} ' (t) $ is perpendicular to $\hat{t} (t)$. Hence, $\hat{b} '(t)$ is perpendicular to $\hat{b} (t)$ since $\hat{b} (t) \cdot \hat{b} (t) =1$. Thus, $\hat{b} ' (t) = \tau (t) \hat{\eta} (t)$ for some $\tau (t) \in \R$. Thus, we have a smooth map $\tau : I \to \R$.
