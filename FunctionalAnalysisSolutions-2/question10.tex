\section{Question 10}
\horz
Let $X$ be a normed linear space and $F: X \to \mathbb C$ be a continuous, non zero linear functional. 
\begin{itemize}
\item[(a)] Show that $ N(F)=\sup\Big\{ \frac{|F(x)|}{\|x\|} : x \in X, x \neq 0\Big\} < \infty.$
\item[(b)] Suppose $M = \ker F$ and $x_0 \notin M.$ Show that $d(x_0, M) = \frac{|F(x_0)|}{N(F)}.$
\end{itemize}
\horz

\begin{proof}[Solution]
    Since $F$ is continuous, we have that there must be some $M > 0$ such that 
    \begin{align*}
	\abs{F(x)} \le M \norm{x} \text{ for all } x \in X\text{.} 
    \end{align*}
    Thus, we have that 
    \begin{align*}
	\frac{\abs{F(x)}}{\norm{x}} \le M \text{ for all } x \in X\text{.} 
    \end{align*}
    This shows that $N\left( F \right) < \infty$. \footnote{One can easily check that $N(F)$ is $\norm{F}_{X^*}$ in disguise!}

    Now, let $M=\ker F$. Observe that by definition of quotient norm, we have that 
    \begin{align*}
	\norm{[x]}_{X/M} = \inf_{m\in M} \norm{x-m}_{M} = d\left( x, M \right)
    \end{align*}
    for every $x\in X$.
    Therefore, we need to show that $\abs{F\left( x_{0} \right)} = N(F) \norm{[x_{0}]}_{X/M}$. Let $x\in M$ be arbitrary. Then by Question 8, we have that $x=y + \lambda x_{0}$ for some $y \in \ker F$ and some $\lambda \in \C$. In case, $\lambda = 0$, we have that 
    \begin{align*}
\frac{\abs{F\left( x \right)}}{\norm{x}} &= \frac{F\left( y\right)}{\norm{y}} \\
&= 0 \le \frac{\abs{F\left( x_{0} \right)}}{\norm{[x_{0}]}_{X/M}}
    \end{align*}
    Else if $\lambda \ne 0$, we have that
    \begin{align*}
	\frac{\abs{F\left( x \right)}}{\norm{x}} &=  \frac{\abs{F\left( y+ \lambda x_{0} \right)}}{\norm{\lambda x_{0} + y}} \\
	&= \frac{\abs{F\left( x_{0} \right)}}{\norm{x_{0} + \frac{1}{\lambda}y}}
\le \frac{\abs{F\left( x_{0} \right)}}{\norm{[x_{0}]}_{X/M}}
    \end{align*}
    Since $x\in X$ was arbitrary, taking supremum, we have
    \begin{equation}
	N(F) \le \frac{\abs{F(x_{0})}}{\norm{x_{0}}_{X/M}} \leadsto N(F) \norm{x_{0}}_{X/M} \le \abs{F\left( x_{0} \right)}\text{.}
	\label{eqn:q10-1}
    \end{equation}
     To prove the reverse inequality, observe that if $y\in M$ then
     \begin{align*}
	 \frac{\abs{F\left( x_{0} \right)}}{\norm{x_{0}-m}} \le N(F) &\leadsto \frac{\abs{F\left( x_{0} \right)}}{N(F)} \le \norm{x_{0}-m}\text{.}
     \end{align*}
     Since $m$ is arbitrary, we have that
     \begin{align}
	 \frac{\abs{F\left( x_{0} \right)}}{N(F)} \le \norm{[x_{0}]}_{X/M}
	 \leadsto \abs{F\left( x_{0} \right)} \le N(F)  \norm{[x_{0}]}_{X/M}\text{.}
	 \label{eqn:q10-2}
     \end{align}
     Combining \ref{eqn:q10-1} and \ref{eqn:q10-2}, we have what we wanted.
\end{proof}
