\section{Question 11 \texorpdfstring{$\checkmark$}{\text{✓}}}
\horz
Let $X$ be a finite dimensional normed linear space and $M$ be a proper closed subspace of $X$. Show that the unit sphere  $S := \{x: \|x\|=1\}$ on $X$ is compact. Use this to show that there exist a unit vector $x$ such that $\text{dist}(x, M) = 1.$ This need not to be true if $X$ is infinite dimensional. Show that the choice
    \begin{align*}
       & X = \{f \in C[0, 1] : f(0) = 0\}\\
        &M= \{f \in X : \int_{0}^{1}f = 0\}
    \end{align*}
provides a counter example. (This also shows that in F. Riesz's Lemma the constant $t$ can not be taken to be equal to 1 in general.)
\horz

\begin{proof}
    Note that $S$ is compact in view of Question 6.
    Consider the map $f: S \to \C$ given by 
    \begin{equation*}
	x \stackrel{f}{\mapsto} d\left( x, M \right)
    \end{equation*}
    Since we have $\abs{d\left( x,M \right) -d\left( y,M \right)} \le d\left( x,y \right)$, $f$ is continuous. We now show that $\sup f\left( S \right) = 1$.
    Let $x\in S$. Then we have that
    \begin{align*}
	d(x,M) &= \inf_{y \in M} \norm{x-y} \\
	&\le \norm{x}=1\text{.} \\
    \end{align*}
    Thus, $\sup f\left( S \right) \le 1$. Now, let $\varepsilon >0$ be arbitrary. Then by F. Riesz lemma, there exists a vector $x_{0} \in M$ such that 
    \begin{align*}
	1-\varepsilon < \norm{y-x_{0}} \text{for all } y \in M &\leadsto 1-\varepsilon \le \inf_{y\in M} \norm{y-x_{0}}=d(x_{0},M) \\
	& \leadsto 1\le \sup f(S) + \varepsilon
    \end{align*}
    Since $\varepsilon > 0$ is arbitrary, we have that $\sup f\left( S \right) \ge 1$. This shows that $\sup f\left( S \right) =1$.

    Now, continuous functions on compact sets achieve their supremum, therefore, there must be some vector $x \in S$ such that $f(x)=d(x,M) =1$.

    Now, we move on to the next part of the question. Consider the following sets:
    \begin{align*}
	X&= \left\{ f \in C[0,1] : f(0)=0 \right\} \\
	M&= \left\{ f\in X : \int_{0}^{1} f = 0 \right\}
    \end{align*}
    We need to show that there is no vector $f\in X$ whose $\norm{f}_{\infty}=1$ but $d(f,M)=1$. Assume the contrary and suppose that such an $f$ exists.

    Consider the linear map $F : X \to \R$ given by
    \begin{align*}
	T(g)=\int_{0}^{1} g
    \end{align*}
    It is evident that $\ker F =M$. Hence, by Question 10, we have that
    \begin{equation*}
	\abs{T(g)}=d\left( g, M \right) \norm{T}
    \end{equation*}
    holds for every $g\in X$. (In case, $g\in M$ then both sides are zero!)
    
    Hence, we have, in particular, that 
    \begin{equation*}
	\abs{T(f)}=\norm{T}
    \end{equation*}
    
    Now, we claim that $\norm{T}=1$. To prove this, let $f\in X$ with $\norm{f}_{\infty}\le1$. Then we have that
    \begin{align*}
	\abs{T\left( f \right)} &=  \abs{\int_{0}^{1} f} \\
	&\le \int_{0}^{1} \abs{f} \\
	& \le \norm{f}_{\infty} \le 1\text{.}
    \end{align*}
    This shows that $\norm{T} \le 1$. To show the reverse inequality, let $\varepsilon > 0$. It is not hard to draw a "trapezoid like" function whose area is larger than $1-\varepsilon$. This shows that $\norm{T}=1$.

    Thus, we have that $\abs{T(f)}=\abs{\int_{0}^{1} f} = 1 = \norm{f}_{\infty}$. Consider the following:
    \begin{align*}
	1=\abs{\int_{0}^{1} f } \le \int_{0}^{1} \abs{f} \le \norm{f}_{\infty} =1\text{.}
    \end{align*}
    This shows that $\int_{0}^{1} \abs{f}=1$ and $\norm{f}_{\infty}=1$. Now, we show that $\abs{f} \equiv 1$ on $[0,1]$. If we do so, we will be done because then $f$ cannot be in $X$ as $\abs{f\left( 0 \right)}=1 \ne 0$.

    To show that $\abs{f}\equiv 1$ on [0,1], we define a function $g : \left[ 0,1 \right]$ given by $g=\norm{f}_{\infty} - \abs{f} = 1 -f$. Observe that $g$ is nonnegative and $\int g =0$ and hence $g\equiv 0 \leadsto \abs{f}\equiv 1$ on $[0,1]$.

    This completes the proof.
\end{proof}
