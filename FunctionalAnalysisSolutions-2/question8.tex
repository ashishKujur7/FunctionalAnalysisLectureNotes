\section{Question 8 \texorpdfstring{$\checkmark$}{\text{✓}}}
\horz
Let $X$ be a normed linear space and $F: X \to \mathbb C$ be a non zero linear functional. Suppose $F(x_0) \neq 0$ for some $x_0\in X.$ Show that $X= \ker F \oplus \mbox{span} \{x_0\},$ that is,
\begin{itemize}
\item[(i)] $\ker F \cap \mbox{span} \{x_0\} = \{0\}.$
\item[(ii)] $ X = \ker F + \mbox{span} \{x_0\}.$
\end{itemize}
Show that $F$ is continuous if and only if $\ker F$ is a closed subspace in $X.$ (Hint : Use the continuity of the projection map $\pi:X \to  X/\ker F$ defined by $\pi(x) = [x],\,x\in X.$ )
\horz
\begin{proof}
    Let $X$ be a normed linear space and $F: X \to \C$ be nonzero linear functional. Since $F$ is nonzero, there must be some $x_{0} \in X$ such that $F\left( x_{0} \right) \ne 0$. We now proceed to show that $X=\ker F \oplus \operatorname{span} \left\{ x_{0} \right\}$.

    We first show that $X=\ker F + \operatorname {span} \left\{ x_{0} \right\}$. Let $x\in X$. Then $F\left( x \right) \in \C$. Since $F\left( x_{0} \right) \ne 0$. There must be some $\lambda \in \C$ such that $F\left( x \right) = \lambda F \left( x_{0} \right)$. Thus, we have that $F\left( x-\lambda x_{0} \right) = 0$. Thus, $x-\lambda x_{0}  \in \ker F$. Hence, $x= \lambda x_{0} + y$ for some $y \in \ker F$. This shows that $X=\ker F + \operatorname{span} \left\{ x_{0} \right\}$. 

    Now, we proceed to show that $\ker F \cap \operatorname{span} \left\{ x_{0} \right\} = \left\{ 0 \right\}$. To do so, let $y\in \ker F \cap \operatorname{span } \left\{ x_{0} \right\}$. Then we have thath $y= \lambda x_{0}$ for some $\lambda \in \C$. Hence, we have that $F\left( y \right) = \lambda F\left( x_{0} \right) = 0$. Since $F\left( x_0 \right) \ne 0$, we have that $\lambda = 0$ and thus, $y=0$. This completes the proof of the claim.

    The above two paragraphs show that $X = \ker F \oplus \operatorname{span} \left\{ x_{0} \right\}$.

    Now, we proceed to show that $F$ is continuous iff $\ker F$ is a closed subspace of $X$. Let's begin the proof in the $\left( \Rightarrow \right)$ direction. Suppose that $F$ is continuous. Then we have that $\ker F = F^{-1} \left( \left\{ 0 \right\} \right)$ and hence it must be closed.
    
    To show the reverse direction, namely $\left( \Leftarrow \right)$, we first show that the projection map is continuous. First, we observe that for any $x\in X$, we have that
    \begin{align*}
	\norm{[x]} &= \inf_{y\in \ker F} \norm{x-y} & \text{by definition} \\
	&\le \norm{x} & 0 \in \ker F
    \end{align*}
    Now, this shows that the projection map $\pi : X \to X/\ker F$ is bounded and since it is a linear map, it is continuous.

    Now, consider the map $\tilde{T} : X / \ker F \to \C$ given by
    \begin{align*}
	[x] \stackrel{\tilde{T}}{\mapsto} F\left( x \right)
    \end{align*}
    We showed that $X= \ker F \oplus \operatorname{span} \left\{ x_{0} \right\}$. By the first isomorphism theorem for vector spaces, we have that $X/\ker F \cong \operatorname{span} \left\{ x_{0} \right\}$. This shows that $X/\ker F$ is finite dimensional. Since $\tilde{T}$ is linear and $X/\ker F$ is finite dimensional, we have thath $\tilde{T}$ is continuous.

    Observe that $T = \tilde{T} \circ \pi : X \to \C$ is continuous linear functional by virtue of being composition of two continuous linear maps. This completes the proof.
\end{proof}
