\section{Question 17}

Assume $\left( V, \norm{} \right)$ is a real normed linear space which satisfies the parallelogram identity, that is, for all $a,b \in V$, 
\begin{equation*}
    \norm{a+b}^{2} + \norm{a-b}^{2} = 2\norm{a}^{2} +2 \norm{b}^{2}
\end{equation*}

We intend to define the inner product on $V$ by
\begin{align*}
    \ip{v,w} = \frac{\norm{x+y}^{2} - \norm{x-y}^{2}}{4}
\end{align*}

We show that $\ip{\cdot, \cdot}$ is an inner product.

The symmetric property is evident.

We proceed to show linearity in the first variable, that is, we need to show that 
\begin{align*}
    \norm{x_{1} + x_{2} + y}^{2} - \norm{x_{1}+x_{2}-y}^{2} = \norm{x_{1} + y}^2-\norm{x_{1}-y}^{2} + \norm{x_{2}+y}^{2} - \norm{x_{2}-y}^{2}
\end{align*}

Setting $a=x_{1} $ and $b=x_{2}+y$ in the parallelogram identity, we get
\begin{align*}
    \norm{x_{1} + y + x_{2}}^{2} + \norm{x_{1} - y - x_{2}}^{2} = 2 \norm{x_{1} } ^{2} + 2 \norm{x_{2} + y}^{2}
\end{align*}

Doing the same for $a=x_{2} -y $ and $b=x_{2}$, we have
\begin{align*}
    \norm{x_{1} - y + x_{2}}^{2} + \norm{x_{1} - y - x_{2}}^{2} = 2 \norm{x_{1} -y} ^{2} + 2 \norm{x_{2}}^{2}
\end{align*}

Subtracting the above two equations, we get
\begin{align*}
    \norm{x_{1} + x_{2} + y}^{2} - \norm{x_{1} - y + x_{2}} ^{2} = 2\norm{x_{1}}^{2} + 2 \norm{x_{2} + y} ^{2} - 2 \norm{x_{1} - y}^{2} - 2 \norm{x_{2}}^{2}
\end{align*}
 
Switching the roles of $x_{2}$ and $x_{1}$, we get
\begin{align*}
    \norm{x_{2} + x_{1} + y}^{2} - \norm{x_{2} - y + x_{1}} ^{2} = 2\norm{x_{2}}^{2} + 2 \norm{x_{1} + y} ^{2} - 2 \norm{x_{2} - y}^{2} - 2 \norm{x_{1}}^{2}
\end{align*}

Adding the above two equations, we get
\begin{align*}
    2\norm{x_{1} + x_{2} + y}^{2} - 2\norm{x_{1} +x_{2} -y }^{2} = 2 \norm{x_{2} + y}^{2} - 2 \norm{x_{1} -y}^{2} + 2 \norm{x_{1} + y} ^{2} -2 \norm{x_{2}-y}^{2}
\end{align*}

Rearranging the above equation, we observe that we have established what we wanted to prove!

Linearity is the other variable follows by symmetry and the linearity in the first variable!

Now, finally we proceed to show that for any $\lambda \in \R$, we have that 
\begin{align*}
    \ip{\lambda x, y} = \lambda \ip {x,y}
\end{align*}

Note that by linearity in the first variable, we have that for $n\in \Z$,
\begin{equation*}
    \ip{nx,y} = n \ip{x,y}
\end{equation*}
In a similar fashion, it can be shown that for $r\in \Q$,

\begin{equation*}
    \ip{rx,y} = r \ip{x,y}
\end{equation*}

Let us assume \textcolor{red}{Cauchy-Schwarz!} at the moment. Let $r\in \R$. Let $r_{n}$ be a sequence of rationals converging to $r\in \R$.

Observe that fixing $y\in V$, it is easily seen that 
\begin{equation*}
    x \mapsto \frac{\norm{x+y}^{2} - \norm{x-y}^{2}}{4}
\end{equation*}
is continuous by virtue of translation, norm and square of a function being continuous!
Then the result follows!

Irregardless, we prove Cauchy Schwarz! It can be seen by minimizing $r$ is the function $r \mapsto \norm{rx + y}^{2}$. One needs to see that for $r\in \Q$
\begin{align*}
    \norm{rx+y}^{2} = \ip{rx+y , rx +y}
    = r^{2} \norm{x}^{2} + 2r\ip{x,y} + \norm{y}^{2} \ge 0
\end{align*}
Hence the above holds for any $r\in \R$ by taking limits. Minimizing the function, we get the Cauchy Schwarz inequality.


We now proceed to the complex case!

Let $V, \norm{}$ be a complex normed linear space. By the polarization identity, we have that for $x, y \in V$,
\begin{align*}
    \ip{x,y} &= \frac{1}{4} \sum_{k=1}^{4} \norm{x+i^{k}y}^{2} i^{k} \\
    &= \frac{\norm{x+y}^{2} - \norm{x-y}^{2}}{4} - \frac{\norm{x+iy}^{2} - \norm{x-iy}^{2}}{4} \\
    &= q(x,y)+i q(x,iy)
\end{align*}
where $q(x,y)=  \frac{\norm{x+y}^{2} - \norm{x-y}^{2}}{4}$. We have already shown that $q$ is an inner product over $\R$. 

Now one can use the properties of inner product for $q$ to show that $\ip{\cdot , \cdot}$ is an inner product over $\C$.

