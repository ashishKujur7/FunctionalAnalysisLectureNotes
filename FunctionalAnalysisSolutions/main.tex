\documentclass[12pt]{article}
\usepackage[margin=1in]{geometry}
\usepackage{amsfonts, amsmath}
\usepackage[T1]{fontenc}
\usepackage{mathrsfs, enumitem}
\usepackage{hyperref}
\usepackage[utf8]{inputenc}
\usepackage{amssymb}
\usepackage{amsfonts}
\usepackage{amsmath}
\usepackage{amsthm}
\usepackage{color}
\usepackage{hyperref}
\usepackage{csquotes}
%\usepackage{fourier}
\usepackage{tikz-cd}
\usepackage{lipsum}
\usepackage{cancel}

\newtheorem{theorem}{Theorem}[subsection]
\newtheorem{lemma}[theorem]{Lemma}
\newtheorem{claim}[theorem]{Claim}
\newtheorem{proposition}[theorem]{Proposition}
\newtheorem{corollary}[theorem]{Corollary}
\newtheorem{fact}[theorem]{Fact}
\newtheorem{notation}[theorem]{Notation}
\newtheorem{observation}[theorem]{Observation}
\newtheorem{conjecture}[theorem]{Conjecture}
\newtheorem{exercise}[theorem]{Exercise}

\theoremstyle{definition}
\newtheorem{definition}[theorem]{Definition}
\newtheorem{example}[theorem]{Example}
\numberwithin{equation}{subsection}

\theoremstyle{remark}
\newtheorem{remark}[theorem]{Remark}
\theoremstyle{plain}
\newcommand{\ignore}[1]{}

% section symbol
%\renewcommand{\thesection}{\S\arabic{section}}

% \renewcommand{\Pr}{{\bf Pr}}
% \newcommand{\Prx}{\mathop{\bf Pr\/}}
% \newcommand{\E}{{\bf E}}
% \newcommand{\Ex}{\mathop{\bf E\/}}
% \newcommand{\Var}{{\bf Var}}
% \newcommand{\Varx}{\mathop{\bf Var\/}}
% \newcommand{\Cov}{{\bf Cov}}
% \newcommand{\Covx}{\mathop{\bf Cov\/}}

% shortcuts for symbol names that are too long to type
\newcommand{\eps}{\epsilon}
\newcommand{\lam}{\lambda}
\renewcommand{\l}{\ell}
\newcommand{\la}{\langle}
\newcommand{\ra}{\rangle}
\newcommand{\wh}{\widehat}
\newcommand{\wt}{\widetilde}

% % "blackboard-fonted" letters for the reals, naturals etc.
\newcommand{\R}{\mathbb R}
\newcommand{\N}{\mathbb N}
\newcommand{\Z}{\mathbb Z}
\newcommand{\F}{\mathbb F}
\newcommand{\Q}{\mathbb Q}
\newcommand{\C}{\mathbb C}

% % operators that should be typeset in Roman font
% \newcommand{\poly}{\mathrm{poly}}
% \newcommand{\polylog}{\mathrm{polylog}}
% \newcommand{\sgn}{\mathrm{sgn}}
% \newcommand{\avg}{\mathop{\mathrm{avg}}}
% \newcommand{\val}{{\mathrm{val}}}

% % complexity classes
% \renewcommand{\P}{\mathrm{P}}
% \newcommand{\NP}{\mathrm{NP}}
% \newcommand{\BPP}{\mathrm{BPP}}
% \newcommand{\DTIME}{\mathrm{DTIME}}
% \newcommand{\ZPTIME}{\mathrm{ZPTIME}}
% \newcommand{\BPTIME}{\mathrm{BPTIME}}
% \newcommand{\NTIME}{\mathrm{NTIME}}

% values associated to optimization algorithm instances
\newcommand{\Opt}{{\mathsf{Opt}}}
\newcommand{\Alg}{{\mathsf{Alg}}}
\newcommand{\Lp}{{\mathsf{Lp}}}
\newcommand{\Sdp}{{\mathsf{Sdp}}}
\newcommand{\Exp}{{\mathsf{Exp}}}

% if you think the sum and product signs are too big in your math mode; x convention
% as in the probability operators
\newcommand{\littlesum}{{\textstyle \sum}}
\newcommand{\littlesumx}{\mathop{{\textstyle \sum}}}
\newcommand{\littleprod}{{\textstyle \prod}}
\newcommand{\littleprodx}{\mathop{{\textstyle \prod}}}

% horizontal line across the page
\newcommand{\horz}{
\vspace{-.4in}
\begin{center}
\begin{tabular}{p{\textwidth}}\\
\hline
\end{tabular}
\end{center}
}

% calligraphic letters
\newcommand{\calA}{{\cal A}}
\newcommand{\calB}{{\cal B}}
\newcommand{\calC}{{\cal C}}
\newcommand{\calD}{{\cal D}}
\newcommand{\calE}{{\cal E}}
\newcommand{\calF}{{\cal F}}
\newcommand{\calG}{{\cal G}}
\newcommand{\calH}{{\cal H}}
\newcommand{\calI}{{\cal I}}
\newcommand{\calJ}{{\cal J}}
\newcommand{\calK}{{\cal K}}
\newcommand{\calL}{{\cal L}}
\newcommand{\calM}{{\cal M}}
\newcommand{\calN}{{\cal N}}
\newcommand{\calO}{{\cal O}}
\newcommand{\calP}{{\cal P}}
\newcommand{\calQ}{{\cal Q}}
\newcommand{\calR}{{\cal R}}
\newcommand{\calS}{{\cal S}}
\newcommand{\calT}{{\cal T}}
\newcommand{\calU}{{\cal U}}
\newcommand{\calV}{{\cal V}}
\newcommand{\calW}{{\cal W}}
\newcommand{\calX}{{\cal X}}
\newcommand{\calY}{{\cal Y}}
\newcommand{\calZ}{{\cal Z}}

% bold letters (useful for random variables)
%----------------------------------------------
% \renewcommand{\a}{{\boldsymbol a}}
% \renewcommand{\b}{{\boldsymbol b}}
% \renewcommand{\c}{{\boldsymbol c}}
% \renewcommand{\d}{{\boldsymbol d}}
% \newcommand{\e}{{\boldsymbol e}}
% \newcommand{\f}{{\boldsymbol f}}
% \newcommand{\g}{{\boldsymbol g}}
% \newcommand{\h}{{\boldsymbol h}}
% \renewcommand{\i}{{\boldsymbol i}}
% \renewcommand{\j}{{\boldsymbol j}}
% \renewcommand{\k}{{\boldsymbol k}}
% \newcommand{\m}{{\boldsymbol m}}
% \newcommand{\n}{{\boldsymbol n}}
% \renewcommand{\o}{{\boldsymbol o}}
% \newcommand{\p}{{\boldsymbol p}}
% \newcommand{\q}{{\boldsymbol q}}
% \renewcommand{\r}{{\boldsymbol r}}
% \newcommand{\s}{{\boldsymbol s}}
% \renewcommand{\t}{{\boldsymbol t}}
% \renewcommand{\u}{{\boldsymbol u}}
% \renewcommand{\v}{{\boldsymbol v}}
% \newcommand{\w}{{\boldsymbol w}}
% \newcommand{\x}{{\boldsymbol x}}
% \newcommand{\y}{{\boldsymbol y}}
% \newcommand{\z}{{\boldsymbol z}}
% \newcommand{\A}{{\boldsymbol A}}
% \newcommand{\B}{{\boldsymbol B}}
% \newcommand{\C}{{\boldsymbol C}}
% \newcommand{\D}{{\boldsymbol D}}
% \newcommand{\E}{{\boldsymbol E}}
% \newcommand{\F}{{\boldsymbol F}}
% \newcommand{\G}{{\boldsymbol G}}
% \renewcommand{\H}{{\boldsymbol H}}
% \newcommand{\I}{{\boldsymbol I}}
% \newcommand{\J}{{\boldsymbol J}}
% \newcommand{\K}{{\boldsymbol K}}
% \renewcommand{\L}{{\boldsymbol L}}
% \newcommand{\M}{{\boldsymbol M}}
% \renewcommand{\O}{{\boldsymbol O}}
% \renewcommand{\P}{{\mathbb{P}}}
% \newcommand{\Q}{{\boldsymbol Q}}
% \newcommand{\R}{{\boldsymbol R}}
% \renewcommand{\S}{{\boldsymbol S}}
% \newcommand{\T}{{\boldsymbol T}}
% \newcommand{\U}{{\boldsymbol U}}
% \newcommand{\V}{{\boldsymbol V}}
% \newcommand{\W}{{\boldsymbol W}}
% \newcommand{\X}{{\boldsymbol X}}
% \newcommand{\Y}{{\boldsymbol Y}}
% \newcommand{\Z}{{\boldsymbol Z}}

% script letters
\newcommand{\scrA}{{\mathscr A}}
\newcommand{\scrB}{{\mathscr B}}
\newcommand{\scrC}{{\mathscr C}}
\newcommand{\scrD}{{\mathscr D}}
\newcommand{\scrE}{{\mathscr E}}
\newcommand{\scrF}{{\mathscr F}}
\newcommand{\scrG}{{\mathscr G}}
\newcommand{\scrH}{{\mathscr H}}
\newcommand{\scrI}{{\mathscr I}}
\newcommand{\scrJ}{{\mathscr J}}
\newcommand{\scrK}{{\mathscr K}}
\newcommand{\scrL}{{\mathscr L}}
\newcommand{\scrM}{{\mathscr M}}
\newcommand{\scrN}{{\mathscr N}}
\newcommand{\scrO}{{\mathscr O}}
\newcommand{\scrP}{{\mathscr P}}
\newcommand{\scrQ}{{\mathscr Q}}
\newcommand{\scrR}{{\mathscr R}}
\newcommand{\scrS}{{\mathscr S}}
\newcommand{\scrT}{{\mathscr T}}
\newcommand{\scrU}{{\mathscr U}}
\newcommand{\scrV}{{\mathscr V}}
\newcommand{\scrW}{{\mathscr W}}
\newcommand{\scrX}{{\mathscr X}}
\newcommand{\scrY}{{\mathscr Y}}
\newcommand{\scrZ}{{\mathscr Z}}

\newcommand{\im}{{\text{im }}}
\newcommand{\ip}[1]{\left\langle #1 \right\rangle}
\newcommand{\norm}[1]{\left\lVert #1 \right\rVert}
\newcommand{\abs}[1]{\left\lvert #1 \right\rvert}

\newcommand\blfootnote[1]{%
  \begingroup
  \renewcommand\thefootnote{}\footnote{#1}%
  \addtocounter{footnote}{-1}%
  \endgroup
}

\title{Solutions to Functional Analysis Assignment 1}
\author{\textsc{Ashish Kujur}}

\date{}

\begin{document}

\maketitle
\section*{Note}
A checkmark $\checkmark$ indicates the question has been done.
\tableofcontents


\section{Question 1 \texorpdfstring{\checkmark}{\text{✓}}}
\horz
Let $(V,\langle \cdot, \cdot \rangle)$ be an inner product space and $x,y$ be two non zero vector in $V.$ 
Show that $\|x+y\| = \|x\|+\|y\|$ holds if and only if $x=cy$ for some scalar $c >0.$
\horz

\begin{proof}[Solution]
    Let $x,y$ be two nonzero vectors in an inner product space $V$. Consider the following equivalences:
    \begin{align*}
	\norm{x+y} = \norm{x} + \norm{y} &\Longleftrightarrow  \norm{x+y}^{2} = \left( \norm{x} + \norm{y} \right)^{2} \\
	&\Longleftrightarrow \ip{x+y, x+y} = \norm{x}^{2} + \norm{y}^2 + 2 \norm{x} \norm{y} \\
	&\Longleftrightarrow \Re \ip{x,y} = \norm{x} \norm{y}
    \end{align*}

    Suppose that $\norm{x+y} = \norm{x} + \norm{y}$ holds. Then we have from the above equivalence that $\Re \ip{x,y} = \norm{x} \norm{y}$. Since $\norm{x}\norm{y} \le \Re \ip{x,y} \le \abs{\ip{x,y}} \le \norm{x}\norm{y}$, we have that $\ip{x,y}  = \norm{x}\norm{y}$. Since the equality in Cauchy Schwarz inequaliy holds iff $x$ and $y$ are linearly dependent, we must have that $x=cy$ for some $c\in \C$. Thus, we must have that
    \begin{align*}
	\Re \ip{cy, y} = \norm{cy}\norm{y} &\Longleftrightarrow \Re{c \ip{y,y}} = \abs{c} \norm{y}\norm{y} \\
	&\Longleftrightarrow \ip{y,y} \Re c = \abs{c} \norm{y}^{2} \\
	&\Longleftrightarrow \Re{c} = \abs{c} \\
	&\Longleftrightarrow c>0
    \end{align*}

    The argument is reversible and the proof is complete!
\end{proof}

\section{Question 2 \texorpdfstring{$\checkmark$}{\text{✓}}}
\horz
Let $\{f_n\}_{n\in\mathbb N}$ be the sequence of function in $\mathscr C[0,1]$ given  by $f_n(x)=x^n,\,x\in[0,1],\,n\in\mathbb N.$ Let $d_1$ and $d_{\infty}$ be the metric induced by $\|\cdot\|_1$ and $\|\cdot\|_{\infty}$ as discussed in the previous question, that is, $d_1(f,g)=\|f-g\|_p$ and $d_{\infty}(f,g)=\|f-g\|_{\infty},\,\,f,g\in\mathscr C[0,1].$ Show that the sequence $d_1(f_n,0) \to 0$ as $n\to \infty$ but  $d_{\infty}(f_n,0)$ does not tends to $0$ as $n\to \infty.$ Show that even $\{f_n\}_{n\in\mathbb N}$ has no convergent subsequence in the metric space $(\mathscr C[0,1], d_{\infty}).$ Can you conclude from the above that $(\mathscr C[0,1], d_{\infty})$ is not equivalent to $(\mathscr C[0,1], d_{1})?$ Show that $(\mathscr C[0,1], d_{2})$ is not equivalent to $(\mathscr C[0,1], d_{1})?$
\horz

\begin{proof}[Solution]
    We first show that $f_{n} := x^{n}$ converge to $0$ in the $1$-norm. This is easy to see:
    \begin{align*}
	\norm{f_{n}-0}_{1} &= \int_{0}^{1} \abs{f_{n}} dx \\
	&= \int_{0}^{1} x^{n} dx \\
	&= \frac{1}{n+1} \to 0 \text{ as } n\to \infty 
    \end{align*}
    We now show that $f_{n}$ has no convergent subsequence. For the sake of contradiction, let $\left\{ f_{n_{k}} \right\}$ be sequence which converges uniformly (same as convergence in sup norm) to some $f\in C[0,1]$. To obtain contradiction, we make use of the following fact:
    \begin{center}
	\textit{If $f_{n}: X \to \C$ is a sequence of function which converge to $f: X \to \C$ uniformly and $\left\{ x_{n} \right\}$ is a sequence in $X$ which converges to $x\in X$ then $\left\{ f_{n} \left( x_{n} \right) \right\}$ converges to $f\left( x \right)$.}
    \end{center}

    Now, since uniform convergence is stronger than pointwise convergence, using the above fact, we have that
	\begin{align*}
	    f(1) = \lim_{n\to \infty} f_{n_{k}} \left( 1 \right) = 1
	\end{align*}
	Also, since $\left( 1-1/n \right)$ converges to $1$, we have that
	\begin{align*}
	f(1)= \lim_{k\to \infty} f_{n_{k}} \left( 1-1/n_{k} \right) &=  \left( 1- \frac{1}{n_k} \right)^{n_{k}}= \frac{1}{e}
	\end{align*}
	 It is obvious to see now that that the two metrics are not equivalent.
\end{proof}

\section{Question 3 \texorpdfstring{$\checkmark$}{\text{✓}}}
\horz
Let $(X,\|\cdot\|)$ be a normed linear space (in short NLS) and $(X,d)$ be the associated metric, that is, $d(a,b)=\|a-b\|,\,\,a,b\in X.$ Show that a ball $B_d(a,r)$ is always a convex subset of $X.$ 
\horz
\begin{proof}[Solution]
    It suffices to show that $B(0,1)$ is convex because every other ball is just this (modulo translations and dilations).
    We proceed to show that $B(0,1)$ is convex. Let $x,y \in B(0,1)$. Then we have that for $t\in[0,1]$,
    \begin{align*}
	\norm{tx+(1-t)y} &= t \norm{x} + (1-t)\norm{y} \\
	&< t+(1-t) \\
	&= 1
    \end{align*}
    And we're done.
\end{proof}


\section{Question 4 \texorpdfstring{$\checkmark$}{\text{✓}}}
\horz
Let $M$ be a subspace of an inner product space $(V,\langle \cdot, \cdot \rangle).$ Show that $\overline{M},$ the closure of $M,$ in $V$ is also a subspace. Moreover show that $M^{\perp}= {\overline{M}}^{\perp}.$
\horz
\begin{proof}[Solution] We make the following claims:

\begin{claim}[orthogonal complement of a set and the orthogonal completement of its closure are same!]
    Let $M$ be a subset of a inner product space $H$. Then $M^{\perp} =  \left( \overline{M} \right)^{\perp}$
\end{claim}
\begin{proof}
    It follows by definition that $M \subset \overline M$ and hence $\left( \overline M \right)^{\perp} \subset M^{\perp}$. Now for reverse the inclusion, let $v \in M^{\perp}$ and let $y \in \overline M$. We need to show that $\ip{v,y} = 0$. Since $y\in \overline M$ there is a sequence $\left( y_{n} \right)$ in $M$ such that $y_{n} \to y$. Since $v \in M^{\perp}$, we have that $\ip {v , y_{n}}=0 $ for all $n\in \N$. Since $\ip{v, y_{n}} \to \ip{v,y}$, we have by uniqueness of limits that $\ip{v,y} = 0$. This completes the proof.
\end{proof}

\begin{claim}[orthogonal complement of orthogonal complement]
    
    Let $M$ be a closed subspace of the Hilbert space $H$. Then 
    \begin{align*}
	M = \left( M^{\perp} \right) ^{\perp}
    \end{align*}
    
\label{claim:o-comp-squared}
\end{claim}
\begin{proof}[Proof of Claim]
    Let us first show that $M \subset \left( M ^{\perp} \right)^{\perp}$ (which in fact holds for any set $M$). Let $v \in M$ and $w \in M^{\perp}$. It is clear by definition of $M^{\perp}$ that $\ip{v,w} = 0$. Hence, $v \in \left( M^{\perp} \right)^{\perp}$.

    Let us proceed to show the inclusion in the other direction. Let $v \in \left( M^{\perp} \right)^{\perp}$. Since $M$ is closed, by Projection Theorem, we have that $v = Pv + Qv$ where $Pv \in M$ and $Qv \in M^{\perp}$. By the previous paragraph, we have that $M \subset \left( M^{\perp} \right)^{\perp}$ and hence $Pv \in \left( M^{\perp} \right) ^{\perp}$. Hence, we have that $Qv \in \left( M^{\perp} \right)^{\perp}$. Now, $Qv \in M^{\perp} \cap \left( M^{\perp} \right) ^{\perp}$. Hence, $Qx = 0$ and thus, $v=Pv \in M$.
\end{proof}

Now, we start the proof. Let $M$ be subspace of $V$. Consider the following:
 \begin{align*}
    \left( M^{\perp} \right) ^{\perp} &= \left( \left( \overline M \right) ^{\perp} \right) ^{\perp} & \text{by Claim 1}  \\
    &= \overline{\overline{M}} & \text{by Claim 2} \\
    &= \overline M
    \end{align*}

    
\end{proof}

\section{Question 5}

\horz

Let $M$ be a subspace of a Hilbert space $H$. Show that $(M^{\perp})^{\perp}= \overline{M}.$

\horz

\section{Question 6 \texorpdfstring{$\checkmark$}{\text{✓}}}
\horz
Let $X$ be a finite-dimensional normed linear space and $E\subset X$. Show that $E$ is compact if and only if $E$ is closed and bounded subset of $X.$ 
\horz

\begin{proof}[Solution]
    Let $X$ be $n$ dimensional normed linear space with the norm $\norm{}_{1}$. We can also give an inner product structure $X$ in the following way: Fix a basis $\left\{ v_{i} : i\in \left\{ 1,2,\ldots , n \right\} \right\}$ of $X$ and we define an inner product:
    \begin{equation}
	\ip{\sum_{i=1}^{n} x_{i}v_{i}, \sum_{i=1}^{n} y_{i}v_{i}} = \sum_{i=1}^{n} x_{i}\overline{y_{i}}	
	\label{eqn:q6}
    \end{equation}
    Since every finite dimensional inner product space is isometrically isomorphic to $\C^{n}$ with the $2$-norm, we have that $X$ with the norm induced by the inner product defined in Equation \ref{eqn:q6} is isometrically isomorphic with $\C^{n}$ with the $2$-norm. Let's call this norm induced by the inner product by $\norm{}_{2}$. Let $T: \left( X, \norm{}_{2} \right) \to \left( \C^{n} , \norm{}_{2} \right)$ be an isometric isomorphism.

    We claim that in $\left( X, \norm{}_{2} \right)$, $E$ is compact iff $E$ is closed and bounded. Since in every metric space, we have that compact subsets are closed and bounded, we need to check only one direction. To do so, let $E$ be closed and bounded subset of $X$. Then $T(E)$ is closed and bounded because $T$ is an isometric isomorphism (and hence it is an homeomorphism and preserves length). Since $T(E)$ is a closed and bounded subset of $\C ^{n}$, we have that $T\left( E \right)$ is compact. Since $T$ is an homeomorphism, $E=T^{-1} \left( T\left( E \right) \right)$ is compact. This completes the proof of the claim.

    Now, every norm on finite dimensional vector space is equivalent, so, the topology generated by $1$-norm is the same as the topology generated by the $2$-norm. So, we have that following:
    \begin{align*}
	E \text{ is compact in } \left( X, \norm{}_{1} \right) & \Leftrightarrow E \text{ is compact in } \left( X, \norm{}_{2} \right) & \text{topologies are the same!} \\
    & \Leftrightarrow E \text{ is closed and bounded in } \left( X, \norm{}_{2} \right) & \text{by the above claim} \\
& \stackrel{(\star )}{\Leftrightarrow} E \text{ is closed and bounded in } \left( X, \norm{}_{1} \right) & \text{topologies are the same!} \\
    \end{align*}
    Observe that in $\left( \star \right)$, closed is a topological property but not boundedness. But boundness is due to definition of equivalence of norms. This completes the proof.
\end{proof}

\section{Question 7 \texorpdfstring{$\checkmark$}{\text{✓}}}

\horz

(Direct sum of two Hilbert spaces) : Let $H_1$ and $H_2$ be two Hilbert spaces. Now consider the vector space $H_1\times H_2.$ For two vector $h=(h_1,h_2)$ and $g=(g_1,g_2)$ in $H_1\times H_2,$ define
\begin{align*}
\langle h,g\rangle = \langle h_1,g_1\rangle_{H_1} + \langle h_2,g_2\rangle_{H_2}.
\end{align*}
Show that  $(H_1\times H_2, \langle \cdot,\cdot\rangle )$ is a Hilbert space. This Hilbert space is called as direct sum of $H_1$ and $H_2$ and denoted as $H_1\oplus H_2.$ 

\horz

\begin{proof}[Solution]
    I pass the burden to Question 8.
\end{proof}

\section{Question 8 \texorpdfstring{$\checkmark$}{\text{✓}}}
\horz
Let $X$ be a normed linear space and $F: X \to \mathbb C$ be a non zero linear functional. Suppose $F(x_0) \neq 0$ for some $x_0\in X.$ Show that $X= \ker F \oplus \mbox{span} \{x_0\},$ that is,
\begin{itemize}
\item[(i)] $\ker F \cap \mbox{span} \{x_0\} = \{0\}.$
\item[(ii)] $ X = \ker F + \mbox{span} \{x_0\}.$
\end{itemize}
Show that $F$ is continuous if and only if $\ker F$ is a closed subspace in $X.$ (Hint : Use the continuity of the projection map $\pi:X \to  X/\ker F$ defined by $\pi(x) = [x],\,x\in X.$ )
\horz
\begin{proof}
    Let $X$ be a normed linear space and $F: X \to \C$ be nonzero linear functional. Since $F$ is nonzero, there must be some $x_{0} \in X$ such that $F\left( x_{0} \right) \ne 0$. We now proceed to show that $X=\ker F \oplus \operatorname{span} \left\{ x_{0} \right\}$.

    We first show that $X=\ker F + \operatorname {span} \left\{ x_{0} \right\}$. Let $x\in X$. Then $F\left( x \right) \in \C$. Since $F\left( x_{0} \right) \ne 0$. There must be some $\lambda \in \C$ such that $F\left( x \right) = \lambda F \left( x_{0} \right)$. Thus, we have that $F\left( x-\lambda x_{0} \right) = 0$. Thus, $x-\lambda x_{0}  \in \ker F$. Hence, $x= \lambda x_{0} + y$ for some $y \in \ker F$. This shows that $X=\ker F + \operatorname{span} \left\{ x_{0} \right\}$. 

    Now, we proceed to show that $\ker F \cap \operatorname{span} \left\{ x_{0} \right\} = \left\{ 0 \right\}$. To do so, let $y\in \ker F \cap \operatorname{span } \left\{ x_{0} \right\}$. Then we have thath $y= \lambda x_{0}$ for some $\lambda \in \C$. Hence, we have that $F\left( y \right) = \lambda F\left( x_{0} \right) = 0$. Since $F\left( x_0 \right) \ne 0$, we have that $\lambda = 0$ and thus, $y=0$. This completes the proof of the claim.

    The above two paragraphs show that $X = \ker F \oplus \operatorname{span} \left\{ x_{0} \right\}$.

    Now, we proceed to show that $F$ is continuous iff $\ker F$ is a closed subspace of $X$. Let's begin the proof in the $\left( \Rightarrow \right)$ direction. Suppose that $F$ is continuous. Then we have that $\ker F = F^{-1} \left( \left\{ 0 \right\} \right)$ and hence it must be closed.
    
    To show the reverse direction, namely $\left( \Leftarrow \right)$, we first show that the projection map is continuous. First, we observe that for any $x\in X$, we have that
    \begin{align*}
	\norm{[x]} &= \inf_{y\in \ker F} \norm{x-y} & \text{by definition} \\
	&\le \norm{x} & 0 \in \ker F
    \end{align*}
    Now, this shows that the projection map $\pi : X \to X/\ker F$ is bounded and since it is a linear map, it is continuous.

    Now, consider the map $\tilde{T} : X / \ker F \to \C$ given by
    \begin{align*}
	[x] \stackrel{\tilde{T}}{\mapsto} F\left( x \right)
    \end{align*}
    We showed that $X= \ker F \oplus \operatorname{span} \left\{ x_{0} \right\}$. By the first isomorphism theorem for vector spaces, we have that $X/\ker F \cong \operatorname{span} \left\{ x_{0} \right\}$. This shows that $X/\ker F$ is finite dimensional. Since $\tilde{T}$ is linear and $X/\ker F$ is finite dimensional, we have thath $\tilde{T}$ is continuous.

    Observe that $T = \tilde{T} \circ \pi : X \to \C$ is continuous linear functional by virtue of being composition of two continuous linear maps. This completes the proof.
\end{proof}

\section{Question 9}
\horz

Consider the normed linear space $(\mathbb R ^{2}, \|\cdot\|_1 ),$ where the the distance $d$ is given by 
\begin{align*}
d(x,y) = |x_1-y_1| + |x_2-y_2|,\,\,x=(x_1,x_2)\in \mathbb R^2, \,y=(y_1,y_2)\in \mathbb R^2.
\end{align*}
 Now consider the set $S= \{ ( x_{1} , x_{2}) : x_{1} + x_{2} =1 \}$. Show that the distance of the zero vector from $S,$ that is, $d(0,S)$  is achieved at infinitely many points in $S$.
 
 \horz

\section{Question 10 \texorpdfstring{$\checkmark$}{\text{✓}}}

\horz
Consider $C[0,1],$ the space of all complex valued continuous function on the interval $[0,1],$ equipped with the supremum norm, $\|\cdot\|_{\infty},$ that is, $\|f\|_{\infty}=\sup_{x\in [0,1]} |f(x)|.$ Let $S$ be the subset
    \begin{equation*}
	S=\left\{ f \in C[0,1] \, : \, \int_{0}^{1/2} f\left( x \right) dx - \int_{1/2}^{1} f \left( x \right) dx =1 \right\}
    \end{equation*}
    Show that the set $S$ is closed and convex but the distance is never achieved. That is, there is no $f\in S$ such that $\|f\|_{\infty} = d(0,S)$.
    
\horz
\begin{proof}[Solution]
   We begin by showing that $S$ is convex. Let $f, g \in S$ and $t\in [0,1]$. Then we have that 
    \begin{align*}
	\int_{0}^{1/2} \left( t f\left( x \right) + \left( 1-t \right) g\left( x \right)\right) dx - \int_{1/2}^{1} \left( t f\left( x \right) + \left( 1-t \right) g\left( x \right) dx  \right) &= t + (1-t) \\
	&= 1
    \end{align*}
    Note that the second equality follows by the virtue of $f,g \in S$.

    Now, we proceed to show that the $S$ is closed. Let $\left( f_{n} \right) $ be a sequence of functions in $S$ converging to $f \in C\left[ 0,1 \right]$. We need to prove that $f \in S$. Now convergence in supremum norm is the same as the uniform convergence, so, we have that following:
    \begin{align*}
	\lim_{n\to \infty}\left( \int_{0}^{1/2} f_{n}\left( x \right) dx - \int_{1/2}^{1} f_{n} \left( x \right) dx \right) =1
    \end{align*}
    implies 
    \begin{align*}
\int_{0}^{1/2} f\left( x \right) dx - \int_{1/2}^{1} f \left( x \right) dx =1 
\end{align*}
and thus $f \in S$.
Consider the zero function and the set $S$, we show that that there is no $f \in S$ such that $d (0,S) = d(f,0)=\norm{f}_{\infty}$. We show this in gentle steps as it follows.

Now, we proceed to show that $d\left( 0,S \right) = 1$. To do so, observe that we need to show that $\inf \left\{ \norm{f} _{\infty} : f \in S \right\} = 1$. First of all, if $f\in S$ then we have that 
\begin{align*}
    \int_{0}^{1/2} f - \int_{1/2}^{1} f =1 & \leadsto \abs {\int_{0}^{1/2} f - \int_{1/2}^{1} f } =1  \\
&\leadsto  \abs {\int_{0}^{1/2} f} +\abs{ \int_{1/2}^{1} f } \ge 1  \\
&\leadsto \norm{f}_{\infty} \ge 1
\end{align*}

Hence, we have that $1$ is a lowerbound for the set $S$. Now, let $\varepsilon > 0$ be given. We show that there is some function $f \in S$ such that $1 + \varepsilon > \norm{f}_{\infty}$. This will establish that $d\left( 0,S \right) = 1$. Select a $1/n < \varepsilon$.
Consider the function
\begin{align*}
    f(x) = \color{red}{tooLazyToFigureThisOut!}  
\end{align*}

It can be shown that this function $f$ has sup norm equals $1+\varepsilon$ and is a member of $S$.
Now, we proceed to show  that there is no function $f \in S$ such that $\norm{f}_{\infty} = 1$.

I sketch a idea of how to do this because it is way too long otherwise. Using the hypothesis show that 
\begin{equation*}
    \int_{0}^{1} f = 1/2
\end{equation*}
and 
\begin{equation*}
    \int_{1/2}^{1} f = -1/2
\end{equation*}
Further show that $f\equiv 1$ on $[0,1/2]$ and $f\equiv -1$ on $[1/2,1]$. But mother continuity won't let this happen.
\end{proof}

\section{Question 11}
\horz
Let $X$ be a finite dimensional normed linear space and $M$ be a proper closed subspace of $X$. Show that the unit sphere  $S := \{x: \|x\|=1\}$ on $X$ is compact. Use this to show that there exist a unit vector $x$ such that $\text{dist}(x, M) = 1.$ This need not to be true if $X$ is infinite dimensional. Show that the choice
    \begin{align*}
       & X = \{f \in C[0, 1] : f(0) = 0\}\\
        &M= \{f \in X : \int_{0}^{1}f = 0\}
    \end{align*}
provides a counter example. (This also shows that in F. Riesz's Lemma the constant $t$ can not be taken to be equal to 1 in general.)
\horz

\begin{proof}
    Note that $S$ is compact in view of Question 6.
    Consider the map $f: S \to \C$ given by 
    \begin{equation*}
	x \stackrel{f}{\mapsto} d\left( x, M \right)
    \end{equation*}
    Since we have $\abs{d\left( x,M \right) -d\left( y,M \right)} \le d\left( x,y \right)$, $f$ is continuous. We now show that $\sup f\left( S \right) = 1$.
    Let $x\in S$. Then we have that
    \begin{align*}
	d(x,M) &= \inf_{y \in M} \norm{x-y} \\
	&\le \norm{x}=1\text{.} \\
    \end{align*}
    Thus, $\sup f\left( S \right) \le 1$. Now, let $\varepsilon >0$ be arbitrary. Then by F. Riesz lemma, there exists a vector $x_{0} \in M$ such that 
    \begin{align*}
	1-\varepsilon < \norm{y-x_{0}} \text{for all } y \in M &\leadsto 1-\varepsilon \le \inf_{y\in M} \norm{y-x_{0}}=d(x_{0},M) \\
	& \leadsto 1\le \sup f(S) + \varepsilon
    \end{align*}
    Since $\varepsilon > 0$ is arbitrary, we have that $\sup f\left( S \right) \ge 1$. This shows that $\sup f\left( S \right) =1$.

    Now, continuous functions on compact sets achieve their supremum, therefore, there must be some vector $x \in S$ such that $f(x)=d(x,M) =1$.

    Now, we move on to the next part of the question. Consider the following sets:
    \begin{align*}
	X&= \left\{ f \in C[0,1] : f(0)=0 \right\} \\
	M&= \left\{ f\in X : \int_{0}^{1} f = 0 \right\}
    \end{align*}
    We need to show that there is no vector $f\in X$ whose $\norm{f}_{\infty}=1$ but $d(f,M)=1$.
\end{proof}

\section{Question 12}
\horz
Compute 
\begin{align*}
\min_{a,b,c \in\mathbb R} \int_{-1}^1|x^3-a-bx-cx^2|^2 dx
\end{align*}
and find 
\begin{align*}
\max_{g\in S} \int_{-1}^1 x^3g(x) dx,
\end{align*}
where $S= \bigg\{g\in L^2[-1,1] : \int_{-1}^{1} g(x)dx= \int_{-1}^{1} xg(x)dx = \int_{-1}^{1} x^2g(x)dx =0, \int_{-1}^{1} |g(x)|^2dx=1 \bigg\}.$
\horz

\section{Question 13}
\horz
Compute 
\begin{align*}
\min_{a,b,c\in \mathbb R} \int_{0}^{\infty}|x^3-a-bx-cx^2|^2e^{-x} dx
\end{align*}
\horz

\section{Question 14}
\horz
Fix a positive integer $N$, put $\omega= e ^{2\pi i/N}.$ prove the following orthogonality relations

\begin{align*}
\frac{1}{N} \sum\limits_{n=1}^N=\omega^{nk} = 
\begin{cases}
1, & k=0,\\
0, & 1\leqslant k \leqslant N-1.
\end{cases}
\end{align*}
Using this identity show that
\begin{align*}
\langle x,y\rangle = \frac{1}{N} \sum\limits_{n=1}^N \|x+\omega^n y\|^2 \omega^n
\end{align*} holds true in every inner product space provided $N\geqslant 3.$
\horz

\section{Question 15 \texorpdfstring{$\checkmark$}{\text{✓}}}
Consider the map $\varphi : V \to \C ^{n}$ given by 
\begin{align*}
    \varphi \left( v \right)
    =
    \begin{bmatrix}
	\ip{v, b_{1}} \\
	\vdots \\
	\ip{v, b_{n}}
    \end{bmatrix}
\end{align*}
for all $v\in V$. We show that this map $\varphi$ is injective. Our proof will be then complete by the rank nullity theorem.

So, let $v\in V$ and suppose that $\varphi \left( v \right) = 0$. Then $\ip{v, b_{i}} = 0$ for all $i=1,2, \ldots,  n$. Since $b_{1} , \ldots , b_{n}$ is a basis for $V$, there exists $\alpha _{1} , \ldots , \alpha _{n}$ such that
\begin{align*}
    v= \alpha_{1} b_{1} + \ldots + \alpha _{n} b_{n}
\end{align*}
Hence, we have that 
\begin{align*}
    \ip{v, v} &= \ip{v, \alpha_{1} b_{1} + \ldots + \alpha _{n} b_{n} } \\
    &= \sum_{i=1}^{n} \overline{\alpha_{i}} \ip{v, b_{i}} \\
    &= 0
\end{align*}
Hence $v=0$. This completes the proof!

\section{Question 16}
\horz
Let $V$ be a finite dimensional inner product space with inner product $\langle\cdot ,\cdot \rangle.$ Suppose $\beta_b= (b_1,b_2,\ldots,b_n)$ and  $\beta_e=(e_1,e_2,\ldots,e_n)$ are two ordered basis for $V$ which are related by the following relation:
$e_ j= \sum_{k=1}^n P_{k,j}b_k,$ for $j=1,2,\ldots,n.$ In short (in matrix multiplication notation) they are related by the following :
\begin{align*} \label{Basis change matrix}
(e_1,e_2,\ldots,e_n)= (b_1,b_2,\ldots,b_n)\begin{pmatrix}
P_{1,1} & P_{1,2}& \ldots & P_{1,n}\\
P_{2,1} & P_{2,2}& \ldots & P_{2,n}\\
\vdots & \vdots & \ddots & \vdots\\
P_{n,1} & P_{n,2}& \ldots & P_{n,n}
\end{pmatrix},  \,\,\mbox{that is,}\,\,\beta_e=\beta_b P.
%\,\,P= \begin{pmatrix}
%P_{1,1} & P_{1,2}& \ldots & P_{1,n}\\
%P_{2,1} & P_{2,2}& \ldots & P_{2,n}\\
%\vdots & \vdots & \ddots & \vdots\\
%P_{n,1} & P_{n,2}& \ldots & P_{n,n}
%\end{pmatrix},
\end{align*}
Let $G_e$ and $G_b$  be the Grammian matrix given by 
\begin{align*}
G_e= \begin{pmatrix}
\langle e_1,e_1\rangle & \langle e_2,e_1\rangle & \ldots & \langle e_n,e_1\rangle\\
\langle e_1,e_2\rangle & \langle e_2,e_2\rangle & \ldots & \langle e_n,e_2\rangle\\
\vdots & \vdots & \ddots & \vdots\\
\langle e_1,e_n\rangle & \langle e_2,e_n\rangle & \ldots & \langle e_n,e_n\rangle
\end{pmatrix}, \,G_b= \begin{pmatrix}
\langle b_1,b_1\rangle & \langle b_2,b_1\rangle & \ldots & \langle b_n,b_1\rangle\\
\langle b_1,b_2\rangle & \langle b_2,b_2\rangle & \ldots & \langle b_n,b_2\rangle\\
\vdots & \vdots & \ddots & \vdots\\
\langle b_1,b_n\rangle & \langle b_2,b_n\rangle & \ldots & \langle b_n,b_n\rangle
\end{pmatrix}
\end{align*}
(a) Show that  $G_e = {\bar{P}}^tG_bP,$ where $P$ is the matrix $(\!( P_{i,j})\!).$\\
(b) Show that the matrix $G_b$ is positive definite, that is, 
\begin{itemize}
\item[(i)] ${\bar{G_b}}^t= G_b,$ that is, $G_b$ is self adjoint,
\item[(ii)] $\langle G_b x,x\rangle_2 > 0$ for every non zero $x\in \mathbb C^n.$ Here $\langle\cdot,\cdot\rangle_2 $ denotes the standard Eucledian inner product on $\mathbb C^n.$
\end{itemize}
(c) Show that $\{e_1,e_2,\ldots,e_n\}$ is an orthonormal basis of $V$ if and only if $P {\bar{P}}^t={G_b}^{-1}.$\\
(d) Let $T$ be a linear map from $V$ into itself. Suppose the matrix representation of the linear map $T$ w.r.t the basis $\beta_b$ and $\beta_e$ is given by $[T]_{\beta_b}$
and $[T]_{\beta_e}$ respectively. Show that 
\begin{align*}
[T]_{\beta_e} = [T]_{{\beta_b}P} = P^{-1}[T]_{\beta_b} P.
\end{align*}
\horz

\section{Question 17 \texorpdfstring{$\checkmark$}{\text{✓}}}
\horz
Let $(V,\| \cdot\|)$ be a normed linear space where the norm $\|\cdot\|$ on $V$ satisfies the parallelogram law, that is,
\begin{align*}
\|x+y\|^2 + \|x-y\|^2= 2 \|x\|^2 + 2 \|y\|^2,\,\,\,x,y\in V.
\end{align*}
Show that the norm $\|\cdot\|$ is induced by an inner product on $V,$ that is, $\|x\|^2= \langle x,x\rangle$ for some inner product $\langle \cdot,\cdot \rangle$ on $V.$
\horz
\begin{proof}[Solution (as it was done in class).]
Assume $\left( V, \norm{} \right)$ is a real normed linear space which satisfies the parallelogram identity, that is, for all $a,b \in V$, 
\begin{equation*}
    \norm{a+b}^{2} + \norm{a-b}^{2} = 2\norm{a}^{2} +2 \norm{b}^{2}
\end{equation*}

We intend to define the inner product on $V$ by
\begin{align*}
    \ip{v,w} = \frac{\norm{x+y}^{2} - \norm{x-y}^{2}}{4}
\end{align*}

We show that $\ip{\cdot, \cdot}$ is an inner product.

The symmetric property is evident.

We proceed to show linearity in the first variable, that is, we need to show that 
\begin{align*}
    \norm{x_{1} + x_{2} + y}^{2} - \norm{x_{1}+x_{2}-y}^{2} = \norm{x_{1} + y}^2-\norm{x_{1}-y}^{2} + \norm{x_{2}+y}^{2} - \norm{x_{2}-y}^{2}
\end{align*}

Setting $a=x_{1} $ and $b=x_{2}+y$ in the parallelogram identity, we get
\begin{align*}
    \norm{x_{1} + y + x_{2}}^{2} + \norm{x_{1} - y - x_{2}}^{2} = 2 \norm{x_{1} } ^{2} + 2 \norm{x_{2} + y}^{2}
\end{align*}

Doing the same for $a=x_{2} -y $ and $b=x_{2}$, we have
\begin{align*}
    \norm{x_{1} - y + x_{2}}^{2} + \norm{x_{1} - y - x_{2}}^{2} = 2 \norm{x_{1} -y} ^{2} + 2 \norm{x_{2}}^{2}
\end{align*}

Subtracting the above two equations, we get
\begin{align*}
    \norm{x_{1} + x_{2} + y}^{2} - \norm{x_{1} - y + x_{2}} ^{2} = 2\norm{x_{1}}^{2} + 2 \norm{x_{2} + y} ^{2} - 2 \norm{x_{1} - y}^{2} - 2 \norm{x_{2}}^{2}
\end{align*}
 
Switching the roles of $x_{2}$ and $x_{1}$, we get
\begin{align*}
    \norm{x_{2} + x_{1} + y}^{2} - \norm{x_{2} - y + x_{1}} ^{2} = 2\norm{x_{2}}^{2} + 2 \norm{x_{1} + y} ^{2} - 2 \norm{x_{2} - y}^{2} - 2 \norm{x_{1}}^{2}
\end{align*}

Adding the above two equations, we get
\begin{align*}
    2\norm{x_{1} + x_{2} + y}^{2} - 2\norm{x_{1} +x_{2} -y }^{2} = 2 \norm{x_{2} + y}^{2} - 2 \norm{x_{1} -y}^{2} + 2 \norm{x_{1} + y} ^{2} -2 \norm{x_{2}-y}^{2}
\end{align*}

Rearranging the above equation, we observe that we have established what we wanted to prove!

Linearity is the other variable follows by symmetry and the linearity in the first variable!

Now, finally we proceed to show that for any $\lambda \in \R$, we have that 
\begin{align*}
    \ip{\lambda x, y} = \lambda \ip {x,y}
\end{align*}

Note that by linearity in the first variable, we have that for $n\in \Z$,
\begin{equation*}
    \ip{nx,y} = n \ip{x,y}
\end{equation*}
In a similar fashion, it can be shown that for $r\in \Q$,

\begin{equation*}
    \ip{rx,y} = r \ip{x,y}
\end{equation*}

Let us assume \textcolor{red}{Cauchy-Schwarz!} at the moment. Let $r\in \R$. Let $r_{n}$ be a sequence of rationals converging to $r\in \R$.

Observe that fixing $y\in V$, it is easily seen that 
\begin{equation*}
    x \mapsto \frac{\norm{x+y}^{2} - \norm{x-y}^{2}}{4}
\end{equation*}
is continuous by virtue of translation, norm and square of a function being continuous!
Then the result follows!

Irregardless, we prove Cauchy Schwarz! It can be seen by minimizing $r$ is the function $r \mapsto \norm{rx + y}^{2}$. One needs to see that for $r\in \Q$
\begin{align*}
    \norm{rx+y}^{2} = \ip{rx+y , rx +y}
    = r^{2} \norm{x}^{2} + 2r\ip{x,y} + \norm{y}^{2} \ge 0
\end{align*}
Hence the above holds for any $r\in \R$ by taking limits. Minimizing the function, we get the Cauchy Schwarz inequality.


We now proceed to the complex case!

Let $V, \norm{}$ be a complex normed linear space. By the polarization identity, we have that for $x, y \in V$,
\begin{align*}
    \ip{x,y} &= \frac{1}{4} \sum_{k=1}^{4} \norm{x+i^{k}y}^{2} i^{k} \\
    &= \frac{\norm{x+y}^{2} - \norm{x-y}^{2}}{4} - \frac{\norm{x+iy}^{2} - \norm{x-iy}^{2}}{4} \\
    &= q(x,y)+i q(x,iy)
\end{align*}
where $q(x,y)=  \frac{\norm{x+y}^{2} - \norm{x-y}^{2}}{4}$. We have already shown that $q$ is an inner product over $\R$. 

Now one can use the properties of inner product for $q$ to show that $\ip{\cdot , \cdot}$ is an inner product over $\C$.

\end{proof}

\section{Question 18}

\horz

Let $V$ be a finite dimensional inner product space with inner product $\langle\cdot ,\cdot \rangle.$ Suppose $W$ is a subspace of $V$ and $\beta_k=(v_1,v_2,\ldots,v_k)$ is a ordered basis for $W.$ Let $y=v_{k+1}$ is a vector outside $W.$
\begin{center}
\textbf{A distance formula to keep in mind} (Proof not needed)
\end{center}
Then the distance of $v_{k+1}$ from $W$ is given by $d(v_{k+1},W)= \frac{\sqrt{\det G_{\beta_{k+1}}}}{\sqrt{\det G_{\beta_k}}},$ where $G_{\beta_k}$ and $G_{\beta_{k+1}}$ are the Grammian matrix associated to the vectors $\beta_k=(v_1,v_2,\ldots,v_k)$ and $\beta_{k+1}=(v_1,v_2,\ldots,v_k,v_{k+1}).$\\

{\textbf{Sketch of the proof}} : Volume of the k dimensional parallelepiped formed by the vectors in $\beta_k$ $\times \,\, d(v_{k+1},W) = $ Volume of the (k+1) dimensional parallelepiped formed by the vectors in $\beta_{k+1}.$


\textbf{Problem}
Let $\mathcal P_3$ be the vector space of all polynomials over $\mathbb R$ of degree less than or equal to $3,$ with the inner product 
\begin{align*}
\langle f,g\rangle = \int_{0}^1f(t)g(t)dt,\,\,f,g\in \mathcal P_3.
\end{align*} 
Let $\mathcal P_2$ be the subspace of  $\mathcal P_3$ given by the set of all polynomials over $\mathbb R$ of degree less than or equal to $2.$ Find the distance of $x^3$ from $\mathcal P_2.$

\horz

\end{document}
