\documentclass[12pt]{article}
\usepackage[margin=1in]{geometry}
\usepackage{amsfonts, amsmath}
\usepackage[T1]{fontenc}
\usepackage{mathrsfs, enumitem}
\usepackage{hyperref}
\usepackage[utf8]{inputenc}
\usepackage{amssymb}
\usepackage{amsfonts}
\usepackage{amsmath}
\usepackage{amsthm}
\usepackage{color}
\usepackage{hyperref}
\usepackage{csquotes}
%\usepackage{fourier}
\usepackage{tikz-cd}
\usepackage{lipsum}
\usepackage{cancel}

\newtheorem{theorem}{Theorem}[subsection]
\newtheorem{lemma}[theorem]{Lemma}
\newtheorem{claim}[theorem]{Claim}
\newtheorem{proposition}[theorem]{Proposition}
\newtheorem{corollary}[theorem]{Corollary}
\newtheorem{fact}[theorem]{Fact}
\newtheorem{notation}[theorem]{Notation}
\newtheorem{observation}[theorem]{Observation}
\newtheorem{conjecture}[theorem]{Conjecture}
\newtheorem{exercise}[theorem]{Exercise}

\theoremstyle{definition}
\newtheorem{definition}[theorem]{Definition}
\newtheorem{example}[theorem]{Example}
\numberwithin{equation}{subsection}

\theoremstyle{remark}
\newtheorem{remark}[theorem]{Remark}
\theoremstyle{plain}
\newcommand{\ignore}[1]{}

% section symbol
%\renewcommand{\thesection}{\S\arabic{section}}

% \renewcommand{\Pr}{{\bf Pr}}
% \newcommand{\Prx}{\mathop{\bf Pr\/}}
% \newcommand{\E}{{\bf E}}
% \newcommand{\Ex}{\mathop{\bf E\/}}
% \newcommand{\Var}{{\bf Var}}
% \newcommand{\Varx}{\mathop{\bf Var\/}}
% \newcommand{\Cov}{{\bf Cov}}
% \newcommand{\Covx}{\mathop{\bf Cov\/}}

% shortcuts for symbol names that are too long to type
\newcommand{\eps}{\epsilon}
\newcommand{\lam}{\lambda}
\renewcommand{\l}{\ell}
\newcommand{\la}{\langle}
\newcommand{\ra}{\rangle}
\newcommand{\wh}{\widehat}
\newcommand{\wt}{\widetilde}

% % "blackboard-fonted" letters for the reals, naturals etc.
\newcommand{\R}{\mathbb R}
\newcommand{\N}{\mathbb N}
\newcommand{\Z}{\mathbb Z}
\newcommand{\F}{\mathbb F}
\newcommand{\Q}{\mathbb Q}
\newcommand{\C}{\mathbb C}

% % operators that should be typeset in Roman font
% \newcommand{\poly}{\mathrm{poly}}
% \newcommand{\polylog}{\mathrm{polylog}}
% \newcommand{\sgn}{\mathrm{sgn}}
% \newcommand{\avg}{\mathop{\mathrm{avg}}}
% \newcommand{\val}{{\mathrm{val}}}

% % complexity classes
% \renewcommand{\P}{\mathrm{P}}
% \newcommand{\NP}{\mathrm{NP}}
% \newcommand{\BPP}{\mathrm{BPP}}
% \newcommand{\DTIME}{\mathrm{DTIME}}
% \newcommand{\ZPTIME}{\mathrm{ZPTIME}}
% \newcommand{\BPTIME}{\mathrm{BPTIME}}
% \newcommand{\NTIME}{\mathrm{NTIME}}

% values associated to optimization algorithm instances
\newcommand{\Opt}{{\mathsf{Opt}}}
\newcommand{\Alg}{{\mathsf{Alg}}}
\newcommand{\Lp}{{\mathsf{Lp}}}
\newcommand{\Sdp}{{\mathsf{Sdp}}}
\newcommand{\Exp}{{\mathsf{Exp}}}

% if you think the sum and product signs are too big in your math mode; x convention
% as in the probability operators
\newcommand{\littlesum}{{\textstyle \sum}}
\newcommand{\littlesumx}{\mathop{{\textstyle \sum}}}
\newcommand{\littleprod}{{\textstyle \prod}}
\newcommand{\littleprodx}{\mathop{{\textstyle \prod}}}

% horizontal line across the page
\newcommand{\horz}{
\vspace{-.4in}
\begin{center}
\begin{tabular}{p{\textwidth}}\\
\hline
\end{tabular}
\end{center}
}

% calligraphic letters
\newcommand{\calA}{{\cal A}}
\newcommand{\calB}{{\cal B}}
\newcommand{\calC}{{\cal C}}
\newcommand{\calD}{{\cal D}}
\newcommand{\calE}{{\cal E}}
\newcommand{\calF}{{\cal F}}
\newcommand{\calG}{{\cal G}}
\newcommand{\calH}{{\cal H}}
\newcommand{\calI}{{\cal I}}
\newcommand{\calJ}{{\cal J}}
\newcommand{\calK}{{\cal K}}
\newcommand{\calL}{{\cal L}}
\newcommand{\calM}{{\cal M}}
\newcommand{\calN}{{\cal N}}
\newcommand{\calO}{{\cal O}}
\newcommand{\calP}{{\cal P}}
\newcommand{\calQ}{{\cal Q}}
\newcommand{\calR}{{\cal R}}
\newcommand{\calS}{{\cal S}}
\newcommand{\calT}{{\cal T}}
\newcommand{\calU}{{\cal U}}
\newcommand{\calV}{{\cal V}}
\newcommand{\calW}{{\cal W}}
\newcommand{\calX}{{\cal X}}
\newcommand{\calY}{{\cal Y}}
\newcommand{\calZ}{{\cal Z}}

% bold letters (useful for random variables)
%----------------------------------------------
% \renewcommand{\a}{{\boldsymbol a}}
% \renewcommand{\b}{{\boldsymbol b}}
% \renewcommand{\c}{{\boldsymbol c}}
% \renewcommand{\d}{{\boldsymbol d}}
% \newcommand{\e}{{\boldsymbol e}}
% \newcommand{\f}{{\boldsymbol f}}
% \newcommand{\g}{{\boldsymbol g}}
% \newcommand{\h}{{\boldsymbol h}}
% \renewcommand{\i}{{\boldsymbol i}}
% \renewcommand{\j}{{\boldsymbol j}}
% \renewcommand{\k}{{\boldsymbol k}}
% \newcommand{\m}{{\boldsymbol m}}
% \newcommand{\n}{{\boldsymbol n}}
% \renewcommand{\o}{{\boldsymbol o}}
% \newcommand{\p}{{\boldsymbol p}}
% \newcommand{\q}{{\boldsymbol q}}
% \renewcommand{\r}{{\boldsymbol r}}
% \newcommand{\s}{{\boldsymbol s}}
% \renewcommand{\t}{{\boldsymbol t}}
% \renewcommand{\u}{{\boldsymbol u}}
% \renewcommand{\v}{{\boldsymbol v}}
% \newcommand{\w}{{\boldsymbol w}}
% \newcommand{\x}{{\boldsymbol x}}
% \newcommand{\y}{{\boldsymbol y}}
% \newcommand{\z}{{\boldsymbol z}}
% \newcommand{\A}{{\boldsymbol A}}
% \newcommand{\B}{{\boldsymbol B}}
% \newcommand{\C}{{\boldsymbol C}}
% \newcommand{\D}{{\boldsymbol D}}
% \newcommand{\E}{{\boldsymbol E}}
% \newcommand{\F}{{\boldsymbol F}}
% \newcommand{\G}{{\boldsymbol G}}
% \renewcommand{\H}{{\boldsymbol H}}
% \newcommand{\I}{{\boldsymbol I}}
% \newcommand{\J}{{\boldsymbol J}}
% \newcommand{\K}{{\boldsymbol K}}
% \renewcommand{\L}{{\boldsymbol L}}
% \newcommand{\M}{{\boldsymbol M}}
% \renewcommand{\O}{{\boldsymbol O}}
% \renewcommand{\P}{{\mathbb{P}}}
% \newcommand{\Q}{{\boldsymbol Q}}
% \newcommand{\R}{{\boldsymbol R}}
% \renewcommand{\S}{{\boldsymbol S}}
% \newcommand{\T}{{\boldsymbol T}}
% \newcommand{\U}{{\boldsymbol U}}
% \newcommand{\V}{{\boldsymbol V}}
% \newcommand{\W}{{\boldsymbol W}}
% \newcommand{\X}{{\boldsymbol X}}
% \newcommand{\Y}{{\boldsymbol Y}}
% \newcommand{\Z}{{\boldsymbol Z}}

% script letters
\newcommand{\scrA}{{\mathscr A}}
\newcommand{\scrB}{{\mathscr B}}
\newcommand{\scrC}{{\mathscr C}}
\newcommand{\scrD}{{\mathscr D}}
\newcommand{\scrE}{{\mathscr E}}
\newcommand{\scrF}{{\mathscr F}}
\newcommand{\scrG}{{\mathscr G}}
\newcommand{\scrH}{{\mathscr H}}
\newcommand{\scrI}{{\mathscr I}}
\newcommand{\scrJ}{{\mathscr J}}
\newcommand{\scrK}{{\mathscr K}}
\newcommand{\scrL}{{\mathscr L}}
\newcommand{\scrM}{{\mathscr M}}
\newcommand{\scrN}{{\mathscr N}}
\newcommand{\scrO}{{\mathscr O}}
\newcommand{\scrP}{{\mathscr P}}
\newcommand{\scrQ}{{\mathscr Q}}
\newcommand{\scrR}{{\mathscr R}}
\newcommand{\scrS}{{\mathscr S}}
\newcommand{\scrT}{{\mathscr T}}
\newcommand{\scrU}{{\mathscr U}}
\newcommand{\scrV}{{\mathscr V}}
\newcommand{\scrW}{{\mathscr W}}
\newcommand{\scrX}{{\mathscr X}}
\newcommand{\scrY}{{\mathscr Y}}
\newcommand{\scrZ}{{\mathscr Z}}

\newcommand{\im}{{\text{im }}}
\newcommand{\ip}[1]{\left\langle #1 \right\rangle}
\newcommand{\norm}[1]{\left\lVert #1 \right\rVert}
\newcommand{\abs}[1]{\left\lvert #1 \right\rvert}

\newcommand\blfootnote[1]{%
  \begingroup
  \renewcommand\thefootnote{}\footnote{#1}%
  \addtocounter{footnote}{-1}%
  \endgroup
}

\title{Solutions to Functional Analysis Assignment 1}
\author{\textsc{Ashish Kujur}}

\date{}

\begin{document}

\maketitle
\section*{Note}
A checkmark $\checkmark$ indicates the question has been done.
\tableofcontents


\section{Question 1 \texorpdfstring{$\checkmark$}{\text{✓}}}
\horz
Let $a,b\in\mathbb R$ and $\mathscr C[a,b]$ denotes the space of real valued continuous function on $[a,b].$ Let $\|f\|_p$ and $\|f\|_{\infty}$ denotes the norm on $\mathscr C[a,b]$ given by 
$\|f\|_p= (\int |f(x)|^p dx )^{\frac{1}{p}},$ for $p\geqslant 1$ and $\|f\|_{\infty}= \sup \{|f(x)| : x\in [a,b]\}.$ 
 Establish the following inequalities :

 \begin{enumerate}[label=(\roman*)]
\item $|\displaystyle \int f(x) g(x) dx| \leqslant \|f\|_p\|g\|_q, \,\,\,f,g\in \mathscr C[a,b], p > 1$ and $1/p +1/q =1.$ 
\item $|\displaystyle \int f(x) g(x) dx| \leqslant \|f\|_1\|g\|_{\infty}, \,\,\,f,g\in \mathscr C[a,b].$
\item $\|f+g\|_p \leqslant \|f\|_p + \|g\|_p,\,\, f,g\in \mathscr C[a,b],$ for every  $p\geqslant 1$ and $p=\infty.$
\end{enumerate}
\horz
\begin{proof}
    Check any measure theory text. \checkmark
\end{proof}

\section{Question 2 \texorpdfstring{$\checkmark$}{\text{✓}}}

\horz

Let $(V,\langle \cdot, \cdot \rangle)$ be an inner product space. Let $\overline{B(0,1)}$ denotes the closed unit ball in $V,$ that is, $$\overline{B(0,1)} = \{x\in V : \|x\|\leqslant 1\}.$$
Show that $\overline{B(0,1)}$ is strictly convex, that is, for any two distinct vector $x,y\in H,$ if $\|x\|=1, \|y\|=1,$ then $\|tx+(1-t)y\| < 1$ for each $t\in (0,1).$
 
\horz

\begin{proof}[Solution]
    Let $x,y \in V$ with $\norm{x}=\norm{y}=1$ and $x\ne y$. A simple computation shows that: 
    \begin{align*}
	\norm{tx+(1-t)y}^{2} -1 &= t^{2} \norm{x}^{2} + 2t(1-t)\Re(\ip{x,y}) + (1-t)^{2} \norm{y}^{2}  -1 \\
	&= t^{2} +2t\left( t-1 \right) \Re\ip{x,y} + 1+t^{2}-2t -1 \\
	&= 2t^{2} -2t + 2t\left( t-1 \right) \Re\left( \ip{x,y} \right) \\
	    &= 2t(1-t) \left( \Re (\ip{x,y}) -1 \right)
	\end{align*}
	Since $t\ne 0$ and $t\ne 1$, we will be done if we show that $\Re (\ip{x,y}) < 1$. If it happened that $\Re \left( \ip{x,y} \right) \ge 1$ then we would have
	\begin{equation*}
	    1 \le \Re  \left( \ip{x,y} \right) \le \abs{\ip{x,y}} \le \norm{x} \norm{y} \le 1
	\end{equation*}
	This shows that $\abs{\ip{x,y}} = \norm{x}\norm{y}$. Equality in CS inequality means that $x$ and $y$ are linearly dependent. Thus, we may assume that $x=\lambda y$ for some $\lambda \in \C$. But then
	\begin{align*}
	    x= \lambda y \Rightarrow \norm{x} = \abs{\lambda}\norm{y} \Rightarrow \abs{\lambda}=1\text{.}
	\end{align*}

	Hence, we may write $x=e^{i\theta} y$ for some $\theta \in [0,2\pi)$. Thus, we have
	\begin{align*}
	    \Re{\ip{x,y}} &= \Re {\ip{x, e^{i\theta} x}} \\
	    &= \Re{(e^{-i\theta})} \\
	    &= \cos \theta 
	\end{align*}
	By assumption that $\Re \left( \ip{x,y} \right) \ge 1$, we have that $\cos \theta \ge 1$. This implies $\cos \theta = 1$. Thus, $\theta = 0$ by our assumption that $\theta \in [0, 2\pi)$. This implies $x=y$ which contradicts our assumption $x\ne y$.
\end{proof}

\section{Question 3 \texorpdfstring{$\checkmark$}{\text{✓}}}
\horz
Let $(X,\|\cdot\|)$ be a normed linear space (in short NLS) and $(X,d)$ be the associated metric, that is, $d(a,b)=\|a-b\|,\,\,a,b\in X.$ Show that a ball $B_d(a,r)$ is always a convex subset of $X.$ 
\horz
\begin{proof}[Solution]
    It suffices to show that $B(0,1)$ is convex because every other ball is just this (modulo translations and dilations).
    We proceed to show that $B(0,1)$ is convex. Let $x,y \in B(0,1)$. Then we have that for $t\in[0,1]$,
    \begin{align*}
	\norm{tx+(1-t)y} &= t \norm{x} + (1-t)\norm{y} \\
	&< t+(1-t) \\
	&= 1
    \end{align*}
    And we're done.
\end{proof}


\section{Question 4 \texorpdfstring{$\checkmark$}{\text{✓}}}
\horz
Let $(X,\|\cdot\|)$ be a normed linear space and $Y$ be a proper subspace of $X$. Show that $Y^{\circ},$ the set of interior point of $Y$, is empty.
\horz

\begin{proof}[Solution]
    Let $Y$ be proper subspace of a normed linear space $X$. Suppose that the interior of $Y$ was nonempty. Then there exists $y\in Y$ and $r>0$ such that $B(y,r) \subset Y$. Since $Y$ is a subspace, we have that $B(0,r) \subset Y$. We now show that $X \subset Y$ which would contradict that $Y$ is a proper subspace of $X$. 

    Let $x\in X \setminus \left\{ 0 \right\}$. Then note that $\frac{r}{2} \frac{x}{\norm{x}} \in B(0,r)$ and  hence $\frac{r}{2} \frac{x}{\norm{x}} \in Y$. But then we have that $x \in Y$ because any $Y$ is a subspace and hence 
    \begin{align*}
	x= \frac{2\norm{x}}{r} \left( \frac{r}{2} \frac{x}{\norm{x}} \right) \in Y
    \end{align*}
    And we're done.
\end{proof}

\section{Question 5}

\horz

Let $M$ be a subspace of a Hilbert space $H$. Show that $(M^{\perp})^{\perp}= \overline{M}.$

\horz

\section{Question 6}
\horz
Let $(V,\langle \cdot, \cdot \rangle)$ be an inner product space. Let $\overline{B(0,1)}$ denotes the closed unit ball in $V,$ that is, $$\overline{B(0,1)} = \{x\in V : \|x\|\leqslant 1\}.$$ Show that $\overline{B(0,1)}$ is compact if and only if dimension of $V$ is finite. \\
(Hint : if $\mathcal B = \{u_{\alpha} : \alpha\in I\}$ is a collection of orthonormal vectors in $V,$ then $\|u_{\alpha}-u_{\beta}\|=\sqrt{2}$ for  every $\alpha,\beta\in I$ and $\alpha \neq \beta.$ )
\horz

\begin{proof}[Solution]
    $\left( \Longleftarrow \right)$ Suppose that $\dim V$ is finite. In Lecture 5, we showed that $V$ is isometrically isomorphic to $\C ^{n}$ with Euclidean norm, that is, there exists a linear map $T: V \to \C^{n}$ which is an isometry. 
    Since every isometry is an homeomorphism\footnote{\href{https://planetmath.org/isometry}{proof here!}}, we have that $T^{-1} \left( \overline{B_{\C^{n}}\left( 0,1 \right)} \right) = \overline{B_{V} \left( 0,1 \right)}$. Since a continuous image of a compact set is compact, we have that $\overline{B_{V}(0,1)}$ is compact!

    $\left( \Longrightarrow \right)$ Follow the hint?
\end{proof}

\section{Question 7}

\horz

(Direct sum of two Hilbert spaces) : Let $H_1$ and $H_2$ be two Hilbert spaces. Now consider the vector space $H_1\times H_2.$ For two vector $h=(h_1,h_2)$ and $g=(g_1,g_2)$ in $H_1\times H_2,$ define
\begin{align*}
\langle h,g\rangle = \langle h_1,g_1\rangle_{H_1} + \langle h_2,g_2\rangle_{H_2}.
\end{align*}
Show that  $(H_1\times H_2, \langle \cdot,\cdot\rangle )$ is a Hilbert space. This Hilbert space is called as direct sum of $H_1$ and $H_2$ and denoted as $H_1\oplus H_2.$ 

\horz

\begin{proof}[Solution]
    Routine Check \checkmark.
\end{proof}

\section{Question 8}

\horz
(Direct sum of family of Hilbert spaces) : Let $\{H_k\}_{k\in \mathbb N}$ be a sequence of Hilbert space. Consider the vector space $H$ defined by 
\begin{align*}
H = \Big\{ (h_k)_{k\in\mathbb N} : h_k\in H_k \mbox{\,for all\,} n\in\mathbb N, \,\mbox{and}\, \sum\limits_{k=1}^{\infty}\|h_k\|^2 < \infty\Big\}.
\end{align*}
For $h= (h_k)_{k\in\mathbb N} \in H$ and $g = (g_k)_{k\in\mathbb N}\in H,$ define
\begin{align*}
\langle h,g\rangle = \sum\limits_{k=1}^{\infty} \langle h_k,g_k\rangle_{H_k}.
\end{align*} 
Show that $H,\langle \cdot, \cdot \rangle$ is a Hilbert space. This $H$ is called the direct sum of the family of Hilbert spaces  $\{H_k\}_{k\in \mathbb N}$ and denoted as $\bigoplus\limits _{k\in\mathbb N} H_k.$
\horz

\begin{proof}[Solution]
    First, we need to show that $H$ is a vector space. To do so, it suffices to show that it is closed under sum and scalar multiplication (as this is a subspace of the vector space of functions from $\N$ to $\cup_{i=1}^{n} H_{i}$). 

Let $\left( h_{k} \right)_{k \in \N}$, $\left( g_{k} \right)_{k\in \N}$ be two elements of $H$. Then for any $k\in \N$, we have that
\begin{equation*}
\norm{h_{k} + g_{k}} ^{2} \stackrel{\textcolor{red}{?}}\le \norm{h_{k}}^{2} \end{equation*}
If $\alpha \in \C$ then we have that 
\begin{align*}
    \sum_{k=1}^{\infty} \norm{\alpha h_{k}}^{2} \le \sum_{k=1}^{+\infty} \alpha \norm{h_{k}}^{2} < \infty
\end{align*}
This shows that $H$ is a vector space. Now, we proceed to show that the prescribed inner product is indeed an inner product.

\end{proof}

\section{Question 9}
\horz
Let $X$ be a normed linear space and $S=\{x\in X: \|x\|=1\}$ be the unit sphere in $X$. Show that $X$ is complete if and only if $S$ is complete. 
\horz
\begin{proof}[Solution]
    $\left( \Rightarrow \right)$ Suppose that $\left( V,d \right)$ is complete. Then $S$ is complete because $S$ is closed in $V$. 

    $\left( \Leftarrow \right)$ Suppose that $\left( S,d \right)$ is complete. Let $\left\{ x_{n} \right\}$ be a Cauchy sequence in $V$.

    We consider two different cases, namely.
    \begin{enumerate}[label=(\roman*)]
	\item $x_{n} = 0$ for infinitely many $n\in \N$ and
	\item $x_{n} = 0$ for at most finitely many $n\in \N$.
    \end{enumerate}

    We conisder the first case. Suppose that $x_{n} = 0$ for infinitely many $n\in \N$. We, therefore, can select a subsequence $\left\{ x_{n_k} \right\}$ of $\left\{ x_{n} \right\}$ such that $x_{n_{k}} = 0$ for every $k\in \N$. We now show that $\left\{ x_{n} \right\}$ converges to $0$.

    Let $\varepsilon > 0$ be given. Then there is some $N \in \N$ such that
    \begin{align*}
	\norm{x_{n} - x_{m}} < \varepsilon \text{ whenever } n,m \ge N\text{.}
    \end{align*}
    Select $k\in \N$ such that $n_{k} \ge N$. Now, we have that
    \begin{align*}
	\norm{x_{n}} &= \norm{x_{n} - x_{n_k}} \\
	&< \varepsilon
    \end{align*}
    This shows that $\left\{ x_{n} \right\}$ converges to $0$.

    We now consider the second case. Suppose that $x_{n} = 0$ for at most finitely many $n\in \N$. Therefore, there is some $N\in \N$ such that $x_{n} \ne 0$ for $n\ge N$. Convergence of sequence depends only on its tail, so, we assume without loss of generality, that $x_{n} \ne 0$ for every $n\in \N$.
    
    Now, consider the sequence $\left\{ y_{n} \right\} $ in $S$ given by
    \begin{equation*}
	y_{n} = \frac{x_{n}}{\norm{x_{n}}}
    \end{equation*}
    for each $n\in \N$.

    We now claim that $\norm{x_{n}}$ converges to some $\lambda \ge 0$. Since $\left\{ x_{n} \right\}$ is Cauchy, so, is $\left\{ \norm{x_{n}} \right\}$. But Cauchy in $\R$ implies convergence, and, hence $\left\{ \norm{x_{n}} \right\}$ converges to some $\lambda \ge 0$. If $\norm{x_{n}}$ converges to $0$ then we have that $\{ x_{n} \}$ converges to $0$. And we would be done. So, suppose that $\lambda > 0$. So, by the definition of convergence, there must be some $K \in \N$ such that $\norm{x_{n}} > \frac{\lambda}{2}$ for every $n \ge K$.

Consider the following for $n, m \in \N$,
\begin{align*}
    \norm{y_{m} - y_{n}} &=  \norm{\frac{x_{m}}{\norm{x_{m}}} - \frac{x_{n}}{\norm{x_{m}}}} \\
    &= \norm{\frac{\norm{x_{n}}x_{m} - \norm{x_m}x_{n}}{\norm{x_{m}}\norm{x_{n}}}} \\
    &= \norm{\frac{\norm{x_{n}}x_{m} - x_{m} \norm{x_{m}} + \norm{x_{m}} x_{m} - \norm{x_{m}} }{x_{n}\norm{x_{m}\norm{x_n}}} } \\
    &= \norm{\frac{x_{m} \left( \norm{x_{n}} - \norm{x_m} \right) + \norm{x_m} \left( x_{m} -x_{n} \right)}{\norm{x_{m}}\norm{x_{n}}}} \\
    &\le \frac{1}{\norm{x_{n}}\norm{x_{m}}} \left( \norm{x_{m}} \abs{\norm{x_{n}} - \norm{x_{m}}} + \norm{x_{m}} \norm{x_{m} - x_{n}} \right) \\
    & \le \frac{2}{\norm{x_{n}}} \norm{x_{n} - x_{m}}
\end{align*}

    Now, let $\varepsilon > 0$ be given. Since $\left\{ x_{n} \right\}$ is Cauchy, we have that there is some $M \in \N$ such that 
    \begin{align*}
	\norm{x_{n} - x_{m}} < \frac{\lambda}{4} \varepsilon \text{ whenever } n,m \ge M\text{.}
    \end{align*}

    For $m,n \ge \max \left\{ M,K \right\}$ we have that
    \begin{align*}
	\norm{y_{m} - y_{n}} &\le \frac{2}{\norm{x_{n}}} \norm{x_{n} - x_{m}} \\
	& < \frac{2}{\lambda/2} \frac{\lambda}{4} \varepsilon = \varepsilon
    \end{align*}
    This shows that $\left\{ y_{n} \right\}$ is Cauchy and hence convergent. Since $x_{n} = y_{n} \norm{x_{n}}$ for all $n\in \N$ and product of two convergent sequences is convergent. We have that $\left\{ x_{n} \right\}$ is convergenet.
\end{proof}

\section{Question 10 \texorpdfstring{$\checkmark$}{\text{✓}}}
\horz
Let $X$ be a normed linear space and $F: X \to \mathbb C$ be a continuous, non zero linear functional. 
\begin{itemize}
\item[(a)] Show that $ N(F)=\sup\Big\{ \frac{|F(x)|}{\|x\|} : x \in X, x \neq 0\Big\} < \infty.$
\item[(b)] Suppose $M = \ker F$ and $x_0 \notin M.$ Show that $d(x_0, M) = \frac{|F(x_0)|}{N(F)}.$
\end{itemize}
\horz

\begin{proof}[Solution]
    Since $F$ is continuous, we have that there must be some $M > 0$ such that 
    \begin{align*}
	\abs{F(x)} \le M \norm{x} \text{ for all } x \in X\text{.} 
    \end{align*}
    Thus, we have that 
    \begin{align*}
	\frac{\abs{F(x)}}{\norm{x}} \le M \text{ for all } x \in X\text{.} 
    \end{align*}
    This shows that $N\left( F \right) < \infty$. \footnote{One can easily check that $N(F)$ is $\norm{F}_{X^*}$ in disguise!}

    Now, let $M=\ker F$. Observe that by definition of quotient norm, we have that 
    \begin{align*}
	\norm{[x]}_{X/M} = \inf_{m\in M} \norm{x-m}_{M} = d\left( x, M \right)
    \end{align*}
    for every $x\in X$.
    Therefore, we need to show that $\abs{F\left( x_{0} \right)} = N(F) \norm{[x_{0}]}_{X/M}$. Let $x\in M$ be arbitrary. Then by Question 8, we have that $x=y + \lambda x_{0}$ for some $y \in \ker F$ and some $\lambda \in \C$. In case, $\lambda = 0$, we have that 
    \begin{align*}
\frac{\abs{F\left( x \right)}}{\norm{x}} &= \frac{F\left( y\right)}{\norm{y}} \\
&= 0 \le \frac{\abs{F\left( x_{0} \right)}}{\norm{[x_{0}]}_{X/M}}
    \end{align*}
    Else if $\lambda \ne 0$, we have that
    \begin{align*}
	\frac{\abs{F\left( x \right)}}{\norm{x}} &=  \frac{\abs{F\left( y+ \lambda x_{0} \right)}}{\norm{\lambda x_{0} + y}} \\
	&= \frac{\abs{F\left( x_{0} \right)}}{\norm{x_{0} + \frac{1}{\lambda}y}}
\le \frac{\abs{F\left( x_{0} \right)}}{\norm{[x_{0}]}_{X/M}}
    \end{align*}
    Since $x\in X$ was arbitrary, taking supremum, we have
    \begin{equation}
	N(F) \le \frac{\abs{F(x_{0})}}{\norm{x_{0}}_{X/M}} \leadsto N(F) \norm{x_{0}}_{X/M} \le \abs{F\left( x_{0} \right)}\text{.}
	\label{eqn:q10-1}
    \end{equation}
     To prove the reverse inequality, observe that if $y\in M$ then
     \begin{align*}
	 \frac{\abs{F\left( x_{0} \right)}}{\norm{x_{0}-m}} \le N(F) &\leadsto \frac{\abs{F\left( x_{0} \right)}}{N(F)} \le \norm{x_{0}-m}\text{.}
     \end{align*}
     Since $m$ is arbitrary, we have that
     \begin{align}
	 \frac{\abs{F\left( x_{0} \right)}}{N(F)} \le \norm{[x_{0}]}_{X/M}
	 \leadsto \abs{F\left( x_{0} \right)} \le N(F)  \norm{[x_{0}]}_{X/M}\text{.}
	 \label{eqn:q10-2}
     \end{align}
     Combining \ref{eqn:q10-1} and \ref{eqn:q10-2}, we have what we wanted.
\end{proof}

\section{Question 11}
\horz
Let $X$ be a finite dimensional normed linear space and $M$ be a proper closed subspace of $X$. Show that the unit sphere  $S := \{x: \|x\|=1\}$ on $X$ is compact. Use this to show that there exist a unit vector $x$ such that $\text{dist}(x, M) = 1.$ This need not to be true if $X$ is infinite dimensional. Show that the choice
    \begin{align*}
       & X = \{f \in C[0, 1] : f(0) = 0\}\\
        &M= \{f \in X : \int_{0}^{1}f = 0\}
    \end{align*}
provides a counter example. (This also shows that in F. Riesz's Lemma the constant $t$ can not be taken to be equal to 1 in general.)
\horz

\begin{proof}
    Note that $S$ is compact in view of Question 6.
    Consider the map $f: S \to \C$ given by 
    \begin{equation*}
	x \stackrel{f}{\mapsto} d\left( x, M \right)
    \end{equation*}
    Since we have $\abs{d\left( x,M \right) -d\left( y,M \right)} \le d\left( x,y \right)$, $f$ is continuous. We now show that $\sup f\left( S \right) = 1$.
    Let $x\in S$. Then we have that
    \begin{align*}
	d(x,M) &= \inf_{y \in M} \norm{x-y} \\
	&\le \norm{x}=1\text{.} \\
    \end{align*}
    Thus, $\sup f\left( S \right) \le 1$. Now, let $\varepsilon >0$ be arbitrary. Then by F. Riesz lemma, there exists a vector $x_{0} \in M$ such that 
    \begin{align*}
	1-\varepsilon < \norm{y-x_{0}} \text{for all } y \in M &\leadsto 1-\varepsilon \le \inf_{y\in M} \norm{y-x_{0}}=d(x_{0},M) \\
	& \leadsto 1\le \sup f(S) + \varepsilon
    \end{align*}
    Since $\varepsilon > 0$ is arbitrary, we have that $\sup f\left( S \right) \ge 1$. This shows that $\sup f\left( S \right) =1$.

    Now, continuous functions on compact sets achieve their supremum, therefore, there must be some vector $x \in S$ such that $f(x)=d(x,M) =1$.

    Now, we move on to the next part of the question. Consider the following sets:
    \begin{align*}
	X&= \left\{ f \in C[0,1] : f(0)=0 \right\} \\
	M&= \left\{ f\in X : \int_{0}^{1} f = 0 \right\}
    \end{align*}
    We need to show that there is no vector $f\in X$ whose $\norm{f}_{\infty}=1$ but $d(f,M)=1$.
\end{proof}

\section{Question 12}
\horz
Compute 
\begin{align*}
\min_{a,b,c \in\mathbb R} \int_{-1}^1|x^3-a-bx-cx^2|^2 dx
\end{align*}
and find 
\begin{align*}
\max_{g\in S} \int_{-1}^1 x^3g(x) dx,
\end{align*}
where $S= \bigg\{g\in L^2[-1,1] : \int_{-1}^{1} g(x)dx= \int_{-1}^{1} xg(x)dx = \int_{-1}^{1} x^2g(x)dx =0, \int_{-1}^{1} |g(x)|^2dx=1 \bigg\}.$
\horz

\section{Question 13}
\horz
Compute 
\begin{align*}
\min_{a,b,c\in \mathbb R} \int_{0}^{\infty}|x^3-a-bx-cx^2|^2e^{-x} dx
\end{align*}
\horz

\section{Question 14}
\horz
Fix a positive integer $N$, put $\omega= e ^{2\pi i/N}.$ prove the following orthogonality relations

\begin{align*}
\frac{1}{N} \sum\limits_{n=1}^N=\omega^{nk} = 
\begin{cases}
1, & k=0,\\
0, & 1\leqslant k \leqslant N-1.
\end{cases}
\end{align*}
Using this identity show that
\begin{align*}
\langle x,y\rangle = \frac{1}{N} \sum\limits_{n=1}^N \|x+\omega^n y\|^2 \omega^n
\end{align*} holds true in every inner product space provided $N\geqslant 3.$
\horz

\section{Question 15 \texorpdfstring{$\checkmark$}{\text{✓}}}
Consider the map $\varphi : V \to \C ^{n}$ given by 
\begin{align*}
    \varphi \left( v \right)
    =
    \begin{bmatrix}
	\ip{v, b_{1}} \\
	\vdots \\
	\ip{v, b_{n}}
    \end{bmatrix}
\end{align*}
for all $v\in V$. We show that this map $\varphi$ is injective. Our proof will be then complete by the rank nullity theorem.

So, let $v\in V$ and suppose that $\varphi \left( v \right) = 0$. Then $\ip{v, b_{i}} = 0$ for all $i=1,2, \ldots,  n$. Since $b_{1} , \ldots , b_{n}$ is a basis for $V$, there exists $\alpha _{1} , \ldots , \alpha _{n}$ such that
\begin{align*}
    v= \alpha_{1} b_{1} + \ldots + \alpha _{n} b_{n}
\end{align*}
Hence, we have that 
\begin{align*}
    \ip{v, v} &= \ip{v, \alpha_{1} b_{1} + \ldots + \alpha _{n} b_{n} } \\
    &= \sum_{i=1}^{n} \overline{\alpha_{i}} \ip{v, b_{i}} \\
    &= 0
\end{align*}
Hence $v=0$. This completes the proof!

\section{Question 16 \texorpdfstring{$\checkmark$}{\text{✓}}}
\horz
Let $V$ be a finite dimensional inner product space with inner product $\langle\cdot ,\cdot \rangle.$ Suppose $\beta_b= (b_1,b_2,\ldots,b_n)$ and  $\beta_e=(e_1,e_2,\ldots,e_n)$ are two ordered basis for $V$ which are related by the following relation:
$e_ j= \sum_{k=1}^n P_{k,j}b_k,$ for $j=1,2,\ldots,n.$ In short (in matrix multiplication notation) they are related by the following :
\begin{align*} \label{Basis change matrix}
(e_1,e_2,\ldots,e_n)= (b_1,b_2,\ldots,b_n)\begin{pmatrix}
P_{1,1} & P_{1,2}& \ldots & P_{1,n}\\
P_{2,1} & P_{2,2}& \ldots & P_{2,n}\\
\vdots & \vdots & \ddots & \vdots\\
P_{n,1} & P_{n,2}& \ldots & P_{n,n}
\end{pmatrix},  \,\,\mbox{that is,}\,\,\beta_e=\beta_b P.
%\,\,P= \begin{pmatrix}
%P_{1,1} & P_{1,2}& \ldots & P_{1,n}\\
%P_{2,1} & P_{2,2}& \ldots & P_{2,n}\\
%\vdots & \vdots & \ddots & \vdots\\
%P_{n,1} & P_{n,2}& \ldots & P_{n,n}
%\end{pmatrix},
\end{align*}
Let $G_e$ and $G_b$  be the Grammian matrix given by 
\begin{align*}
G_e= \begin{pmatrix}
\langle e_1,e_1\rangle & \langle e_2,e_1\rangle & \ldots & \langle e_n,e_1\rangle\\
\langle e_1,e_2\rangle & \langle e_2,e_2\rangle & \ldots & \langle e_n,e_2\rangle\\
\vdots & \vdots & \ddots & \vdots\\
\langle e_1,e_n\rangle & \langle e_2,e_n\rangle & \ldots & \langle e_n,e_n\rangle
\end{pmatrix}, \,G_b= \begin{pmatrix}
\langle b_1,b_1\rangle & \langle b_2,b_1\rangle & \ldots & \langle b_n,b_1\rangle\\
\langle b_1,b_2\rangle & \langle b_2,b_2\rangle & \ldots & \langle b_n,b_2\rangle\\
\vdots & \vdots & \ddots & \vdots\\
\langle b_1,b_n\rangle & \langle b_2,b_n\rangle & \ldots & \langle b_n,b_n\rangle
\end{pmatrix}
\end{align*}
(a) Show that  $G_e = {\bar{P}}^tG_bP,$ where $P$ is the matrix $(\!( P_{i,j})\!).$\\
(b) Show that the matrix $G_b$ is positive definite, that is, 
\begin{itemize}
\item[(i)] ${\bar{G_b}}^t= G_b,$ that is, $G_b$ is self adjoint,
\item[(ii)] $\langle G_b x,x\rangle_2 > 0$ for every non zero $x\in \mathbb C^n.$ Here $\langle\cdot,\cdot\rangle_2 $ denotes the standard Eucledian inner product on $\mathbb C^n.$
\end{itemize}
(c) Show that $\{e_1,e_2,\ldots,e_n\}$ is an orthonormal basis of $V$ if and only if $P {\bar{P}}^t={G_b}^{-1}.$\\
(d) Let $T$ be a linear map from $V$ into itself. Suppose the matrix representation of the linear map $T$ w.r.t the basis $\beta_b$ and $\beta_e$ is given by $[T]_{\beta_b}$
and $[T]_{\beta_e}$ respectively. Show that 
\begin{align*}
[T]_{\beta_e} = [T]_{{\beta_b}P} = P^{-1}[T]_{\beta_b} P.
\end{align*}
\horz

\begin{proof}[Solution]
    \begin{enumerate}[label=(\alph*)]
	\item Let $G_e$ and $G_b$ be the matrix whose entries are given by
   \begin{align*}
       \left( G_{e} \right)_{ij} = \ip{e_{j}, e_{i}}
   \end{align*}
   and 
   \begin{align*}
       \left( G_{b} \right)_{ij} = \ip{b_{j}, b_{i}}
   \end{align*}
   for each $i,j \in \left\{ 1,2,\ldots , n \right\}$.
   Consider the following:
   \begin{align*}
       \left( G_{e} \right) _{ij} &= \ip{e_{j}, e_{i}}  \\
       &= \ip{\sum_{k=1}^{n} P_{kj} b_{k}, \sum_{l=1}^{n} P_{li} b_{l}} \\
       &= \sum_{k=1}^{n} \sum_{l=1}^{n} P_{kj} \ip{b_{k}, b_{l}} \overline{P_{li}} \\
       &= \sum_{k=1}^{n} \sum_{l=1}^{n} \left( P^{*} \right)_{il} (G_{b} )_{lk} P_{kj} \\
       &= \left( P^{*} G_{b} P \right)_{ij}
   \end{align*}
   \item Clearly, we have that 
       \begin{align*}
	   \left( G_{e} ^{*} \right)_{ij} &= \left( \overline{G_{e}} \right) _{ji} \\
	   &= \overline{\ip{e_{i}, e_{j}}} \\
       &= \ip{e_{j}, e_{i}} \\
	   &=  \left( G_{e} \right)_{ij}
       \end{align*}
       for all $i,j$. Hence $G_{e}$ is self-adjoint. Observe that
       \begin{equation*}
	   \ip{G_{b} e_{j} , e_{i}}_{2} = \ip {b_{j}, b_{i}}
       \end{equation*}
       for all $i,j$. Now let $x\in V$. Then there exists unique $x_{1}, x_{2}, \ldots , x_{n}  \in \C$ such that
       \begin{align*}
	   x= \sum_{i=1}^{n} x_{i}e_{i}
       \end{align*}
        Then we have that
	\begin{align*}
	    \ip{G_{b} x , x}_{2} &= \ip{G_{b} \left( \sum_{j=1}^{n}  x_{j} e_{j}  \right), \sum_{i=1}^{n} x_{i} e_{i}} \\
	    &= \sum_{j=1}^{n} \sum_{i=1}^{n} x_{j}\overline{x_{i}} \ip{G_{b} e_{j}, e_{i}} \\
	    &= \sum_{j=1}^{n} \sum_{i=1}^{n} x_{j}\overline{x_{i}} \ip{b_{j}, b_{i}} \\
	    &= \ip{\sum_{j=1}^{n} x_{j} b_{j}, \sum_{i=1}^{n} x_{i} b_{i}}  \\
	    & \ge 0
	\end{align*}
	\item By definition we have that $G_{e} = I$ iff $\left\{ e_{1}, \ldots , e_{n} \right\}$ is orthonormal. ALso from item (a) we have that 
	    \begin{align*}
		\text{$\left\{ e_1 , e_2 , \ldots , e_{n}\right\}$ is orthonormal } &\Leftrightarrow I=P^{*} G_{b} P \\
		& \Leftrightarrow \left( P^{*} \right)^{-1} P^{-1} = G_{b} \\
		& \Leftrightarrow \left( PP^{*} \right)^{-1} G_{b} \\ 
		& \Leftrightarrow PP^{*} = G_{b}^{-1}
	    \end{align*}
	\item Clearly, $P = \left( I \right)_{\beta_{e}}^{\beta ^{b}}$. Then
	    \begin{align*}
		P \left( T \right)_{\beta_{e}} &=  \left( I \right)_{\beta^{e}} ^{\beta_{e}} \\
		&=  \left( I \circ T \right) _{\beta_{e}}^{\beta^{b}} \\
		&= \left( T \circ I \right) _{\beta_{e}}^{\beta^{b}} \\
		&= \left( T \right)_{\beta_{b}}^{\beta_{b}} \left( I \right)_{\beta_{e}}^{\beta_{b}} \\
		&= \left( T \right)_{\beta_{b}} P
	    \end{align*}
	    Hence, we have that 
	    \begin{equation*}
		\left( T \right)_{\beta_e} = P^{-1} \left( T \right)_{\beta_{b}} P
	    \end{equation*}
   \end{enumerate}
   This completes the proof.
\end{proof}

\section{Question 17 \texorpdfstring{$\checkmark$}{\text{✓}}}
\horz
Let $(V,\| \cdot\|)$ be a normed linear space where the norm $\|\cdot\|$ on $V$ satisfies the parallelogram law, that is,
\begin{align*}
\|x+y\|^2 + \|x-y\|^2= 2 \|x\|^2 + 2 \|y\|^2,\,\,\,x,y\in V.
\end{align*}
Show that the norm $\|\cdot\|$ is induced by an inner product on $V,$ that is, $\|x\|^2= \langle x,x\rangle$ for some inner product $\langle \cdot,\cdot \rangle$ on $V.$
\horz
\begin{proof}[Solution (as it was done in class).]
Assume $\left( V, \norm{} \right)$ is a real normed linear space which satisfies the parallelogram identity, that is, for all $a,b \in V$, 
\begin{equation*}
    \norm{a+b}^{2} + \norm{a-b}^{2} = 2\norm{a}^{2} +2 \norm{b}^{2}
\end{equation*}

We intend to define the inner product on $V$ by
\begin{align*}
    \ip{v,w} = \frac{\norm{x+y}^{2} - \norm{x-y}^{2}}{4}
\end{align*}

We show that $\ip{\cdot, \cdot}$ is an inner product.

The symmetric property is evident.

We proceed to show linearity in the first variable, that is, we need to show that 
\begin{align*}
    \norm{x_{1} + x_{2} + y}^{2} - \norm{x_{1}+x_{2}-y}^{2} = \norm{x_{1} + y}^2-\norm{x_{1}-y}^{2} + \norm{x_{2}+y}^{2} - \norm{x_{2}-y}^{2}
\end{align*}

Setting $a=x_{1} $ and $b=x_{2}+y$ in the parallelogram identity, we get
\begin{align*}
    \norm{x_{1} + y + x_{2}}^{2} + \norm{x_{1} - y - x_{2}}^{2} = 2 \norm{x_{1} } ^{2} + 2 \norm{x_{2} + y}^{2}
\end{align*}

Doing the same for $a=x_{2} -y $ and $b=x_{2}$, we have
\begin{align*}
    \norm{x_{1} - y + x_{2}}^{2} + \norm{x_{1} - y - x_{2}}^{2} = 2 \norm{x_{1} -y} ^{2} + 2 \norm{x_{2}}^{2}
\end{align*}

Subtracting the above two equations, we get
\begin{align*}
    \norm{x_{1} + x_{2} + y}^{2} - \norm{x_{1} - y + x_{2}} ^{2} = 2\norm{x_{1}}^{2} + 2 \norm{x_{2} + y} ^{2} - 2 \norm{x_{1} - y}^{2} - 2 \norm{x_{2}}^{2}
\end{align*}
 
Switching the roles of $x_{2}$ and $x_{1}$, we get
\begin{align*}
    \norm{x_{2} + x_{1} + y}^{2} - \norm{x_{2} - y + x_{1}} ^{2} = 2\norm{x_{2}}^{2} + 2 \norm{x_{1} + y} ^{2} - 2 \norm{x_{2} - y}^{2} - 2 \norm{x_{1}}^{2}
\end{align*}

Adding the above two equations, we get
\begin{align*}
    2\norm{x_{1} + x_{2} + y}^{2} - 2\norm{x_{1} +x_{2} -y }^{2} = 2 \norm{x_{2} + y}^{2} - 2 \norm{x_{1} -y}^{2} + 2 \norm{x_{1} + y} ^{2} -2 \norm{x_{2}-y}^{2}
\end{align*}

Rearranging the above equation, we observe that we have established what we wanted to prove!

Linearity is the other variable follows by symmetry and the linearity in the first variable!

Now, finally we proceed to show that for any $\lambda \in \R$, we have that 
\begin{align*}
    \ip{\lambda x, y} = \lambda \ip {x,y}
\end{align*}

Note that by linearity in the first variable, we have that for $n\in \Z$,
\begin{equation*}
    \ip{nx,y} = n \ip{x,y}
\end{equation*}
In a similar fashion, it can be shown that for $r\in \Q$,

\begin{equation*}
    \ip{rx,y} = r \ip{x,y}
\end{equation*}

Let us assume \textcolor{red}{Cauchy-Schwarz!} at the moment. Let $r\in \R$. Let $r_{n}$ be a sequence of rationals converging to $r\in \R$.

Observe that fixing $y\in V$, it is easily seen that 
\begin{equation*}
    x \mapsto \frac{\norm{x+y}^{2} - \norm{x-y}^{2}}{4}
\end{equation*}
is continuous by virtue of translation, norm and square of a function being continuous!
Then the result follows!

Irregardless, we prove Cauchy Schwarz! It can be seen by minimizing $r$ is the function $r \mapsto \norm{rx + y}^{2}$. One needs to see that for $r\in \Q$
\begin{align*}
    \norm{rx+y}^{2} = \ip{rx+y , rx +y}
    = r^{2} \norm{x}^{2} + 2r\ip{x,y} + \norm{y}^{2} \ge 0
\end{align*}
Hence the above holds for any $r\in \R$ by taking limits. Minimizing the function, we get the Cauchy Schwarz inequality.


We now proceed to the complex case!

Let $V, \norm{}$ be a complex normed linear space. By the polarization identity, we have that for $x, y \in V$,
\begin{align*}
    \ip{x,y} &= \frac{1}{4} \sum_{k=1}^{4} \norm{x+i^{k}y}^{2} i^{k} \\
    &= \frac{\norm{x+y}^{2} - \norm{x-y}^{2}}{4} - \frac{\norm{x+iy}^{2} - \norm{x-iy}^{2}}{4} \\
    &= q(x,y)+i q(x,iy)
\end{align*}
where $q(x,y)=  \frac{\norm{x+y}^{2} - \norm{x-y}^{2}}{4}$. We have already shown that $q$ is an inner product over $\R$. 

Now one can use the properties of inner product for $q$ to show that $\ip{\cdot , \cdot}$ is an inner product over $\C$.

\end{proof}

\section{Question 18}

\horz

Let $V$ be a finite dimensional inner product space with inner product $\langle\cdot ,\cdot \rangle.$ Suppose $W$ is a subspace of $V$ and $\beta_k=(v_1,v_2,\ldots,v_k)$ is a ordered basis for $W.$ Let $y=v_{k+1}$ is a vector outside $W.$
\begin{center}
\textbf{A distance formula to keep in mind} (Proof not needed)
\end{center}
Then the distance of $v_{k+1}$ from $W$ is given by $d(v_{k+1},W)= \frac{\sqrt{\det G_{\beta_{k+1}}}}{\sqrt{\det G_{\beta_k}}},$ where $G_{\beta_k}$ and $G_{\beta_{k+1}}$ are the Grammian matrix associated to the vectors $\beta_k=(v_1,v_2,\ldots,v_k)$ and $\beta_{k+1}=(v_1,v_2,\ldots,v_k,v_{k+1}).$\\

{\textbf{Sketch of the proof}} : Volume of the k dimensional parallelepiped formed by the vectors in $\beta_k$ $\times \,\, d(v_{k+1},W) = $ Volume of the (k+1) dimensional parallelepiped formed by the vectors in $\beta_{k+1}.$


\textbf{Problem}
Let $\mathcal P_3$ be the vector space of all polynomials over $\mathbb R$ of degree less than or equal to $3,$ with the inner product 
\begin{align*}
\langle f,g\rangle = \int_{0}^1f(t)g(t)dt,\,\,f,g\in \mathcal P_3.
\end{align*} 
Let $\mathcal P_2$ be the subspace of  $\mathcal P_3$ given by the set of all polynomials over $\mathbb R$ of degree less than or equal to $2.$ Find the distance of $x^3$ from $\mathcal P_2.$

\horz

\end{document}
