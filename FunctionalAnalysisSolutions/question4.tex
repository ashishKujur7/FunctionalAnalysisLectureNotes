\section{Question 4}
\horz
Let $M$ be a subspace of an inner product space $(V,\langle \cdot, \cdot \rangle).$ Show that $\overline{M},$ the closure of $M,$ in $V$ is also a subspace. Moreover show that $M^{\perp}= {\overline{M}}^{\perp}.$
\horz
\begin{proof}[Solution] We make the following claims:

\begin{claim}[orthogonal complement of a set and the orthogonal completement of its closure are same!]
    Let $M$ be a subset of a inner product space $H$. Then $M^{\perp} =  \left( \overline{M} \right)^{\perp}$
\end{claim}
\begin{proof}
    It follows by definition that $M \subset \overline M$ and hence $\left( \overline M \right)^{\perp} \subset M^{\perp}$. Now for reverse the inclusion, let $v \in M^{\perp}$ and let $y \in \overline M$. We need to show that $\ip{v,y} = 0$. Since $y\in \overline M$ there is a sequence $\left( y_{n} \right)$ in $M$ such that $y_{n} \to y$. Since $v \in M^{\perp}$, we have that $\ip {v , y_{n}}=0 $ for all $n\in \N$. Since $\ip{v, y_{n}} \to \ip{v,y}$, we have by uniqueness of limits that $\ip{v,y} = 0$. This completes the proof.
\end{proof}

\begin{claim}[orthogonal complement of orthogonal complement]
    
    Let $M$ be a closed subspace of the Hilbert space $H$. Then 
    \begin{align*}
	M = \left( M^{\perp} \right) ^{\perp}
    \end{align*}
    
\label{claim:o-comp-squared}
\end{claim}
\begin{proof}[Proof of Claim]
    Let us first show that $M \subset \left( M ^{\perp} \right)^{\perp}$ (which in fact holds for any set $M$). Let $v \in M$ and $w \in M^{\perp}$. It is clear by definition of $M^{\perp}$ that $\ip{v,w} = 0$. Hence, $v \in \left( M^{\perp} \right)^{\perp}$.

    Let us proceed to show the inclusion in the other direction. Let $v \in \left( M^{\perp} \right)^{\perp}$. Since $M$ is closed, by Projection Theorem, we have that $v = Pv + Qv$ where $Pv \in M$ and $Qv \in M^{\perp}$. By the previous paragraph, we have that $M \subset \left( M^{\perp} \right)^{\perp}$ and hence $Pv \in \left( M^{\perp} \right) ^{\perp}$. Hence, we have that $Qv \in \left( M^{\perp} \right)^{\perp}$. Now, $Qv \in M^{\perp} \cap \left( M^{\perp} \right) ^{\perp}$. Hence, $Qx = 0$ and thus, $v=Pv \in M$.
\end{proof}

Now, we start the proof. Let $M$ be subspace of $V$. Consider the following:
 \begin{align*}
    \left( M^{\perp} \right) ^{\perp} &= \left( \left( \overline M \right) ^{\perp} \right) ^{\perp} & \text{by Claim 1}  \\
    &= \overline{\overline{M}} & \text{by Claim 2} \\
    &= \overline M
    \end{align*}

    
\end{proof}
