\documentclass[14pt]{extarticle}
\usepackage[margin=1in]{geometry}
\usepackage{amsfonts, amsmath}
\usepackage[T1]{fontenc}
\usepackage{mathrsfs, enumitem}
\usepackage{dirtytalk,hyperref}
\usepackage[utf8]{inputenc}
\usepackage{amssymb}
\usepackage{amsfonts}
\usepackage{amsmath}
\usepackage{amsthm}
\usepackage{color}
\usepackage{hyperref}
\usepackage{csquotes}
\usepackage{fourier}
\usepackage{tikz-cd}

\newtheorem{theorem}{Theorem}[subsection]
\newtheorem{lemma}[theorem]{Lemma}
\newtheorem{claim}[theorem]{Claim}
\newtheorem{proposition}[theorem]{Proposition}
\newtheorem{corollary}[theorem]{Corollary}
\newtheorem{fact}[theorem]{Fact}
\newtheorem{notation}[theorem]{Notation}
\newtheorem{observation}[theorem]{Observation}
\newtheorem{conjecture}[theorem]{Conjecture}

\theoremstyle{definition}
\newtheorem{definition}[theorem]{Definition}
\newtheorem{example}[theorem]{Example}

\theoremstyle{remark}
\newtheorem{remark}[theorem]{Remark}
\theoremstyle{plain}
\newcommand{\ignore}[1]{}

% section symbol
\renewcommand{\thesection}{\S\arabic{section}}

% \renewcommand{\Pr}{{\bf Pr}}
% \newcommand{\Prx}{\mathop{\bf Pr\/}}
% \newcommand{\E}{{\bf E}}
% \newcommand{\Ex}{\mathop{\bf E\/}}
% \newcommand{\Var}{{\bf Var}}
% \newcommand{\Varx}{\mathop{\bf Var\/}}
% \newcommand{\Cov}{{\bf Cov}}
% \newcommand{\Covx}{\mathop{\bf Cov\/}}

% shortcuts for symbol names that are too long to type
\newcommand{\eps}{\epsilon}
\newcommand{\lam}{\lambda}
\renewcommand{\l}{\ell}
\newcommand{\la}{\langle}
\newcommand{\ra}{\rangle}
\newcommand{\wh}{\widehat}
\newcommand{\wt}{\widetilde}

% % "blackboard-fonted" letters for the reals, naturals etc.
\newcommand{\R}{\mathbb R}
\newcommand{\N}{\mathbb N}
\newcommand{\Z}{\mathbb Z}
\newcommand{\F}{\mathbb F}
\newcommand{\Q}{\mathbb Q}
\newcommand{\C}{\mathbb C}

% % operators that should be typeset in Roman font
% \newcommand{\poly}{\mathrm{poly}}
% \newcommand{\polylog}{\mathrm{polylog}}
% \newcommand{\sgn}{\mathrm{sgn}}
% \newcommand{\avg}{\mathop{\mathrm{avg}}}
% \newcommand{\val}{{\mathrm{val}}}

% % complexity classes
% \renewcommand{\P}{\mathrm{P}}
% \newcommand{\NP}{\mathrm{NP}}
% \newcommand{\BPP}{\mathrm{BPP}}
% \newcommand{\DTIME}{\mathrm{DTIME}}
% \newcommand{\ZPTIME}{\mathrm{ZPTIME}}
% \newcommand{\BPTIME}{\mathrm{BPTIME}}
% \newcommand{\NTIME}{\mathrm{NTIME}}

% values associated to optimization algorithm instances
\newcommand{\Opt}{{\mathsf{Opt}}}
\newcommand{\Alg}{{\mathsf{Alg}}}
\newcommand{\Lp}{{\mathsf{Lp}}}
\newcommand{\Sdp}{{\mathsf{Sdp}}}
\newcommand{\Exp}{{\mathsf{Exp}}}

% if you think the sum and product signs are too big in your math mode; x convention
% as in the probability operators
\newcommand{\littlesum}{{\textstyle \sum}}
\newcommand{\littlesumx}{\mathop{{\textstyle \sum}}}
\newcommand{\littleprod}{{\textstyle \prod}}
\newcommand{\littleprodx}{\mathop{{\textstyle \prod}}}

% horizontal line across the page
\newcommand{\horz}{
\vspace{-.4in}
\begin{center}
\begin{tabular}{p{\textwidth}}\\
\hline
\end{tabular}
\end{center}
}

% calligraphic letters
\newcommand{\calA}{{\cal A}}
\newcommand{\calB}{{\cal B}}
\newcommand{\calC}{{\cal C}}
\newcommand{\calD}{{\cal D}}
\newcommand{\calE}{{\cal E}}
\newcommand{\calF}{{\cal F}}
\newcommand{\calG}{{\cal G}}
\newcommand{\calH}{{\cal H}}
\newcommand{\calI}{{\cal I}}
\newcommand{\calJ}{{\cal J}}
\newcommand{\calK}{{\cal K}}
\newcommand{\calL}{{\cal L}}
\newcommand{\calM}{{\cal M}}
\newcommand{\calN}{{\cal N}}
\newcommand{\calO}{{\cal O}}
\newcommand{\calP}{{\cal P}}
\newcommand{\calQ}{{\cal Q}}
\newcommand{\calR}{{\cal R}}
\newcommand{\calS}{{\cal S}}
\newcommand{\calT}{{\cal T}}
\newcommand{\calU}{{\cal U}}
\newcommand{\calV}{{\cal V}}
\newcommand{\calW}{{\cal W}}
\newcommand{\calX}{{\cal X}}
\newcommand{\calY}{{\cal Y}}
\newcommand{\calZ}{{\cal Z}}

% bold letters (useful for random variables)
%----------------------------------------------
% \renewcommand{\a}{{\boldsymbol a}}
% \renewcommand{\b}{{\boldsymbol b}}
% \renewcommand{\c}{{\boldsymbol c}}
% \renewcommand{\d}{{\boldsymbol d}}
% \newcommand{\e}{{\boldsymbol e}}
% \newcommand{\f}{{\boldsymbol f}}
% \newcommand{\g}{{\boldsymbol g}}
% \newcommand{\h}{{\boldsymbol h}}
% \renewcommand{\i}{{\boldsymbol i}}
% \renewcommand{\j}{{\boldsymbol j}}
% \renewcommand{\k}{{\boldsymbol k}}
% \newcommand{\m}{{\boldsymbol m}}
% \newcommand{\n}{{\boldsymbol n}}
% \renewcommand{\o}{{\boldsymbol o}}
% \newcommand{\p}{{\boldsymbol p}}
% \newcommand{\q}{{\boldsymbol q}}
% \renewcommand{\r}{{\boldsymbol r}}
% \newcommand{\s}{{\boldsymbol s}}
% \renewcommand{\t}{{\boldsymbol t}}
% \renewcommand{\u}{{\boldsymbol u}}
% \renewcommand{\v}{{\boldsymbol v}}
% \newcommand{\w}{{\boldsymbol w}}
% \newcommand{\x}{{\boldsymbol x}}
% \newcommand{\y}{{\boldsymbol y}}
% \newcommand{\z}{{\boldsymbol z}}
% \newcommand{\A}{{\boldsymbol A}}
% \newcommand{\B}{{\boldsymbol B}}
% \newcommand{\C}{{\boldsymbol C}}
% \newcommand{\D}{{\boldsymbol D}}
% \newcommand{\E}{{\boldsymbol E}}
% \newcommand{\F}{{\boldsymbol F}}
% \newcommand{\G}{{\boldsymbol G}}
% \renewcommand{\H}{{\boldsymbol H}}
% \newcommand{\I}{{\boldsymbol I}}
% \newcommand{\J}{{\boldsymbol J}}
% \newcommand{\K}{{\boldsymbol K}}
% \renewcommand{\L}{{\boldsymbol L}}
% \newcommand{\M}{{\boldsymbol M}}
% \renewcommand{\O}{{\boldsymbol O}}
% \renewcommand{\P}{{\mathbb{P}}}
% \newcommand{\Q}{{\boldsymbol Q}}
% \newcommand{\R}{{\boldsymbol R}}
% \renewcommand{\S}{{\boldsymbol S}}
% \newcommand{\T}{{\boldsymbol T}}
% \newcommand{\U}{{\boldsymbol U}}
% \newcommand{\V}{{\boldsymbol V}}
% \newcommand{\W}{{\boldsymbol W}}
% \newcommand{\X}{{\boldsymbol X}}
% \newcommand{\Y}{{\boldsymbol Y}}
% \newcommand{\Z}{{\boldsymbol Z}}

% script letters
\newcommand{\scrA}{{\mathscr A}}
\newcommand{\scrB}{{\mathscr B}}
\newcommand{\scrC}{{\mathscr C}}
\newcommand{\scrD}{{\mathscr D}}
\newcommand{\scrE}{{\mathscr E}}
\newcommand{\scrF}{{\mathscr F}}
\newcommand{\scrG}{{\mathscr G}}
\newcommand{\scrH}{{\mathscr H}}
\newcommand{\scrI}{{\mathscr I}}
\newcommand{\scrJ}{{\mathscr J}}
\newcommand{\scrK}{{\mathscr K}}
\newcommand{\scrL}{{\mathscr L}}
\newcommand{\scrM}{{\mathscr M}}
\newcommand{\scrN}{{\mathscr N}}
\newcommand{\scrO}{{\mathscr O}}
\newcommand{\scrP}{{\mathscr P}}
\newcommand{\scrQ}{{\mathscr Q}}
\newcommand{\scrR}{{\mathscr R}}
\newcommand{\scrS}{{\mathscr S}}
\newcommand{\scrT}{{\mathscr T}}
\newcommand{\scrU}{{\mathscr U}}
\newcommand{\scrV}{{\mathscr V}}
\newcommand{\scrW}{{\mathscr W}}
\newcommand{\scrX}{{\mathscr X}}
\newcommand{\scrY}{{\mathscr Y}}
\newcommand{\scrZ}{{\mathscr Z}}

\newcommand{\im}{{\text{im }}}
\newcommand{\ip}[1]{\left\langle #1 \right\rangle}
\newcommand{\norm}[1]{\left\lVert #1 \right\rVert}

\title{Lecture Notes in Functional Analysis}
\author{Ashish Kujur}
\date{Last Updated: \today}

\begin{document}

\maketitle
\section*{Introduction}
This is a set of lecture notes which I took for reviewing stuff that I typed after taking class from \textit{Dr. Md. Ramiz Reza}. All the typos and errors are of mine.
\tableofcontents

\section*{References}
The following textbooks will be used for this course:
\begin{enumerate}
    \item John B. Conway -- A Course in Functional Analysis
    \item Walter Rudin -- Real and Complex Analysis
    \item Bhatia -- Notes on Functional Analysis
    \item Erwin Kreyzsig -- Introductory functional analysis with applications
\end{enumerate}

\section{Lecture 1  --- Introduction to Hilbert Spaces and some examples --- 9th January, 2023}

\subsection{Inner Product Spaces}
\begin{definition}[Inner Product]
    Let $V$ be a vector space over a field $\mathbb F$ (where $\mathbb F$ is $\R$ or $\C$). A function $\langle \cdot , \cdot \rangle : V \times V \to \mathbb F$ is called an \textit{inner product} if it satisfies the following properties
    \begin{enumerate}
	\item $\ip{x, x} \ge 0$
	\item $\ip{x+y, z} = \ip{x, z} + \ip{y, z}$
	\item $\ip{\alpha x, y} = \alpha \ip{x, y}$
	\item $\ip{x, y} = \overline{\ip{y, x}}$
    \end{enumerate}
    for all $x,y,z \in V$ and $\alpha \in \F$. A vector space $V$ with an inner product is called an \textit{inner product space}.
    \label{def:inner-product}
\end{definition}

\begin{example}[Examples of inner product spaces]
    Here are some examples of inner product spaces:
    \begin{enumerate}
	\item The obvious first example is that of $\R ^{n}$ with the standard $2$-inner product given by
	    \begin{equation*}
		\ip {x,y} = \sum_{i=1}^{n} x_{i} \overline{y_{i}}
	    \end{equation*}

	\item One can then consider the space $\ell ^{2} \left( \N \right)$ which is the vector space of all square summable sequences on $\C$. That is,
	    \begin{equation*}
		\ell ^{\infty} = \left\{ \left( x_{n} \right) \in \C ^{\N} \, \mid \, \sum_{i\in \N} |a_i|^{2} < \infty \right\} \end{equation*}
We define an inner product on this vector space by 
 \begin{equation*}
		\ip {x,y} = \sum_{i=1}^{\infty} x_{i} \overline{y_{i}}
	    \end{equation*}
	    One can show using Holder's inequality that the sum turns out to be finite and the "inner product" is indeed an inner product.
	\item Next, we consider the vector space of all polynomials over $\C$ which we denote by $\C \left[ x \right]$. If $p , q \in \C [x]$, we define an inner product on $\C \left[ x \right]$ by 
	    \begin{equation*}
		\ip{p,q} = \int_{0}^{1} p\overline q \, \mathrm{dx}
	    \end{equation*}
	\item One can define inner products on $C[0,1]$ and $L^{2} (X, \mathscr A , \mu)$ in an similar fashion as in item 3. Note that $\left( X, \mathscr A, \mu \right)$ is a measure space.
    \end{enumerate}
\end{example}

\begin{definition}
    Let $V$ be an inner product space. We can define a function $\norm{\cdot} : V \to \R _{\ge 0}$ by
    \begin{equation*}
	\norm{x} = \sqrt{\ip{x,x}}
    \end{equation*}
    We call this function \textit{norm induced by the inner product}. This norm is indeed a norm as one can check!
    \label{def:norm-induced-by-ip}
\end{definition}

The proof of the following theorems are skipped:
\begin{theorem}[Cauchy Schwarz inequality]
    Let $V$ be an inner product space, $x,y \in V$. Then we have that 
    \begin{equation*}
	\lvert \ip{x,y} \le \norm{x}\norm{y}
    \end{equation*}
    \label{thm:CS-inequality}
\end{theorem}

\begin{theorem}[Triangle Inequality]
    Let $V$ be an inner product space, $x,y \in V$. Then we have that 
    \begin{equation*}
	\norm{x+y} \le \norm{x} + \norm{y}
    \end{equation*}
    \label{thm:triangle-inequality}
\end{theorem}

\subsection{Hilbert Spaces}
\begin{definition}
    Let $V$ be an inner product space. One can consider $V$ as a metric space by defining the following metric $d$:
    \begin{equation*}
	d(v,w) = \norm{v-w}
    \end{equation*}
    for all $v,w \in V$. Then $(V,d)$ is a metric space \textcolor{red}{(Check!)}. We say that $V$ is a \textit{Hilbert Space} if $(V,d)$ is a complete metric space.
    \label{def:Hilbert-Space}
\end{definition}

\begin{example} We consider some examples and not-so-example of Hilbert Space:
\begin{enumerate}
    \item $\R ^{n}$ and $\C ^{n}$ with the standard inner product are complete!
    \item $\ell ^{2} (\N )$ is complete.
    \item $L^2(X)$ is complete where $(X, \mathscr A , \mu)$ is a measure space.
    \item $\mathcal {C} [0,1]$ is not complete. \textcolor{blue}{(Needs Baire Category Theorem!)}
    \item Consider$c_{00} = \left\{ \left( x_{n} \right)_{n\in \N} \in \ell ^{2} (\N) \, : \, \left( x_{n} \right) _{n \in \N} \text{ is eventually zero} \right\}$. $c_{00}$ has the induced inner product. We show that $c_{00}$ with this induced product is not complete!
	One consider the sequence of sequences given by
	\begin{equation*}
	    f_{n} = \left( 1, \frac{1}{2} , \ldots , \frac{1}{n} , \ldots \right)
	\end{equation*}
	One can then easily show that $\left( f_n \right)$ is Cauchy but it does not converge in $c_{00}$.
\end{enumerate}
\end{example}

\section{Lecture 2 --- \textit{Hilbert Spaces!} --- 11th January, 2023}
The important goal of this lecture is to show that if $H$ is a Hilbert Space then we show that under certain conditions an element can be projected onto a set. But before that, we prove the following theorem:

\begin{theorem}[Norm is uniformly continuous]
Let $H$ be a Hilbert space. The norm function on $H$, that is,  $\|\cdot\| : H \to \mathbb R$ given by $\|x\|= \sqrt{\langle x,x\rangle},\,\,x\in H,$ is continuous.
    \label{thm:norm-is-uc}
\end{theorem}
\begin{proof}
    Let $x,y \in H$. Then by the triangle inequality, we have the following:
    \begin{equation*}
	\norm{x} = \norm{\left( x-y \right) + y} \le \norm{x-y} + \norm{y}
    \end{equation*}
    and hence
    \begin{equation*}
	\norm{x} - \norm{y} \le \norm{x-y}
    \end{equation*}

    Interchanging the role of $x$ and $y$ in the previous inequality, we have htat 
    \begin{equation*}
	\norm{y} - \norm{x} \le \norm{x-y}
    \end{equation*}
    and thus, we have proved that
    \begin{equation*}
	\left\lvert \norm{x} - \norm{y} \right\rvert \le \norm{x-y}
    \end{equation*}
    which says that $\norm{\cdot}$ is uniformly continuous.
\end{proof}

Note that theorem \ref{thm:norm-is-uc} holds for any normed linear space, that is, there is no use of completeness there.

\subsection{Closed and Convex!}

\begin{theorem}
Let $S$ be a closed convex set in a Hilbert space $H.$ Let $x\in H.$ The distance of $x$ from $S,$ denoted as $d(x,S)$, is given by
\begin{align*}
d(x,S)= \inf\{ \|x-y\|: y\in S\}.
\end{align*} 
It follows that there exist a unique $s_0\in S$ such that $d(x,S)= \|x-s_0\|.$
    \label{thm:closed-and-convex}
\end{theorem}
\begin{proof}
    First of all, recall the parallelogram identity which holds for any innter product spaces, and hence in particular for Hilbert spaces,
    \begin{equation*}
	\norm{x+y}^{2} + \norm{x-y}^{2} = 2 \left( \norm{x}^{2} + \norm{y}^{2} \right)
    \end{equation*}
    The parallelogram law plays a crucial role in the proof of this theorem.
    Now, let's get busy to prove the theorem.
    First of all, by definition of infimum, we can find a sequence $\left( s_{n} \right)$ in $S$ such that $d (s_{n}, x) \to d \left( x,S \right)$. To be economical, let us denote $\delta := d \left( x,S \right)$. We show that $\left( s_n \right)$ is Cauchy sequence in $H$. To do so, let $\varepsilon >0$ be given.

    Observe that for any $n,m \in \mathbb N$,
    \begin{align*}
	\norm{\frac{x-s_{n}}{2} - \frac{x-s_m}{2}}^{2} + \norm{\frac{x-s_{n}}{2} + \frac{x-s_m}{2}}^{2} = \frac{1}{2} \left( \norm{x-s_{n}}^{2} + \norm{x-s_m}^{2} \right)
    \end{align*}
     and hence
     \begin{align}
	 \frac{1}{4} \norm{s_m - s_n}^{2} = \frac{1}{2} \left( \norm{x-s_{n}}^{2} + \norm{x-s_m}^{2} \right) - \frac{1}{4} \norm{x - \frac{s_n + s_m}{2}}^{2}
	 \label{eqn:2.1.1}
     \end{align}
     Now since $d\left( s_n ,x \right)$ converges to $\delta$, we must have that $d \left( s_{n} , x \right)^{2}$ converges to $\delta ^{2}$ and hence there is some $K\in \N$ such that for all $i\ge K$, 
     \begin{equation*}
	 \norm{x-s_{i}}^{2} < \delta^{2} + \frac{\varepsilon ^2}{4} 
     \end{equation*}
     Now for all $n,m \ge K$ and from equation \ref{eqn:2.1.1}, we have that 
\begin{align*}
    \norm{s_m - s_n}^{2} &= 2 \left( \norm{x-s_{n}}^{2} + \norm{x-s_m}^{2} \right) - \norm{x - \frac{s_n + s_m}{2}}^{2} \\
	 &\stackrel{(!)}{<} 2 \cdot 2\left( \delta ^{2} + \frac{\varepsilon ^2}{4} \right)-  4 \delta ^{2} \\
	 &= \varepsilon^2
     \end{align*}
     Note that in inequality $(!)$, we made use of the convexity of $S$ to conlude that $\frac{s_n + s_m}{2} \in S$. 
     This shows that $\left( s_{n} \right)$ is Cauchy. Now, since $H$ is a Hilbert space, $\left( s_{n} \right)$ must converge to some $s_{0} \in H$. Closedness of $S$ allows us to conclude that $s_0$ must be in $S.$

     Hence, $x -s_{n}$ converges to $x- s_{0}$. By Theorem \ref{thm:norm-is-uc}, we conclude that $\norm{x-s_{n}}$ converges to $\norm{x-s_0}$. Since $\norm{x-s_{n}}$ also converges to $\delta$, we have by uniqueness of limits that $\delta = \norm{x-s_{0}}$.

     It remains to prove the uniqueness of such a vector. Let us suppose that $s_0$ and $t_0$ be two vectors such that $\norm{x-s_{0}} = \norm{x-t_{0}}=\delta$.

     Applying parallelogram identity on the vectors $s_{0}$ and $t_0$ as in Equation \ref{eqn:2.1.1}, we get
\begin{align*}
	 \frac{1}{4} \norm{s_0 - t_0}^{2} &= \frac{1}{2} \left( \norm{x-s_{0}}^{2} + \norm{x-t_0}^{2} \right) - \frac{1}{4} \norm{x - \frac{s_0 + t_0}{2}}^{2} \\
	 & \le \delta^2 -  \norm{x - \frac{s_0 + t_0}{2}}^{2} \\
	 & \le 0
     \end{align*}
     Hence, $s_0 = t_0$ and this completes the proof of the theorem.
\end{proof}

\begin{example}[distance is achieved but the vector may not be unique] Consider the normed linear space $\left( \R ^{2}, \norm{\cdot} _{1} \right)$. Now consider the subset $S$ of $\R ^2$ given by  $$S= \left\{ \left( x_{1} , x_{2} \right) : x_{1} + x_{2} =1 \right\}.$$ 
Note that $d \left( (0,0),S \right)=1= d\left( 0,0 \right), \left( 1,0 \right)) = d\left( \left( 0,0 \right), \left( 0,1 \right) \right)$. Hence, the uniqueness is not guaranteed.
\end{example}

\begin{exercise}
    Consider the space $\left( C[0,1] \right)$ with the supremum norm $\| \cdot\|_{\infty},$ that is, $\|f\|_{\infty}= \sup \big\{|f(x)| : x\in[0,1] \big \}.$ Let $S$ be the set
    \begin{equation*}
	S=\left\{ f \in C[0,1] \, : \, \int_{0}^{1/2} f\left( x \right) dx - \int_{1/2}^{1} f \left( x \right) dx =1 \right\}.
    \end{equation*}
    Show that the set $S$ is closed and convex but the distance $ d(0,S)=1,$ is never achieved at any point in $S$. That is, it is not the case that there is some $f\in S$ such that $d(0,f)= \norm{f}_{\infty} = 1$.
\end{exercise}
\begin{proof}[Solution]
    We begin by showing that $S$ is convex. Let $f, g \in S$ and $t\in [0,1]$. Then we have that 
    \begin{align*}
	\int_{0}^{1/2} \left( t f\left( x \right) + \left( 1-t \right) g\left( x \right)\right) dx - \int_{1/2}^{1} \left( t f\left( x \right) + \left( 1-t \right) g\left( x \right) dx  \right) &= t + (1-t) \\
	&= 1
    \end{align*}
    Note that the second equality follows by the virtue of $f,g \in S$.

    Now, we proceed to show that the $S$ is closed. Let $\left( f_{n} \right) $ be a sequence of functions in $S$ converging to $f \in C\left[ 0,1 \right]$. We need to prove that $f \in S$. Now convergence in supremum norm is the same as the uniform convergence, so, we have that following:
    \begin{align*}
	\lim_{n\to \infty}\left( \int_{0}^{1/2} f_{n}\left( x \right) dx - \int_{1/2}^{1} f_{n} \left( x \right) dx \right) =1
    \end{align*}
    implies 
    \begin{align*}
\int_{0}^{1/2} f\left( x \right) dx - \int_{1/2}^{1} f \left( x \right) dx =1 
\end{align*}
and thus $f \in S$.
Consider the zero function and the set $S$, we show that that there is no $f \in S$ such that $d (0,S) = d(f,0)=\norm{f}_{\infty}$. \textcolor{red}{Incomplete!}
\end{proof}

\subsection{Projections}
\begin{theorem}
Let $H$ be a Hilbert space. For any fixed $y\in H,$ consider the map $L_y : H \to \mathbb C$ defined by $L_y(x)= \langle x, y \rangle,\,\,x\in H.$ Then $L_y$ is a continuous linear functional on $H.$ 
    \label{thm:ip-linear-functional}
\end{theorem}

\begin{proof}
    Let $y\in H$ be fixed. Consider the function $L_{y} : H \to \C$ given by $L_{y} \left( x \right) = \ip{x,y}$ for each $x \in H$. We show that $L_{y}$ is Lipschitz continuous.
    Let $x_0 \in H$. If $x\in H$, we have that
    \begin{align*}
	\abs{L_{y} \left( x \right) - L_{y} x_{0}} &= \abs{\ip {x,y}-\ip{x_{0},y}}	\\
	&= \abs{\ip{x-x_{0},y}} \\
	& \le \norm{x-x_{0}} \norm{y}
    \end{align*}
    Note the inequality follows from Cauchy Schwarz and this completes the proof.
\end{proof}

\begin{definition}
    Let $H$ be a Hilbert space. For any $y\in H,$ the symbol $y^{\perp}$ denote the subspace defined by
\begin{align*}
    y^{\perp}:= \{x\in H : \langle x,y\rangle=0\}
    \label{def:orthonormal-of-subspace}
\end{align*}
\end{definition}

Observe that $y^{\perp}$ is a closed subspace of $H$. This is because $y^{\perp}$ is the kernel of the continuous map $L_{y}$ as given by Theorem \ref{thm:ip-linear-functional}.

\begin{definition}
    Let $H$ be a Hilbert space. Let $M$ be any subspace of $H$. Let the symbol $M^{\perp}$ denote the subspace given by 
   \begin{align*}
M^{\perp} = \{x\in H : \langle x,y\rangle =0\,\mbox{for all}\,y\in M\}= \bigcap_{y\in M} y^{\perp}.  \end{align*} 

\end{definition}

Observe that $M^{\perp}$ is always closed since it is intersection of closed subspaces of $H$.

\begin{theorem}[Existence of an Orthogonal Projection onto a closed subspaces]
Let $M$ be a closed subspace of a Hilbert space $H.$ Then 
\begin{itemize}
\item [(a)] Every $x\in H$ has a unique decomposition 
\begin{align*}
x= Px + Qx
\end{align*}
into a sum of $Px\in M$ and $Qx\in M^{\perp}.$ Thus $H = M \oplus M^{\perp}.$
\item [(b)] $Px$ and $Qx$ are the nearest points to $x$ in $M$ and in $M^{\perp}$ respectively.

\item [(c)] The mappings $P: H \to M$ and $Q: H \to M^{\perp}$ are linear and satisfies $P^2=P$ and $Q^2=Q.$ The map $P$ and $Q$ are called the \textbf{orthogonal projection onto $M$ and $M^{\perp}$} respectively.

\item [(d)] $\|x\|^2= \|Px\|^2 + \|Qx\|^2$ for every $x\in H.$
\end{itemize}
\label{thm:existence-of-o-proj}
\end{theorem}

\begin{proof}
    Since subspaces are convex, we can appeal to Theorem \ref{thm:closed-and-convex} as we please. We now start to prove each of the statements of theorem:
    \begin{itemize}
	\item [(a)] Let $x \in H$ be arbitrary. Then by the Theorem \ref{thm:closed-and-convex} there is a unique vector $Px \in M$ such that 
	    \begin{equation*}
		d(x, M) = \norm{x-Px}
	    \end{equation*}
	    Define $Qx \in M$ by $Qx = x-Px$. We need to show that $Qx \in M^{\perp}$. Let $y \in M$. We want to show that $\ip{x-Px , y} = 0$. To do so, observe that
	    \begin{align}
		\ip{Qx - \ip{Qx,y} \frac{y}{\norm{y}^{2}},y}&= \ip{Qx,y} - \ip{\ip{Qx,y} \frac{y}{\norm{y}^{2}},y}
		=0
		\label{eqn:2.2.1}
	    \end{align}

	    Now, 
	    \begin{align*}
		Qx = \underbrace{\left( Qx - \ip{Qx, y} \frac{y}{\norm{y}^{2}} \right)}_{=: v_1} + \underbrace{\ip{Qx,y} \frac{y}{\norm{y}^{2}}}_{=:v_2}
	    \end{align*}

	    Note that by equation \ref{eqn:2.2.1}, $v_1$ and $v_2$ are orthogonal and hence by Pythagoras theorem\footnote{In inner product space, if $\ip{v_{1}, v_{2}}=0$ then $\norm{v_{1} + v_{2}}^{2} = \norm{v_{1}}^{2} + \norm{v_{2}}^{2}$} for inner product spaces, we may write
\begin{align*}
    \delta ^{2} = \norm{Qx}^{2} &= \norm{ Qx - \ip{Qx, y} \frac{y}{\norm{y}^{2}}}^{2} + \norm{\ip{Qx,y} \frac{y}{\norm{y}^{2}}}^{2} \\
    &= \norm{ Qx - \ip{Qx, y} \frac{y}{\norm{y}^{2}}}^{2} + \frac{\abs{\ip{Qx,y}}}{\norm{y}^{2}} \\
    &= \norm{ x- Px - \ip{Qx, y} \frac{y}{\norm{y}^{2}}}^{2} + \frac{\abs{\ip{Qx,y}}}{\norm{y}^{2}}
    &\ge \delta ^{2} +  \frac{\abs{\ip{Qx,y}}}{\norm{y}^{2}}
	    \end{align*}

	    and thus, we have that $\abs{\ip{Qx,y}} = 0$. This completes the proof of (a).
	\item [(b)] By uniqueness of part (a), it follows that $Px$ is the nearest point to $x$ in $M$. It remains to prove that $Qx$ is the nearest point to $x$ in $M^{\perp}$. Note that $x-Qx=Px\in M.$ Now for any $y\in M^{\perp}$ we have  $Qx-y\in M^{\perp}.$ Thus we get 
	\begin{align*}
	\|x-y\|^2= \|(x-Qx) + (Qx -y)\|^2 = \|x-Qx\|^2 + \|Qx -y\|^2 \geqslant \|x-Qx\|^2.
	\end{align*}
This shows that $Qx$ is the nearest point to $x$ in $M^{\perp}.$	

	\item [(c)] Let $x_{1}, x_{2} \in M$. By part $(a)$, we have that 
	    \begin{align*}
		x_{1} &= Px_{1} + Qx_{1} \\
		x_{2} &= Px_{2} + Qx_{2} \\
		x_{1} + x_{2} &= P\left( x_{1} + x_{2} \right) + Q \left( x_{1} + x_{2} \right)
	    \end{align*}

	    Now taking sums and rearranging, we have that
	    \begin{equation*}
		\underbrace{Px_{1} + Px_{2} - P\left( x_{1} + x_{2} \right)}_{\in M} = \underbrace{Q\left( x_{1} + x_{2} \right) - Qx_{1} - Qx_{2}}_{\in M^{\perp}}
	    \end{equation*}
	    Since $M \cap M^{\perp} = \left\{ 0 \right\}$, the linearity of $P$ and $Q$ follows.

	    Now, let $x\in P$. We need to prove that $P^{2}x = Px$. Now note that $Px \in M$. Thus by part (a) we have
	    \begin{equation*}
		Px = P^{2} x + QPx
	    \end{equation*}
	    By uniqueness of part (a), we must have that $Px = P^{2}x$. This completes the proof. $Q^{2}= Q$ can be proved similarly.

	\item [(d)] This follows immediately from Pythagoras theorem.
    \end{itemize}
\end{proof}

\begin{corollary}
Let $M$ be a closed subspace of a Hilbert space $H.$ Then $(M^{\perp})^{\perp} = M.$ In case $M$ is a subspace then $(M^{\perp})^{\perp} = \overline{M},$ the closure of $M$ in $H.$
\label{cor:M-perp-perp}
\end{corollary}




%\section{Lecture 3 --- \textit{Riesz Representation Theorem for Hilbert Spaces} --- 13th January, 2023}

\subsection{Existence of closed subspaces of Hilbert Spaces}

\subsection{Hilbert Spaces}

\subsection{Projections and Orthonormal Sets in finite dimensions\ldots}

%\section{Lecture 4 --- \textit{Projection for subspaces having countable orthonormal basis} --- 17th January, 2022} 
\subsection{Projections and Orthonormal Sets in finite dimensions\ldots}
Suppose $M$ is a finite dimensional subspace of an Hilbert space $H$ and dimension of $M$ is $n.$ Let $\mathcal B = \{u_{i} : 1 \leqslant i\leqslant n\}$ be an orthonormal basis of $M$ (existence of such basis follows from the Gram-Schmidt orthogonalization process to any basis of $M$). For any $x\in H,$ consider the vector $\sum\limits_{j=1}^n \langle x, u_j\rangle u_j$ in $M.$  Thus we obtain

\begin{align*}
x=\Big( x-\sum\limits_{j=1}^n \langle x, u_j\rangle u_j \Big)  + \Big( \sum\limits_{j=1}^n \langle x, u_j\rangle u_j \Big)
\end{align*}
We verify that $x-\sum_{j=1}^n \langle x, u_j\rangle u_j \in M^{\perp}$. Observe that for all $1\le i \le n$,
\begin{align*}
    \ip{x - \sum_{j=1}^{n} \ip{x,u_{j}}u_{j}, u_{i}} &= \ip{x, u_{i}} - \sum_{j=1}^{n} \ip{x,u_{j}}\ip{u_{j}, u_{i}}\\
    &=\ip{x,{u_{i}}} - \ip{x,{u_{i}}} = 0
\end{align*}


Now by the uniqueness of the decomposition (see Theorem \ref{thm:existence-of-o-proj}) $H= M \oplus M^{\perp},$ it follows that 
\begin{align*}
P_{_M}x &= \sum_{j=1}^n \langle x, u_j\rangle u_j,\\
P_{_{M^{\perp}}}x &= x-  \sum_{j=1}^n \langle x, u_j\rangle u_j,
\end{align*}
where $P_{_M}$ and $P_{_{M^{\perp}}}$ denotes the orthogonal projection onto $M$ and $M^{\perp}$ respectively. Furthermore, it follows that 
\begin{align}
\|P_{_M}x\|^2 = \sum_{j=1}^n |\langle x, u_j\rangle|^2 = \|x\|^2- \|P_{_{M^{\perp}}}x\|^2 \leqslant \|x\|^2,\,\,x\in H.
\label{eqn:Bessel-in-finite}
\end{align}

\subsection{Generalising projections to infinite dimensions\ldots}
\begin{proposition}
Suppose $\mathcal B = \{u_{i} :  i\in \mathbb N\}$ is an orthonormal set in a Hilbert space $H.$ Then for any $x\in H,$ we have $$ \sum_{j=1}^{\infty} |\langle x, u_j\rangle|^2 \leqslant \|x\|^2.$$
This is known as the Bessel's inequality. Let $M= \overline{\text{span } \{u_i: i\in\mathbb N\}}$ be the smallest closed subspace spanned by the orthonormal set $\mathcal B.$ It follows that $S_n(x) = \sum_{j=1}^{n} \langle x, u_j\rangle u_j $ is a Cauchy sequence in $M.$  Hence the limit $\lim_{n\to\infty} S_n(x) =  \sum_{j=1}^{\infty} \langle x, u_j\rangle u_j $ exist in the Hilbert space $M$. Moreover, $P_{_M}(x),$ the orthogonal projection of $x$ onto $M,$ is given by
\begin{align*}
P_{_M}(x)= \sum_{j=1}^{\infty} \langle x, u_j\rangle u_j = \lim_{n\to\infty} S_n(x).
\end{align*}
Furthermore equality occurs in the Bessel's inequality if and only if $x\in M.$
\end{proposition}
\begin{proof}
    Fix a natural number $k$ and consider $M_k= \text{span} \{u_j: 1\leqslant j \leqslant k\}.$ By Equation \ref{eqn:Bessel-in-finite} we find that 
\begin{align*}
\|P_{_{M_k}}x\|^2 = \sum_{j=1}^k |\langle x, u_j\rangle|^2 \leqslant \|x\|^2,\,\,x\in H,
\end{align*}
where where $P_{_{M_k}}$ denotes the orthogonal projection onto $M.$ Since this holds true for every $k\in \mathbb N,$ it follows that
$$ \sum_{j=1}^{\infty} |\langle x, u_j\rangle|^2 \leqslant \|x\|^2.$$ 
For $S_n(x) = \sum_{j=1}^{n} \langle x, u_j\rangle u_j ,$  note that 
\begin{align*}
\|S_m(x) - S_k(x)\|^2= \sum_{j=k+1}^m |\langle x, u_j\rangle|^2,\,\, \,\,~~~~k\geqslant m.
\end{align*}
Since $\sum_{j=1}^{\infty} |\langle x, u_j\rangle|^2 < \infty ,$ it follows that $\{S_n(x)\}_{n\in\mathbb N}$ is a Cauchy sequence in $M$ and converges to some vector in $M.$ Let $z= \lim_{n\to\infty} S_n(x).$ It is straightforward to verify that $\langle x-S_n(x), u_j\rangle = 0$ for every $n\geqslant j.$ Since $x-z= \lim_{n\to\infty} (x-S_n(x)),$ it follows that $\langle x-z,u_j\rangle =0.$ This holds true for any $j\in \mathbb N.$ Hence 
we get that $x-z\in M^{\perp}.$ Thus we have
\begin{align*}
x= z+ (x-z) ,\,\,\,x-z\in M^{\perp},\,z\in M.
\end{align*}
By the uniqueness of the decomposition $H= M \oplus M^{\perp},$ we obtain that 
\begin{align*}
P_{_M}(x)= z= \sum_{j=1}^{\infty} \langle x, u_j\rangle u_j = \lim_{n\to\infty} S_n(x),\,\, P_{{_M}^{\perp}}(x)=x-z.
\end{align*}

Now, we proceed to show that the Bessel inequality holds iff $x\in M$. Suppose that $x\in M$. Then we have that $P_{M} \left( x \right) = x$ and hence $x = z + (x-z) = x + (x-z)$ and hence $x=z$. But note that $\norm{z} ^{2} = \norm{\sum_{j=1}^{\infty}\ip{x, u_{j}} u_{j}} = \sum_{j=1}^{\infty} \abs{\ip{x,u_{j}}}$ but this follows from 
\begin{align*}
    \norm{\sum_{j=1}^{\infty}\ip{x, u_{j}} u_{j}} = \sum_{j=1}^{\infty} \abs{\ip{x,u_{j}} }
\end{align*} and then taking $n\to \infty$, we get
\begin{align}
    \norm{\sum_{j=1}^{n}\ip{x, u_{j}} u_{j}} = \sum_{j=1}^{n} \abs{\ip{x,u_{j}}}  \label{eqn:Bessel-in-infinite}
\end{align}
Conversely, suppose that equality in Bessel's inequality holds then from equalion \ref{eqn:Bessel-in-infinite} and the fact that $\norm{x} ^{2} = \norm{z} ^{2} \norm{x-z} ^{2}$ that $\norm{x-z} ^{2} = 0$ and hence $x\in M$.
\end{proof}

\begin{corollary}\label{cor:span-closure-of-onb}
    Suppose $\mathcal B = \{u_{i} :  i\in \mathbb N\}$ is a maximal orthonormal set in a Hilbert space $H.$ Let $M= \overline{\text{span} \{u_i: i\in\mathbb N\}}$ be the smallest closed subspace spanned by the orthonormal set $\mathcal B.$ Then it follows that $M=H$ and for any $x\in H$ we have 
\begin{align*}
x= \sum_{j=1}^{\infty} \langle x, u_j\rangle u_j = \lim_{n\to\infty} S_n(x),\,\,\, \|x\|^2= \sum_{j=1}^{\infty} |\langle x, u_j\rangle|^2.
\end{align*}
\end{corollary}
\begin{proof}
If $M^{\perp}\neq 0,$ then consider a non zero unit vector $u$ in $M^{\perp}.$ Then $\mathcal B \cup \{u\}$ is another family of orthonormal set containing $\mathcal B.$ This contradicts the maximality of $\mathcal B.$ Hence by the maximality of $\mathcal B ,$ it follows that $M^{\perp} = \{0\}.$ Now the corollary follows from the proposition.
\end{proof}

\subsection{Existence of a maximal orthonormal set in a inner product space} Let $(X, \leqslant)$ be the collection of all orthonormal set in $V$ equipped with the partial ordering of set inclusion, that is, for $A,B\in X,$ we have $A\leqslant B$ if $A \subseteq B.$ It is straightforward to verify that if $\mathscr C$ is a chain (totally ordered set) in the partially ordered set  $(X,\leqslant),$ then the chain $\mathscr C$ has an upper bound in $X$ namely the union of the members of $\mathscr C.$ Hence by Zorn's Lemma it follows that $X$ has a maximal element, that is, $V$ has a maximal orthonormal set. Now, we make a definition:

\begin{definition}[orthonormal basis]
    \label{def:orthonormal-basis}
    A maximal orthonormal set in a Hilbert space is called an \textit{orthonormal basis of the Hilbert space}. 
\end{definition}

\subsection{Separability of Hilbert Spaces}
\begin{proposition}
Let $H$ be a Hilbert space. Then $H$ is separable (that is it has a countable dense set) if and only if $H$ admits an at most countable orthonormal basis. 
\label{prop:separable-hilbert-spaces}
\end{proposition}
\begin{proof}
Suppose $\mathcal B = \{u_{\alpha} : \alpha \in I\}$ be an collection of orthonormal set in $H.$ It is straightforward to verify that $\|u_{\alpha}-u_{\beta}\|= \sqrt{2}$ for every $\alpha,\beta\in I$ with $\alpha \neq \beta.$ Thus  the collection of balls $\{B(u_{\alpha},\frac{\sqrt{2}}{2}) : \alpha \in I\}$ are pairwise disjoint. If $I$ is uncountable then we have uncountable such balls  which are pairwise disjoint. This contradicts any existence of countable dense set in $H.$ Thus if $H$ is separable then any orthonormal collection in $H$ has to be at most countable (finite or countably infinite). Hence any maximal orthonormal set in $H$ must be at most countable. This proves that for a separable Hilbert space $H$ we have an at most countable orthonormal basis.

For the converse direction assume that $H$ admits a countable orthonormal basis, say $\mathcal B = \{u_{i} : i \in \mathbb N\}.$ Let $D = \cup_{n\in\mathbb N} D_n,$ where $D_n$ is given by
\begin{align*}
D_n= \Big\{ \sum\limits_{j=1}^n c_j u_j : c_j \in \mathbb Q + i \mathbb Q \Big\}
\end{align*}
Note that each $D_n$ is countable and hence $D$ is countable. It is  straightforward to see that $\overline{D_n} = \text{span } \{u_j : 1\leqslant j \leqslant n\}$ for each $n\in\mathbb N.$ It follows that $\text{span } \{u_j : j\in \mathbb N\} \subseteq \overline{D}.$ This gives us that $\overline{\text{span } \{u_j : j\in \mathbb N\}} \subseteq \overline{D}.$ In view of the Corollary \ref{cor:span-closure-of-onb} we obtain that $D$ is dense in $H$ and hence $H$ is separable.
\end{proof}



%\section{Lecture 5 --- \textit{Examples Galore!} --- 18th January, 2023}


\subsection{Properties of Finite Dimensional Hilbert Spaces}
\begin{proposition}
Suppose $H$ is a finite dimensional Hilbert space. Then $H$ is isometrically isomorphic to the Euclidean space $(\mathbb C^n,\langle \cdot,\cdot \rangle_2),$  where $\langle \cdot,\cdot\rangle_2$ is the standard Euclide
\end{proposition}

\begin{proposition}
 Suppose $H$ is a separable infinite dimensional Hilbert space. Then $H$ is isometrically isomorphic to $\ell^2(\mathbb N).$
 \end{proposition}

\subsection{Revisiting the examples of Hilbert spaces}
\begin{itemize}
\item[(a)] Consider $\ell^2(\mathbb N).$ Let $e_n$ be the sequence in $\ell^2(\mathbb N)$ defined by $e_n(i)=\delta_{i,n}$ for every $i,n \in \mathbb N.$ Then the set $\{e_n : n\in\mathbb N\}$ forms an orthonormal basis for $\ell^2(\mathbb N).$
\item[(b)] Consider $c_{00}$ as a subspace of $\ell^2(\mathbb N).$ It is an inner product space but not complete w.r.t the metric induced by the associated inner product. In fact $c_{00}$ is dense in $\ell^2(\mathbb N),$ that is, the closure of $c_{00}$ in $\ell^2(\mathbb N)$ is the space $\ell^2(\mathbb N)$ itself. 
\item[(c)] Consider $(L^2(\mathbb T), d\sigma),$ where $\sigma$ is the Lebesgue measure (the normalised arc length measure) on the unit circle $\mathbb T.$ Consider the function $z^n : \mathbb T \to \mathbb C$ given by $$z^n(e^{it}) = e^{int},\,t\in [0,2\pi).$$ Note that 
\begin{align*}
\langle z^n, z^m\rangle_{L^2(\mathbb T)} = \frac{1}{2\pi} \int_{0}^{2\pi} e^{i(n-m)t} dt,\,\,n,m\in\mathbb Z.
\end{align*}
Thus $\{z^n: n\in\mathbb Z\}$ forms an orthogonal set. In view of Stone-Weirstrass Theorem, we have $\text{span }\{z^n: n\in\mathbb Z\}$ is dense in $\mathscr C(\mathbb T)$ in uniform norm $\|\cdot\|_{\infty}.$ Since $\|f\|_{2} \leqslant\|f\|_{\infty}$ for every $f\in \mathscr C(\mathbb T),$ it follows that $span\{z^n: n\in\mathbb Z\}$ is dense in $\mathscr C(\mathbb T)$ in $L^2$ norm $\|\cdot\|_{2}.$ 
\end{itemize}


%\section{Lecture 6 --- \textit{insert title here!} --- 23rd January, 2023}

\subsection{newSubsection}

%% !TEX root = main.tex
\section{Lecture 7 --- \textit{insert title here} --- 25th January, 2023}
\horz
\subsection{Possibly, final}

\horz

\subsection{Class Sketch}
\horz

\subsubsection{Definition}
Let $V$ be a vector space over $\R$ or $\C$ . A norm $\norm{ \cdot } : V \to \R _{\ge 0}$ is a funtion (we call this the length) if the following properties are satisfied:
\begin{enumerate}
    \item $\norm{v} \ge 0$ for all $v\in V$
    \item $\norm{v} = 0$ iff $v=0$
    \item $\norm{cv} = \abs{c}\norm{v}$ for all $c\in F$
    \item $\norm{v+w} \le \norm{v} + \norm{w}$
\end{enumerate}

One can define a function on a normed linear space $d(v,w) = \norm{v-w}$. It is not hard to see that $(V,d)$ is a metric space. $(V, \norm{}$ is called a Banach space if the metric space $(V,d)$ is complete where $d$ is the metric induced by the norm.
    

    \subsubsection{Examples}
    Every hilbert space is a banach space since inner product induces a norm.


    $L^{p} \left( X,\mu \right)$ is Banach space where $(X, \mu)$ is any measure space where $p\in [1,\infty)$.

    $X= \left\{ 1,\ldots , n \right\}$ and $\mu$ is the counting measure then $L^{p} \left( X, \mu \right)$ is the $\C ^{n}$ with the $\norm{x}_{p} = \left( \sum_{i=1} ^{n} \abs{x_{i}}^{p} \right) ^{1/p}$ for $p\in [1, \infty)$ and $\norm{x}_{\infty} = \max_{1\le i \le n} \abs{x_{i}}$


    One can also consider case such that $X=[0,1]$ or $X= \R$ with $\mu=\lambda$ the Lebesgue measure.

    Also, $X=C[0,1]$ with $\norm{f}_{\infty} = \sup_{x\in [0,1]} \abs{f(x)}$.

    Consider $X=\N$ and $\mu$ be the counting measure, we get what we call $\ell ^{p} \left( \N \right)$ whenever $p\in [1,\infty)$. If $p=\infty$, we have $\ell ^{\infty} \left( \N \right)$, we have the set of all bounded sequences. \textcolor{red}{explicitly try to write out the norms!}
    

    Consider $C[0,1]$ as a subset of $L^{p} [0,1]$, $1 \le p < \infty$. For any $p$, then $C(X)$ is a dense subset of $L^{p} (X)$ where $X$ is a locally compact Hausdorff space.

    %%%%%%%%%%%%%%%%%%%%%%%%%%%%%%%%%%%%%%%%%%%%%%%%%%%%%%%%%%%%%%%%%%%%%%%%%

    \horz

    See tutorial problem for an example of a Banach space such that the distance of point and a closed convex set is not achieved!

    Let $B$ be a Banach space. Let $M$ be a subspace of $B$. Does there exist subspace $N$ such that $B=M\oplus N$?

    Consider $\R ^{2}$ with $1$-norm. Consider the subspace $M$ spanned by $e_{1}$. Then any subspace spanned by a single vector linearly dependent does the work.
    \horz

    If $H$ is a Hilbert space then $\norm{x+y} ^{2} + \norm{x-y}^{2} = 2\left( \norm{x}^{2} + \norm{y}^{2} \right)$.

    Suppose $\left( V, \norm{} \right)$ is a normed space. Suppose that for all $x,y \in V$ , we have that  $\norm{x+y} ^{2} + \norm{x-y}^{2} = 2\left( \norm{x}^{2} + \norm{y}^{2} \right)$. We claim that $\norm{\cdot }$ is induced by a unique inner product.

    One can define 
    \begin{align*}
	\ip{v,w} = \frac{1}{4} \sum_{k=1}^{4} \norm{v+i^{k} w } ^{2}
    \end{align*}

    It can be shown that the above defines an inner product indeed!

    \horz

    Let $V$ be a real inner product space. Then we have that 
    \begin{align*}
	\norm{v+w}^{2} - \norm{v-w} ^{2} = 4 \ip{v,w}
    \end{align*}

    Suppose $V, \norm{}$ satisfies the parallelogram law then one can define the inner product as above and one can check that it is indeed an inner product.


\horz

A set $\left\{ u_{i} : i \in \N \right\}$ is called a Schauder basis of a normed linear space $( V, \norm{ } )$ if there exists a \textit{unique} sequence in $\F$, $\left\{ c_{i} : i \in \N \right\}$, $x= \lim_{n} \sum_{i=1}^{n} c_{i} u_{i} = \sum_{i=1}^{\infty} c_{i} u_{i}$ for every $x\in V$.

Any basis of a finite dimensional vector space is a Schauder basis.

Consider $\ell ^{1} \left( \N \right)$. As expected, $\left\{ e_{i} : i \in \N \right\}$ is a Schauder basis for $\ell ^{1} \left( \N \right)$. Let $x= \left( x_{i} \right) \in \ell ^{1} \left( \N\right)$. Then the sequence terms give the necessary sequence, that is,
\begin{align*}
    S_{n } = \sum_{i=1}^{n} x_{i} e_{i}
\end{align*}
We claim that $S_{n} \to x$ in $\ell ^{1}  \left( \N \right)$. that is, $\norm{S_{n} - x}_{1} \to 0$. Note that
\begin{equation*}
    \norm{S_{n} - x}_{1} = \sum_{i=n+1} ^{\infty} \abs{x_i} \to 0 \text{ as } n\to \infty
\end{equation*}

It remains to show uniqueness. Let $x= \left( x_{i} \right)$. Suppose $x=\sum c_{i} e_{i} = \sum d_{i} e_{i}$. But this is the same as $\sum c_{i} e_{i} = 0$.  Consider $p_{n} = \sum_{i=1}^{n} c_{i} e_{i}$ and hence $p_{n} \to 0$. Hence $\norm{p_{n}} \to 0$. Thus, $\sum_{i=1}^{n} \abs{c_{i}} \to 0$ and hence $c_{i} = 0$ for all $i\in \N$ and this completes the proof of the claim.


Orthonormal basis are always an example of a separable Hilbert space (Verify!)


\horz

\begin{proposition}
    Let $B$ be a Banach space which admits a Schauder basis $\left\{ u_{i} : i \in \N \right\}$. Then $B$ is separable that is, it admits a countable dense set.
\end{proposition}

We consider an example first. In the case of $\ell ^{2} \left( \N \right)$, consider the set
\begin{equation*}
    \cup_{n=1}^{\infty} \left\{ \sum c_{i} e_{i} \, \mid \, \Q + i \Q \right\}
\end{equation*}
This does the job!


We claim the set
\begin{equation*}
    \cup_{n=1}^{\infty} \left\{ \sum c_{i} u_{i} \, \mid \, \Q + i \Q \right\}
\end{equation*}

Consider $M_n = \text{span} \left\{ u_{1}, \ldots , u_{n} \right\}$ and $D_{n}$ be the set of vectors which are span of $u_{1}, \ldots , u_{n}$ with rational coefficients. Then closure of $D_{n}$ is $M_n$. Then 
\begin{align*}
    \bigcup_{n\in \N} \overline{D_{n}} \subset M_{n} \subset \text{span } \left\{ u_{i} : i \in \N \right\}
\end{align*}
 
Since $\left\{ u_{i} \right\}$ is a Schauder basis, we can write $B= \overline {\text{span } \left\{ u_{i} : i \in \N \right\}}$. Observe that $\cup_{n=1}^{\infty} \overline{D_{n}} \subset \overline {D}$. But then $\overline {D_{n}} = M_{n}$. Hence, we have that $\text{span } \left\{ u_{i} : i \in \N \right\} = \cup_{n=1}^{\infty} M_{n} \subset \overline {D}$. Taking closure again, we have that  \textcolor{red}{complete this!}


\horz

One can show that $\left\{ e_{i} : i\in \N \right\}$ is not a Schauder basis for $\ell ^{\infty} \left( \N \right) $. In fact, it does not admit a Schauder basis at all!

%\section{Lecture 8 --- \textit{Equivalence of norms} --- 30th January, 2023}

\horz

\subsection{Equivalence of Norms}
\begin{definition}
    Let $V$ be a vector space. Let $\norm{}_{1}$ and $\norm{}_{2}$ be two norms on $V$. We say that $\norm{}_{1}$ and $\norm{}_{2}$ are equivalent if there exists positive constants $C_{1}, C_{2}$ such that
    \begin{align*}
	C_{2} \norm{v}_{1} \le \norm{v}_{2} \le C_{1} \norm{v}_{1}
    \end{align*}
    for all $v\in V$.
    \label{def:equivalent-norms}
\end{definition}

Let $B_{i} \left( 0,1 \right)$ be the unit ball of $i$ norm. Then by definition, we have that
\begin{equation*}
    B_{1} \left( 0,1 \right) \subset B_2 \left( 0, c_{2} \right)
\end{equation*} 
and
\begin{equation*}
    B_{2} \left( 0,1 \right) \subset B_{1} \left( 0,\frac{1}{c_{2}} \right)
\end{equation*}

\begin{example}
    Consider $\R ^{2}$ with $1$-norm and $2$-norm. It can be shown that
    \begin{equation*}
	\norm{x}_{2} \le \norm{x}_{1}
    \end{equation*}
    It follows from Cauchy Schwarz that
    \begin{equation*}
	\abs{x_{1}} + \abs{x_{2}} \le \sqrt{2} \sqrt{\abs{x_{1}}^{2} + \abs{x_{2}}^{2}}
    \end{equation*}
    and hence we have that
    \begin{equation*}
	\norm{x}_{1} \le \sqrt{2} \norm{x}_{2}
    \end{equation*}
    Hence, the norms are equivalent.
\end{example}

\begin{proposition}
    Let $V$ be a vector space. Let $\norm{}_{1}$ and $\norm{}_{2}$ be two equivalent norms on $V$. A sequence $\{ x_{n} \}$ converges to $x$ in one norm iff it converges in the other.
    \label{prop:equivalent-sequences-converge}
\end{proposition}

\begin{corollary}
    Let $V$ be a vector space. Let $\norm{}_{1}$ and $\norm{}_{2}$ be two norms on $V$. Let $A$ be subset of $V$. Then closure of $A$ in $1$-norm is the same as the closure of $A$ in $2$-norm.
    \label{cor:equivalent-implies-same-closure}
\end{corollary}

\begin{corollary}
    Let $V$ be a vector space. Let $\norm{}_{1}$ and $\norm{}_{2}$ be two norms on $V$. Let $A$ be subset of $V$. $A$ is closed (corr. open) in $1$-norm iff $A$ is closed (corr. open) in $2$-norm.
    \label{cor:equivalent-implies-same topology}
\end{corollary}

\begin{example}
    \label{ex:p-norm-equiv-1-norm}
    It follows by Holder's inequality that in $\R^{n}$, any $p$-norm is equivalent to $1$-norm.
    \begin{proof}[Proof of Example \ref{ex:p-norm-equiv-1-norm}]
	Recall Holder's inequality, which states that if $a,b \in \C^{n}$ and $p,q$ are conjugate exponents then
	\begin{align*}
	    \sum_{i=1}^{n} \abs{a_{i} b_{i}} \le \norm{a}_{p} \norm{b}_{q}
	\end{align*}
	Now if $x\in \C^{n}$ and taking $y=\left( 1,\ldots , 1 \right)$ we have that
	\begin{align*}
	    \norm{x}_{1} = \sum_{i=1}^{n} \abs{x_{i}} \le 2^{1/q}\norm{x}_{p} 
	\end{align*}
	and also note that
	\begin{align*}
	    \norm{x}_{\infty} \le \norm{x}_{1}
	\end{align*}
	Hence it follows that $1$-norm and $p$-norm are equivalent for any $p \ge 1$.
    \end{proof}
    \end{example}

    \begin{example}[Looking for norms which are not equivalent]
	Consider $c_{00}$ which is the span of $e_{i}$. It is evident that
	\begin{align*}
	    \norm{x}_{\infty} \le \norm{x}_{1}
	\end{align*}
	for every $x\in c_{00}$.
	But consider the sequence $A_{n} = \left( \underbrace{\frac{1}{n} , \ldots , \frac{1}{n}}_{n \text{ terms}}, \ldots \right)$. Note that this sequence converges to $0$ in $\infty$ norm, however, this does not converge in the $1$ norm for the very simple reason that
	\begin{equation*}
	    \norm{A_{n}}_{1} =1
	\end{equation*}
	for every $n\in \N$.
    \end{example}

    \begin{proposition}
	Suppose $V$ be a normed linear space. Let $Y = \text{span } \left\{ v_{1,\ldots , v_{n}} \right\} \subset V$ where $v_{i}$ is a basis. Then
	there exists positive constants such that
	\begin{align*}
	    C \norm{x}_{2} \le \norm{\sum_{i=1}^{n} x_{i} v_{i}} \le M \norm{x}_{2}
	\end{align*}
	\label{prop:finite-dim-noob}
    \end{proposition}
    \begin{proof}
	Observe that
	\begin{align*}
	    \norm{\sum_{i=1}^{n} x_{i} v_{i}} \le \sum_{i=1}^{n} \abs{x_{i}} \norm{v_{i}}
	    \stackrel{\text{CS inequality}}{\le }M \norm{x}_{2}
	\end{align*}
	where $M= \left( \abs{\norm{v_{1}}^{2} + \ldots + \norm{v_{n}}^{2}} \right)^{1/2}$.

	Consider the function $f$ on the unit sphere $S$ given by
	$f(x)=\norm{\sum_{i=1}^{n} x_{i} v_{i}}$ for every $x\in S$.

	Now we claim that $f$ is continuous.
	Consider
	\begin{align*}
	    \abs{f(x)-f(y)} = \abs{\norm{\sum_{i} x_{i} v_{i}}  - \norm{\sum_{i} y_{i} v_{i}} } \le M \norm{x-y}_{2}
	\end{align*}

	Since $f(x)>0$ for every $x\in S$ and since $f$ is continuous, we have hthat there is some $x_{0} \in S$ such that $\delta = f(x_{0}) = \inf_{x\in S} f\left( x \right)$.

	Thus, we have that $\norm{\sum_{i=1}^{n} x_{i} v_{i}} \ge \delta$ for every $x\in S$.

	Now, let $x\in V$. Then $\left( \frac{x_{1}}{\norm{x}}, \ldots , \frac{x_{n}}{\norm{x}} \right) \in S$ and the result follows.
    \end{proof}

    \begin{corollary}
	Let $\left( C^{n}, \norm{} \right)$ be a normed linear space. Then this normed linear space is equivalent with the usual Euclidean norm on $\C ^{n}$.
    \end{corollary}
    \begin{proof}
	Take $v_{i}$ to be standard basis.
    \end{proof}

    \begin{corollary}
	Let $Y$ be a finite dimensional subspsace of a normed linear space $V$. Then $Y$ is complete (and hence closed).
    \end{corollary}
    \begin{proof}
	Let $Y$ be a finite dimensional closed subspace. Let $v_{1}, \ldots , v_{k}$ be a basis for $Y$. Let $\left\{ x_{n} \right\}$ be a Cauchy sequence in $Y$.
    Suppose that for each $n \in \N$, we have that
    \begin{equation*}
	x_{n} = a_{n1} v_{1} + \ldots + a_{nk} v_{k}\text{ .}
    \end{equation*}

    Let $\left\{ a_{n} \right\}$ be the sequence in $\C ^{n}$ given
    \begin{align*}
	a_{n} = \left( a_{n1}, \ldots , a_{nk} \right)
    \end{align*}
    for each $n\in \N$.
    Then from Proposition \ref{prop:finite-dim-noob}, we have for $n,m \in \N$ and $i\in \left\{ 1,2, \ldots , k \right\}$, we have that
    \begin{align*}
	C |a_{ni}-a_{mi}| \le C \norm{a_{n}-a_{m}}_{2} \le \norm{x_{n} - x_{m}} \le M \norm{a_{n}-a_{m}}_{2}
    \end{align*}
    Since $\left\{ x_{n} \right\}$ is Cauchy, we have that each $a_{ni}$ is Cauchy and hence converges at some $a_{i}$.

    Now, we show that $\left\{ x_{n} \right\}$ converges to the vector
    \begin{align*}
	x:=a_{1} v_{1} + \ldots + a_{k} v_{k}
    \end{align*}
    This is easy to see by Cauchy Schwarz inequality that
    \begin{align*}
	\norm{x_{n} -x} \le M \norm{a_{n} - a}_{2}
    \end{align*}
    This completes the proof.
    \end{proof}
     
    \begin{exercise}
	Consider $C[0,1]$. Show that that sup norm and any $p$ norm are not equivalent.
    \end{exercise}
    \begin{proof}[Solution]
	Does this work?
    \end{proof}

    \begin{exercise}
	$c_{00}$ and the set of all polynomials are not complete.\\
	Hint: Use BCT!
    \end{exercise}
    \begin{proof}
	A space generated by an infinite but countably many number of linearly independent vectors $v_{1}, v_2 , \ldots$ cannot be complete! Consider the subspaces $M_{n} =\operatorname{span} \left[ v_{1}, v_{2}, \ldots , v_{n} \right]$. Each of which are finite dimensional and hence closed and do not have an interior,so, by Baire's Theorem, we have that they cannot be complete!
    \end{proof}

%\section{Lecture 9 --- \textit{Baire Category Theorem, Riesz Lemma, Quotient Spaces} --- 1st February, 2023}
\horz

\subsection{Consequence of Baire's Theorem}

It follows from Baire's theorem that
\begin{proposition}
    If $X$ is a Banach space then any Hamel basis of $X$ is uncountable.
    \label{prop:B-space-implies-uncountable-Hamel-basis}
\end{proposition}

Some interesting other consequences are that: $\R^{2}$ is not a union of countably many straight lines.

\subsection{Riesz says NO to compact unit balls in infinite dimensions}

We saw that in a Hilbert Space, the unit ball in compact iff the space is of finite dimension.

Now, consider the unit ball of $\ell ^{1}$
\begin{align*}
    B_{1} \left( 0 \right) = \left\{ \left( x \right) \mid \norm{x} \le 1 \right\}
\end{align*}
Note that $\left\{ e_{i} \right\}$ is a sequence in the unit ball. Note that for $n,m \in \N$,
\begin{align*}
    \norm{e_{n} - e_{m}}_{1} =2
\end{align*}
This says that $\ell ^{1}$  is not complete.

Now consider $\ell ^{\infty}$. The same argument works but the distance between any two distinct sequences $e_{n}$ and $e_{m}$ is $1$.

The same argument works for $\ell ^{p}$.

This proves the following:
\begin{proposition}
    Unit ball of $\ell ^{p}$ is not compact.
\end{proposition}

\begin{proposition}[Riesz Lemma]
    Let $X$ be a NLS, $M$ is a closed subspace of $X$. Fix $t \in (0,1)$. Then there is $x_{0} \in X$ such that $\norm{x_{0}}=1$ and $d\left( x_{0}, M \right) \ge t$.
\end{proposition}

We see an application before the proof:
Suppose $\dim X = \infty$. So, let $M_{1}$ be the span of some nonzero vector $v\in X$ whose norm is $1$. This $M_1$ is closed. By the lemma, there is $v_{2}$ in the unit circle such that $d\left( v_{2}, M_{1} \right) \ge 1/2$.

Now consider $M_{2}$ be the span of $v_{1}$ and $v_{2}$. Again by the lemma, there is $v_{3}$ in the unit circle such that $d \left( v_{3} , M_{2} \right) \ge 1/2$.

Suppose that we have obtained a sequence $v_{1}, v_{2}, \ldots, v_{n}$ such that $M_{n}$ is a span of $v_{1}, \ldots , v_{n}$ and then repeating the argument, we can obtained $v_{n+1}$ in the unit circle such that $d \left( v_{n+1} , M_{n} \right) \ge 1/2$.

Now, observe that $\norm{v_{j} - v_{k}} \ge 1/2$. Hence the unit ball cannot be closed.

\begin{proof}[Proof of Riesz Lemma]
    Consider $y \not \in M$. Then $\delta := d(y, M) > 0$ because $M$ is closed.Consider $\delta /t > \delta$. Since $\delta$ is the infimum of distances between $y$ and the points of $M$. We can find $m_{0} \in M$ such that $\norm{y-m_{0} }< \frac{\delta}{t}$.

    Take $x_{0} = (y-m_{0})/\norm{y-m_{0}}$. Note that norm of $x_{0}$ is $1$.

    Then we have that 
    \begin{align*}
	\norm{x_{0} - m} &= \norm{\frac{y-m_{0}}{\norm{y-m_{0}}} - m } \\
	&= \frac{1}{\norm{y-m_{0}}} \norm{y-\underbrace{m_{0}- m \norm{y-m_{0}}_{\in M}}} \\
	    &\ge \frac{\delta}{ \norm{y-m_{0}}} >t
    \end{align*}

This shows that $d \left( x_{0}, M \right) \ge t$.
\end{proof}

\begin{theorem}
    In an infinite dimensional normed linear space, neither the unit circle nor the unit disc is compact.
    \label{thm:no-to-compact-unit-balls}
\end{theorem}
\begin{proof}
    Let $X$ be an infifnite dimensional normed linear space. Select a vector $x_{1}$ in the unit circle, that is, $x_{1}$ has unit norm. Now span of $x_{1}$ is a finite dimensional subspace of $X$ hence must be closed, in view, of previous lecture. Hence, we can apply Riesz lemma, to obtain a vector $x_{2}$ of unit norm such that
    \begin{equation*}
	\norm{x_{2} - \alpha x_{1}} > \frac{1}{2} \text{ for every } \alpha \in \F\text{.}
    \end{equation*}
    Now, we can do the same with span of the vector $x_{2}$ and $x_{1}$ to get a third vector $x_{3}$ of unit norm such that 
\begin{equation*}
    \norm{x_{3} - \alpha x_{1} - \beta x_{2}} > \frac{1}{2} \text{ for every } \alpha, \beta \in \F\text{.}
    \end{equation*}
    Continuing this way, we obtain a sequence of vectors in the unit circle, satisfying, $\norm{x_{n}-x_{m}}> \frac{1}{2}$. Hence, we have obtained a sequence in the unit circle which has no hope of having a converging subsequence, hence, the unit circle is not compact. 

    Note that the same argument shows that the closed unit disc is not compact as well.
\end{proof}

Let $X, Y$ be two normed linear space. One can construct another NLS by $X \oplus Y$ in the following way:
\begin{equation*}
    \norm{\left( x_{1} , x_{2} \right)}_{1} = \norm{x_{1}}_{X} + \norm{x_{2}}_{Y} 
\end{equation*}
and in general in the $p$ norm style.

\subsection{Quotient Space}
Let $X$ be a normed linear space, $M$ be a closed subspace. Then
\begin{align*}
    X/M = \left\{ [x] : x\in X \right\}
\end{align*}

One can define $\norm{[x]} = \inf \left\{ \norm{x-m} : m \in M \right\}$. One can show that with this norm, $X/M$ becomes a NLS indeed!

But is $X/M$ a Banach space? Yes:
\begin{proposition}
    If $X$ is complete then $X/M$ is also complete.
    \label{prop:quotient-are-complete}
\end{proposition}

example: $M=\left\{ \left( x \right) : x_{1} =0 \right\}$ of $c_{00}$.

\begin{lemma}
    Let $X$ be a Banach space. If $\left\{ v_{j} \right\}$ is absolutely summable, that is, $\sum_{j=1}^{\infty} \norm{v_{j}} < \infty$. Then $S_{n} = \sum_{j=1}^{n} v_{j}$ is convergent. 
    \label{lemma:Banach-absolutely-summable}
\end{lemma}
\begin{proof}
    For $n \ge m$, consider the following:
    \begin{align*}
	\norm{S_{n} - S_{m}} &= \norm{\sum_{j=m+1}^{n} v_{j}} \\
	& \le  \sum_{j=m+1}^{n} \norm{v_{j}}
    \end{align*}
    Since $\sum_{j=1}^{\infty} \norm{v_j} < \infty$, we have that its sequence of partial sums is Cauchy and the former inequality is also Cauchy. Since $X$ is a Banach space, $S_{n}$ converges. This completes the proof.
\end{proof}

\begin{theorem}
    Let $X$ be a NLS. The following are equivalent:
    \begin{enumerate}
	\item $X$ is complete
	\item Every absolutely summable sequence is summable.
    \end{enumerate}
    \label{thm:equivalent-thm}
\end{theorem}


\end{document}
