\section{Lecture 11 --- \textit{Linear Maps, I guess!} --- 08th February, 2022}

Let $X$ and $Y$ be two normed linear spaces. We define
\begin{align*}
    L(X,Y) =\left\{ T: X\to Y \, : \, T \text{ linear and continuous } \right\}
\end{align*}

This is a vector space with the following operations:
\begin{align*}
    (T_{1} + T_{2})(x) = T_{1} (x) + T_{2} (x) \\
    (\alpha T ) (x) = \alpha T(x)
\end{align*}
for all $x,y \in X$.

We can also give it a norm by:
\begin{equation*}
    \norm{T}=\sup \left\{ \norm{Tx}_{Y} : \norm{x}_{X} \le 1 \right\}
\end{equation*}

Note that this indeed defines a norm because if $T$ is continuous then there is some $M > 0$ such that $\norm{Tx} \le M \norm{x}$ for each $x \in V$. So $\norm{T} < \infty$. (check the other properties like triangle inequality!)

\begin{observation}
    If $T$ is continuous then $\norm{Tx} \le \norm{T}\norm{x}$ for each $x\in V$. This is easy to see if $x=0$. If $x\ne 0$ then 
    \begin{align*}
	\norm{T \left( \frac{x}{\norm{x}} \right)} &\le \norm{T} \\
	\leadsto \norm{Tx}_{Y} &\le \norm{T} \norm{x}
    \end{align*}
\end{observation}

\begin{lemma}
    Let $X,Y$ be two normed linear spaces. If $Y$ is a Banach space then so is $B(X,Y)$ with the operator norm.
    \label{lem:B(X,Y)-banach}
\end{lemma}
\begin{proof}
    Let $\left\{ T_{n} \right\}$ be a Cauchy sequence in $B(X,Y)$. We claim that for each $x\in X$, we have that $\left\{ T_{n}x \right\}$ is Cauchy in $Y$. This is easy to see by the following:
    \begin{align*}
	\norm{T_{n} x -  T_{m} x} \le \norm{T_{n} - T_{m}} \norm{x}
    \end{align*}
    Thus, $\left\{ T_{n}x \right\}$ is Cauchy in $Y$. Hence, we define the following function $T: X\to Y$ we define 
    \begin{align*}
	\lim_{n\to \infty} T_{n} x =: T (x)
    \end{align*}

It is easy to see that $T$ is linear. Now, we need to check that $T$ is continuous. Since $\left\{ T_{n} \right\}$ is Cauchy, we have that $\left\{ T_{n} \right\}$ are bounded. So there is some $M>0$ such that $\norm{T_{n}} \le M$. Then we have that 
\begin{align*}
    \norm{T_{n} x} \le \norm{T_{n}} \norm{x} &\le M \norm{x} \\
    \leadsto \norm{Tx} &\le M \norm{x} & \text{taking limits}
\end{align*}
Thus, $T$ is continuous. Now, we proceed to show that $T_{n}$ actually converges to the linear operator $T$.

Let $\varepsilon > 0$ be given. Then there is some $K \in \N$ such that for every $n,m\ge K$, we have that
\begin{align*}
    \norm{T_{n} - T_{m}} < \varepsilon
\end{align*}

\end{proof}

\begin{corollary}
    Dual space of any normed linear space is complete!
    \label{cor:dual-space-is-complete}
\end{corollary}


