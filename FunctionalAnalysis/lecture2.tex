\section{Lecture 2 --- \textit{Hilbert Spaces!} --- 11th January, 2023}
The important goal of this lecture is to show that if $H$ is a Hilbert Space then we show that under certain conditions an element can be projected onto a set. But before that, we prove the following theorem:

\begin{theorem}[Norm is uniformly continuous]
Let $H$ be a Hilbert space. The norm function on $H$, that is,  $\|\cdot\| : H \to \mathbb R$ given by $\|x\|= \sqrt{\langle x,x\rangle},\,\,x\in H,$ is continuous.
    \label{thm:norm-is-uc}
\end{theorem}
\begin{proof}
    Let $x,y \in H$. Then by the triangle inequality, we have the following:
    \begin{equation*}
	\norm{x} = \norm{\left( x-y \right) + y} \le \norm{x-y} + \norm{y}
    \end{equation*}
    and hence
    \begin{equation*}
	\norm{x} - \norm{y} \le \norm{x-y}
    \end{equation*}

    Interchanging the role of $x$ and $y$ in the previous inequality, we have htat 
    \begin{equation*}
	\norm{y} - \norm{x} \le \norm{x-y}
    \end{equation*}
    and thus, we have proved that
    \begin{equation*}
	\left\lvert \norm{x} - \norm{y} \right\rvert \le \norm{x-y}
    \end{equation*}
    which says that $\norm{\cdot}$ is uniformly continuous.
\end{proof}

Note that theorem \ref{thm:norm-is-uc} holds for any normed linear space, that is, there is no use of completeness there.

\subsection{Closed and Convex!}

\begin{theorem}
Let $S$ be a closed convex set in a Hilbert space $H.$ Let $x\in H.$ The distance of $x$ from $S,$ denoted as $d(x,S)$, is given by
\begin{align*}
d(x,S)= \inf\{ \|x-y\|: y\in S\}.
\end{align*} 
It follows that there exist a unique $s_0\in S$ such that $d(x,S)= \|x-s_0\|.$
    \label{thm:closed-and-convex}
\end{theorem}
\begin{proof}
    First of all, recall the parallelogram identity which holds for any innter product spaces, and hence in particular for Hilbert spaces,
    \begin{equation*}
	\norm{x+y}^{2} + \norm{x-y}^{2} = 2 \left( \norm{x}^{2} + \norm{y}^{2} \right)
    \end{equation*}
    The parallelogram law plays a crucial role in the proof of this theorem.
    Now, let's get busy to prove the theorem.
    First of all, by definition of infimum, we can find a sequence $\left( s_{n} \right)$ in $S$ such that $d (s_{n}, x) \to d \left( x,S \right)$. To be economical, let us denote $\delta := d \left( x,S \right)$. We show that $\left( s_n \right)$ is Cauchy sequence in $H$. To do so, let $\varepsilon >0$ be given.

    Observe that for any $n,m \in \mathbb N$,
    \begin{align*}
	\norm{\frac{x-s_{n}}{2} - \frac{x-s_m}{2}}^{2} + \norm{\frac{x-s_{n}}{2} + \frac{x-s_m}{2}}^{2} = \frac{1}{2} \left( \norm{x-s_{n}}^{2} + \norm{x-s_m}^{2} \right)
    \end{align*}
     and hence
     \begin{align}
	 \frac{1}{4} \norm{s_m - s_n}^{2} = \frac{1}{2} \left( \norm{x-s_{n}}^{2} + \norm{x-s_m}^{2} \right) - \frac{1}{4} \norm{x - \frac{s_n + s_m}{2}}^{2}
	 \label{eqn:2.1.1}
     \end{align}
     Now since $d\left( s_n ,x \right)$ converges to $\delta$, we must have that $d \left( s_{n} , x \right)^{2}$ converges to $\delta ^{2}$ and hence there is some $K\in \N$ such that for all $i\ge K$, 
     \begin{equation*}
	 \norm{x-s_{i}}^{2} < \delta^{2} + \frac{\varepsilon ^2}{4} 
     \end{equation*}
     Now for all $n,m \ge K$ and from equation \ref{eqn:2.1.1}, we have that 
\begin{align*}
    \norm{s_m - s_n}^{2} &= 2 \left( \norm{x-s_{n}}^{2} + \norm{x-s_m}^{2} \right) - \norm{x - \frac{s_n + s_m}{2}}^{2} \\
	 &\stackrel{(!)}{<} 2 \cdot 2\left( \delta ^{2} + \frac{\varepsilon ^2}{4} \right)-  4 \delta ^{2} \\
	 &= \varepsilon^2
     \end{align*}
     Note that in inequality $(!)$, we made use of the convexity of $S$ to conlude that $\frac{s_n + s_m}{2} \in S$. 
     This shows that $\left( s_{n} \right)$ is Cauchy. Now, since $H$ is a Hilbert space, $\left( s_{n} \right)$ must converge to some $s_{0} \in H$. Closedness of $S$ allows us to conclude that $s_0$ must be in $S.$

     Hence, $x -s_{n}$ converges to $x- s_{0}$. By Theorem \ref{thm:norm-is-uc}, we conclude that $\norm{x-s_{n}}$ converges to $\norm{x-s_0}$. Since $\norm{x-s_{n}}$ also converges to $\delta$, we have by uniqueness of limits that $\delta = \norm{x-s_{0}}$.

     It remains to prove the uniqueness of such a vector. Let us suppose that $s_0$ and $t_0$ be two vectors such that $\norm{x-s_{0}} = \norm{x-t_{0}}=\delta$.

     Applying parallelogram identity on the vectors $s_{0}$ and $t_0$ as in Equation \ref{eqn:2.1.1}, we get
\begin{align*}
	 \frac{1}{4} \norm{s_0 - t_0}^{2} &= \frac{1}{2} \left( \norm{x-s_{0}}^{2} + \norm{x-t_0}^{2} \right) - \frac{1}{4} \norm{x - \frac{s_0 + t_0}{2}}^{2} \\
	 & \le \delta^2 -  \norm{x - \frac{s_0 + t_0}{2}}^{2} \\
	 & \le 0
     \end{align*}
     Hence, $s_0 = t_0$ and this completes the proof of the theorem.
\end{proof}

\begin{example}[distance is achieved but the vector may not be unique] Consider the normed linear space $\left( \R ^{2}, \norm{\cdot} _{1} \right)$. Now consider the subset $S$ of $\R ^2$ given by  $$S= \left\{ \left( x_{1} , x_{2} \right) : x_{1} + x_{2} =1 \right\}.$$ 
Note that $d \left( (0,0),S \right)=1= d\left( 0,0 \right), \left( 1,0 \right)) = d\left( \left( 0,0 \right), \left( 0,1 \right) \right)$. Hence, the uniqueness is not guaranteed.
\end{example}

\begin{exercise}
    Consider the space $\left( C[0,1] \right)$ with the supremum norm $\| \cdot\|_{\infty},$ that is, $\|f\|_{\infty}= \sup \big\{|f(x)| : x\in[0,1] \big \}.$ Let $S$ be the set
    \begin{equation*}
	S=\left\{ f \in C[0,1] \, : \, \int_{0}^{1/2} f\left( x \right) dx - \int_{1/2}^{1} f \left( x \right) dx =1 \right\}.
    \end{equation*}
    Show that the set $S$ is closed and convex but the distance $ d(0,S)=1,$ is never achieved at any point in $S$. That is, it is not the case that there is some $f\in S$ such that $d(0,f)= \norm{f}_{\infty} = 1$.
\end{exercise}
\begin{proof}[Solution]
    We begin by showing that $S$ is convex. Let $f, g \in S$ and $t\in [0,1]$. Then we have that 
    \begin{align*}
	\int_{0}^{1/2} \left( t f\left( x \right) + \left( 1-t \right) g\left( x \right)\right) dx - \int_{1/2}^{1} \left( t f\left( x \right) + \left( 1-t \right) g\left( x \right) dx  \right) &= t + (1-t) \\
	&= 1
    \end{align*}
    Note that the second equality follows by the virtue of $f,g \in S$.

    Now, we proceed to show that the $S$ is closed. Let $\left( f_{n} \right) $ be a sequence of functions in $S$ converging to $f \in C\left[ 0,1 \right]$. We need to prove that $f \in S$. Now convergence in supremum norm is the same as the uniform convergence, so, we have that following:
    \begin{align*}
	\lim_{n\to \infty}\left( \int_{0}^{1/2} f_{n}\left( x \right) dx - \int_{1/2}^{1} f_{n} \left( x \right) dx \right) =1
    \end{align*}
    implies 
    \begin{align*}
\int_{0}^{1/2} f\left( x \right) dx - \int_{1/2}^{1} f \left( x \right) dx =1 
\end{align*}
and thus $f \in S$.
Consider the zero function and the set $S$, we show that that there is no $f \in S$ such that $d (0,S) = d(f,0)=\norm{f}_{\infty}$. \textcolor{red}{Incomplete!}
\end{proof}

\subsection{Projections}
\begin{theorem}
Let $H$ be a Hilbert space. For any fixed $y\in H,$ consider the map $L_y : H \to \mathbb C$ defined by $L_y(x)= \langle x, y \rangle,\,\,x\in H.$ Then $L_y$ is a continuous linear functional on $H.$ 
    \label{thm:ip-linear-functional}
\end{theorem}

\begin{proof}
    Let $y\in H$ be fixed. Consider the function $L_{y} : H \to \C$ given by $L_{y} \left( x \right) = \ip{x,y}$ for each $x \in H$. We show that $L_{y}$ is Lipschitz continuous.
    Let $x_0 \in H$. If $x\in H$, we have that
    \begin{align*}
	\abs{L_{y} \left( x \right) - L_{y} x_{0}} &= \abs{\ip {x,y}-\ip{x_{0},y}}	\\
	&= \abs{\ip{x-x_{0},y}} \\
	& \le \norm{x-x_{0}} \norm{y}
    \end{align*}
    Note the inequality follows from Cauchy Schwarz and this completes the proof.
\end{proof}

\begin{definition}
    Let $H$ be a Hilbert space. For any $y\in H,$ the symbol $y^{\perp}$ denote the subspace defined by
\begin{align*}
    y^{\perp}:= \{x\in H : \langle x,y\rangle=0\}
    \label{def:orthonormal-of-subspace}
\end{align*}
\end{definition}

Observe that $y^{\perp}$ is a closed subspace of $H$. This is because $y^{\perp}$ is the kernel of the continuous map $L_{y}$ as given by Theorem \ref{thm:ip-linear-functional}.

\begin{definition}
    Let $H$ be a Hilbert space. Let $M$ be any subspace of $H$. Let the symbol $M^{\perp}$ denote the subspace given by 
   \begin{align*}
M^{\perp} = \{x\in H : \langle x,y\rangle =0\,\mbox{for all}\,y\in M\}= \bigcap_{y\in M} y^{\perp}.  \end{align*} 

\end{definition}

Observe that $M^{\perp}$ is always closed since it is intersection of closed subspaces of $H$.

\begin{theorem}[Existence of an Orthogonal Projection onto a closed subspaces]
Let $M$ be a closed subspace of a Hilbert space $H.$ Then 
\begin{itemize}
\item [(a)] Every $x\in H$ has a unique decomposition 
\begin{align*}
x= Px + Qx
\end{align*}
into a sum of $Px\in M$ and $Qx\in M^{\perp}.$ Thus $H = M \oplus M^{\perp}.$
\item [(b)] $Px$ and $Qx$ are the nearest points to $x$ in $M$ and in $M^{\perp}$ respectively.

\item [(c)] The mappings $P: H \to M$ and $Q: H \to M^{\perp}$ are linear and satisfies $P^2=P$ and $Q^2=Q.$ The map $P$ and $Q$ are called the \textbf{orthogonal projection onto $M$ and $M^{\perp}$} respectively.

\item [(d)] $\|x\|^2= \|Px\|^2 + \|Qx\|^2$ for every $x\in H.$
\end{itemize}
\label{thm:existence-of-o-proj}
\end{theorem}

\begin{proof}
    Since subspaces are convex, we can appeal to Theorem \ref{thm:closed-and-convex} as we please. We now start to prove each of the statements of theorem:
    \begin{itemize}
	\item [(a)] Let $x \in H$ be arbitrary. Then by the Theorem \ref{thm:closed-and-convex} there is a unique vector $Px \in M$ such that 
	    \begin{equation*}
		d(x, M) = \norm{x-Px}
	    \end{equation*}
	    Define $Qx \in M$ by $Qx = x-Px$. We need to show that $Qx \in M^{\perp}$. Let $y \in M$. We want to show that $\ip{x-Px , y} = 0$. To do so, observe that
	    \begin{align}
		\ip{Qx - \ip{Qx,y} \frac{y}{\norm{y}^{2}},y}&= \ip{Qx,y} - \ip{\ip{Qx,y} \frac{y}{\norm{y}^{2}},y}
		=0
		\label{eqn:2.2.1}
	    \end{align}

	    Now, 
	    \begin{align*}
		Qx = \underbrace{\left( Qx - \ip{Qx, y} \frac{y}{\norm{y}^{2}} \right)}_{=: v_1} + \underbrace{\ip{Qx,y} \frac{y}{\norm{y}^{2}}}_{=:v_2}
	    \end{align*}

	    Note that by equation \ref{eqn:2.2.1}, $v_1$ and $v_2$ are orthogonal and hence by Pythagoras theorem\footnote{In inner product space, if $\ip{v_{1}, v_{2}}=0$ then $\norm{v_{1} + v_{2}}^{2} = \norm{v_{1}}^{2} + \norm{v_{2}}^{2}$} for inner product spaces, we may write
\begin{align*}
    \delta ^{2} = \norm{Qx}^{2} &= \norm{ Qx - \ip{Qx, y} \frac{y}{\norm{y}^{2}}}^{2} + \norm{\ip{Qx,y} \frac{y}{\norm{y}^{2}}}^{2} \\
    &= \norm{ Qx - \ip{Qx, y} \frac{y}{\norm{y}^{2}}}^{2} + \frac{\abs{\ip{Qx,y}}}{\norm{y}^{2}} \\
    &= \norm{ x- Px - \ip{Qx, y} \frac{y}{\norm{y}^{2}}}^{2} + \frac{\abs{\ip{Qx,y}}}{\norm{y}^{2}}
    &\ge \delta ^{2} +  \frac{\abs{\ip{Qx,y}}}{\norm{y}^{2}}
	    \end{align*}

	    and thus, we have that $\abs{\ip{Qx,y}} = 0$. This completes the proof of (a).
	\item [(b)] By uniqueness of part (a), it follows that $Px$ is the nearest point to $x$ in $M$. It remains to prove that $Qx$ is the nearest point to $x$ in $M^{\perp}$. Note that $x-Qx=Px\in M.$ Now for any $y\in M^{\perp}$ we have  $Qx-y\in M^{\perp}.$ Thus we get 
	\begin{align*}
	\|x-y\|^2= \|(x-Qx) + (Qx -y)\|^2 = \|x-Qx\|^2 + \|Qx -y\|^2 \geqslant \|x-Qx\|^2.
	\end{align*}
This shows that $Qx$ is the nearest point to $x$ in $M^{\perp}.$	

	\item [(c)] Let $x_{1}, x_{2} \in M$. By part $(a)$, we have that 
	    \begin{align*}
		x_{1} &= Px_{1} + Qx_{1} \\
		x_{2} &= Px_{2} + Qx_{2} \\
		x_{1} + x_{2} &= P\left( x_{1} + x_{2} \right) + Q \left( x_{1} + x_{2} \right)
	    \end{align*}

	    Now taking sums and rearranging, we have that
	    \begin{equation*}
		\underbrace{Px_{1} + Px_{2} - P\left( x_{1} + x_{2} \right)}_{\in M} = \underbrace{Q\left( x_{1} + x_{2} \right) - Qx_{1} - Qx_{2}}_{\in M^{\perp}}
	    \end{equation*}
	    Since $M \cap M^{\perp} = \left\{ 0 \right\}$, the linearity of $P$ and $Q$ follows.

	    Now, let $x\in P$. We need to prove that $P^{2}x = Px$. Now note that $Px \in M$. Thus by part (a) we have
	    \begin{equation*}
		Px = P^{2} x + QPx
	    \end{equation*}
	    By uniqueness of part (a), we must have that $Px = P^{2}x$. This completes the proof. $Q^{2}= Q$ can be proved similarly.

	\item [(d)] This follows immediately from Pythagoras theorem.
    \end{itemize}
\end{proof}

\begin{corollary}
Let $M$ be a closed subspace of a Hilbert space $H.$ Then $(M^{\perp})^{\perp} = M.$ In case $M$ is a subspace then $(M^{\perp})^{\perp} = \overline{M},$ the closure of $M$ in $H.$
\label{cor:M-perp-perp}
\end{corollary}



