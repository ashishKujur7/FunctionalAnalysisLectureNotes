\section{Lecture 1  --- Introduction to Hilbert Spaces and some examples --- 9th January, 2023}

\subsection{Inner Product Spaces}
\begin{definition}[Inner Product]
    Let $V$ be a vector space over a field $\mathbb F$ (where $\mathbb F$ is $\R$ or $\C$). A function $\langle \cdot , \cdot \rangle : V \times V \to \mathbb F$ is called an \textit{inner product} if it satisfies the following properties
    \begin{enumerate}
	\item $\ip{x, x} \ge 0$
	\item $\ip{x+y, z} = \ip{x, z} + \ip{y, z}$
	\item $\ip{\alpha x, y} = \alpha \ip{x, y}$
	\item $\ip{x, y} = \overline{\ip{y, x}}$
    \end{enumerate}
    for all $x,y,z \in V$ and $\alpha \in \F$. A vector space $V$ with an inner product is called an \textit{inner product space}.
    \label{def:inner-product}
\end{definition}

\begin{example}[Examples of inner product spaces]
    Here are some examples of inner product spaces:
    \begin{enumerate}
	\item The obvious first example is that of $\R ^{n}$ with the standard $2$-inner product given by
	    \begin{equation*}
		\ip {x,y} = \sum_{i=1}^{n} x_{i} \overline{y_{i}}
	    \end{equation*}

	\item One can then consider the space $\ell ^{2} \left( \N \right)$ which is the vector space of all square summable sequences on $\C$. That is,
	    \begin{equation*}
		\ell ^{\infty} = \left\{ \left( x_{n} \right) \in \C ^{\N} \, \mid \, \sum_{i\in \N} |a_i|^{2} < \infty \right\} \end{equation*}
We define an inner product on this vector space by 
 \begin{equation*}
		\ip {x,y} = \sum_{i=1}^{\infty} x_{i} \overline{y_{i}}
	    \end{equation*}
	    One can show using Holder's inequality that the sum turns out to be finite and the "inner product" is indeed an inner product.
	\item Next, we consider the vector space of all polynomials over $\C$ which we denote by $\C \left[ x \right]$. If $p , q \in \C [x]$, we define an inner product on $\C \left[ x \right]$ by 
	    \begin{equation*}
		\ip{p,q} = \int_{0}^{1} p\overline q \, \mathrm{dx}
	    \end{equation*}
	\item One can define inner products on $C[0,1]$ and $L^{2} (X, \mathscr A , \mu)$ in an similar fashion as in item 3. Note that $\left( X, \mathscr A, \mu \right)$ is a measure space.
    \end{enumerate}
\end{example}

\begin{definition}
    Let $V$ be an inner product space. We can define a function $\norm{\cdot} : V \to \R _{\ge 0}$ by
    \begin{equation*}
	\norm{x} = \sqrt{\ip{x,x}}
    \end{equation*}
    We call this function \textit{norm induced by the inner product}. This norm is indeed a norm as one can check!
    \label{def:norm-induced-by-ip}
\end{definition}

The proof of the following theorems are skipped:
\begin{theorem}[Cauchy Schwarz inequality]
    Let $V$ be an inner product space, $x,y \in V$. Then we have that 
    \begin{equation*}
	\lvert \ip{x,y} \le \norm{x}\norm{y}
    \end{equation*}
    \label{thm:CS-inequality}
\end{theorem}

\begin{theorem}[Triangle Inequality]
    Let $V$ be an inner product space, $x,y \in V$. Then we have that 
    \begin{equation*}
	\norm{x+y} \le \norm{x} + \norm{y}
    \end{equation*}
    \label{thm:triangle-inequality}
\end{theorem}

\subsection{Hilbert Spaces}
\begin{definition}
    Let $V$ be an inner product space. One can consider $V$ as a metric space by defining the following metric $d$:
    \begin{equation*}
	d(v,w) = \norm{v-w}
    \end{equation*}
    for all $v,w \in V$. Then $(V,d)$ is a metric space \textcolor{red}{(Check!)}. We say that $V$ is a \textit{Hilbert Space} if $(V,d)$ is a complete metric space.
    \label{def:Hilbert-Space}
\end{definition}

\begin{example} We consider some examples and not-so-example of Hilbert Space:
\begin{enumerate}
    \item $\R ^{n}$ and $\C ^{n}$ with the standard inner product are complete!
    \item $\ell ^{2} (\N )$ is complete.
    \item $L^2(X)$ is complete where $(X, \mathscr A , \mu)$ is a measure space.
    \item $\mathcal {C} [0,1]$ is not complete. \textcolor{blue}{(Needs Baire Category Theorem!)}
    \item Consider$c_{00} = \left\{ \left( x_{n} \right)_{n\in \N} \in \ell ^{2} (\N) \, : \, \left( x_{n} \right) _{n \in \N} \text{ is eventually zero} \right\}$. $c_{00}$ has the induced inner product. We show that $c_{00}$ with this induced product is not complete!
	One consider the sequence of sequences given by
	\begin{equation*}
	    f_{n} = \left( 1, \frac{1}{2} , \ldots , \frac{1}{n} , \ldots \right)
	\end{equation*}
	One can then easily show that $\left( f_n \right)$ is Cauchy but it does not converge in $c_{00}$.
\end{enumerate}
    
\end{example}
