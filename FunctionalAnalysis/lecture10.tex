\section{Lecture 10 --- \textit{Bounded Linear Maps} --- 6th February, 2023}
\horz
\subsection{Continuing from where we left from\ldots}


\begin{theorem}
    Let $X$ be a NLS. The following are equivalent:
    \begin{enumerate}
	\item $X$ is complete.
	\item Every absolutely summable sequence is summable.
    \end{enumerate}
    \label{thm:equivalent-thm-Banach}
\end{theorem}
\begin{proof}
    $\left( \Longrightarrow \right)$ This is just Lemma \ref{lemma:Banach-absolutely-summable}.

    $\left( \Longleftarrow \right)$ Let $\left\{ y_{n} \right\}$ be Cauchy in $Z$. It suffices to find a subsequence of $\left\{ y_{n} \right\}$ which converges. Since $\left\{ y_n \right\}$ is Cauchy, for every $k\in \N$, there is some $N_{k} \in \N$ such that 
    \begin{align*}
	\norm{y_{i} - y_{j}} < \frac{1}{2^{k}} \text{ for every } i, j \ge N_{k}\text{.}
    \end{align*}
    We may select $N_{k}$'s in a way that $N_{1} \le N_{2} \le \ldots$. Now defne the sequence
    \begin{align*}
	t_{j} = y_{N_{j}}
    \end{align*}
    for every $j \in \N$.

    Hence, we have that for every $j \in \N$,
    \begin{align*}
    \norm{t_{j} - t_{j+1}} \le \norm{y_{N_{j}} - y_{N_{j+1}}} \le \frac{1}{2^{j}}
    \end{align*}
    
    Now note that the sequence $\left\{ t_{j} - t_{j+1} \right\}_{j}$ is absolutely summable. By hypothesis, we have that $\left\{ t_{j} - t_{j+1} \right\}_{j}$is summable. But that is the same as saying $\sum_{j} {t_j}$ is convergent. Hence, we have that a subsequence of $\left\{ y_{n} \right\}$, namely, $\left\{ t_{j} \right\}_{j}$ is convergent and hence we are done.
\end{proof}

We proceed to prove Proposition \ref{prop:quotient-are-complete}:

\begin{proof}
    Let $[x_{n}]$ be an absolutely summable in $X/M$, that is, 
    \begin{align*}
	\sum_{n=1}^{\infty} \norm{[x_{n}]} < \infty
    \end{align*}
    We want to show that 
    \begin{align*}
	\sum_{k=1}^{n} [x_{k}] \to [x] 
    \end{align*}
    for some $x\in X$ as $n\to \infty$.

    By the definition of the norm in the quotient space, for each $k\in \N$, we can find $m_{k} \in M$ such that
    \begin{align*}
	\norm{x_{k} + m_{k}} < \norm{[x_{k}]} + \frac{1}{2^{k}}
    \end{align*}
    From the above, we conclude that $\sum_{k=1}^{\infty} (x_{k} + m_{k})$ is summable. Hence,  $\sum_{k=1}^{\infty} (x_{k} + m_{k})$ converges to some $x\in M$. By the continuity of the projection, we are done (see below).
\end{proof}

\begin{proposition}
    Projection map is continuous.
    \label{prop:proj-is-cont}
\end{proposition}

\subsection{Continuous Linear Maps on Normed Linear Spaces}

\begin{example}
    Consider the linear map $L: c_{00} \to \R$ given by
    \begin{align*}
	\left( x_{j} \right) \mapsto x_{1} + 2 x_{2} + \ldots + nx_{n} + \ldots
    \end{align*}
    This map is not continuous because we can consider the sequence $\left\{ e_{j}/j \right\}$. Note that $e_{j} /j \to 0$ but $T\left( e_{j} \right)/j = 1$ for every $j\in \N$.
\end{example}

\begin{proposition}
    Let $T: X\to Y$ be a linear map. The following are equivalent:
    \begin{enumerate}
	\item $T$ is continuous map.
	\item $T$ is continuous at $x=0$.
	\item $T\left( \overline{B_{X}\left( 0,1 \right)} \right)$ is bounded, that is, where $\overline{B_X(0,1)}$ denotes the closed unit ball in $X$ with centre at the origin. Equivalently speaking $$\sup \{ \|Tx\|_Y : \|x\| \leqslant 1\} < \infty.$$
	\item $\norm{Tx} \le M \norm{x}$ for every $x\in X$ .
    \end{enumerate}
    \label{prop:equiv-lin-maps}
\end{proposition}
\begin{proof}
    It is easy to see that item 1 implies item 2. 

    We proceed to show that item 2 implies item 3. Suppose tha $T$ is continuous at $x=0$. We need to show that the set $T\left( \overline{B_{X}\left( 0,1 \right)} \right)$ is bounded. Since $T$ is continuous at $0$, there is some $\delta > 0$ such that 
    \begin{align*}
	\norm{x}_{X} < \delta \leadsto \norm{Tx}_{Y} < 1
    \end{align*}
    Now, let $x\in X$ with $\norm{x}_{X} \le 1$. Then we have that $\norm{\frac{\delta}{2}x}_{X}<\delta$ and hence we have that
    \begin{align*}
	\norm{T\left( \frac{\delta}{2} x \right)}_{Y} <1 \leadsto \norm{Tx}_{Y} < \frac{2}{\delta}
    \end{align*}
    This shows that $T\left( \overline{B_{X} \left( 0,1 \right) } \right)$ is bounded.
    Now, we proceed to show that item 3 implies item 4. Suppose that $T\left( \overline{B_{X} \left( 0,1 \right) } \right)$ is bounded. That is, there is some $M > 0$ such that
    \begin{align*}
	\norm{x}_{X} \le 1 \leadsto \norm{Tx}_{Y} \le M
    \end{align*}
    Now, let $y\in X$ be arbitrary. If $y=0$ then we have that $\norm{Ty} = 0 \le M \norm{y}$. Suppose that $y\ne 0$. Then we have that the vector $y/\norm{y}$ has norm at most $1$ and therefore we have that 
    \begin{align*}
	\norm{T\left( y/\norm{y}_{Y} \right)} \le M \leadsto \norm{T \left( y \right)} \le M \norm{y}
    \end{align*}
    
    Now, we proceed to prove that item 4 implies item 1. Because item 4 with linearity implies Lipschitz continuity.
\end{proof}

\begin{proposition}
    Every linear map on a finite dimensional normed linear space is continuous.
    \label{prop:lin-map-on-fin-dim-cont}
\end{proposition}

\begin{example}
    $T: \ell ^{2} \left( \N \right) \to \ell ^{2} \left( \N \right)$
    \begin{align*}
	\left( x_{1},x_{2}, \ldots \right) \mapsto \left( \lambda_{1} x_{1}, \lambda _{2} x_{2}, \ldots \right)
    \end{align*}
    We show that $T$ is continuous iff $\left\{ \lambda_{i} \right\}$ is bounded.
    $\left( \Rightarrow \right)$ Suppose that $T$ is continuous. By Proposition \ref{prop:equiv-lin-maps}, we have that there is some $M>0$ such that
    \begin{align*}
	\norm{Tx} \le M \norm{x}
    \end{align*}
    for each $x\in \ell ^{2} \left( \N \right)$. Therefore for any $i\in \N$, we have that
    \begin{align*}
	\abs{\lambda_{i}} = \norm{Te_{1}}_{2} \le M \norm{e_{i}}_{2} = M\text{.}
    \end{align*}
    This shows that $\left\{ \lambda_{i} \right\}$ is a bounded sequence. Conversely suppose that $\left\{ \lambda_{i} \right\}$ is a bounded sequence. Therefore, there is some $M> 0$ such that $\abs{\lambda_{i}} \le M$ for all $i\in \N$. Now for any $x\in \ell ^{2} \left( \N \right)$, we have that
    \begin{align*}
	\norm{T \left( x_{1}, x_{2}, \ldots \right)} &= \norm{\left( \lambda_{1} x_{1}, \lambda_{2} x_{2}, \ldots \right)}_{2} \\
	&= \left( \sum_{i=1}^{\infty} \abs{\lambda_{i}x_{i}}^{2} \right)^{\frac{1}{2}} \\
	& \le M \left( \sum_{i=1}^{\infty} \abs{x_{i}}^{2} \right)^{\frac{1}{2}} \\
	&= M\norm{\left( x_{1}, x_{2}, \ldots \right)}_{2}\text{.}
    \end{align*}
\end{example}

\begin{example}[A discontinuous linear functional]
Consider the normed linear space $X = (c_{00},\|\cdot\|_1).$ Consider the linear map $T: X \to \mathbb C$ defined by 
$$T \Big((x_j)_{j\in\mathbb N}\Big)= \sum\limits_{k=1}^{\infty} kx_k,\,\,\,\,((x_j)_{j\in\mathbb N}\in c_{00}.$$ 
Note that $\frac{e_n}{n}\to 0$ as $n \to \infty$ in  $(c_{00},\|\cdot\|_1)$ but $T(\frac{e_n}{n})=1$ for all $n\in\mathbb N.$ Hence $T$ is not continuous. (Here $e_n$ denote the coordinate sequence whose $n$-th term is $1$ and all other terms are $0.$)
    
\end{example}
