\section{Lecture 10 --- \textit{insert title here} --- 6th February, 2023}
\horz
\subsection{Continuing from where we left from\ldots}


\begin{theorem}
    Let $X$ be a NLS. The following are equivalent:
    \begin{enumerate}
	\item $X$ is complete.
	\item Every absolutely summable sequence is summable.
    \end{enumerate}
    \label{thm:equivalent-thm-Banach}
\end{theorem}
\begin{proof}
    $\left( \Longrightarrow \right)$ This is just Lemma \ref{lemma:Banach-absolutely-summable}.

    $\left( \Longleftarrow \right)$ Let $\left\{ y_{n} \right\}$ be Cauchy in $Z$. It suffices to find a subsequence of $\left\{ y_{n} \right\}$ which converges. Since $\left\{ y_n \right\}$ is Cauchy, for every $k\in \N$, there is some $N_{k} \in \N$ such that 
    \begin{align*}
	\norm{y_{i} - y_{j}} < \frac{1}{2^{k}} \text{ for every } i, j \ge N_{k}\text{.}
    \end{align*}
    We may select $N_{k}$'s in a way that $N_{1} \le N_{2} \le \ldots$. Now defne the sequence
    \begin{align*}
	t_{j} = y_{N_{j}}
    \end{align*}
    for every $j \in \N$.

    Hence, we have that for every $j \in \N$,
    \begin{align*}
    \norm{t_{j} - t_{j+1}} \le \norm{y_{N_{j}} - y_{N_{j+1}}} \le \frac{1}{2^{j}}
    \end{align*}
    
    Now note that the sequence $\left\{ t_{j} - t_{j+1} \right\}_{j}$ is absolutely summable. By hypothesis, we have that $\left\{ t_{j} - t_{j+1} \right\}_{j}$is summable. But that is the same as saying $\sum_{j} {t_j}$ is convergent. Hence, we have that a subsequence of $\left\{ y_{n} \right\}$, namely, $\left\{ t_{j} \right\}_{j}$ is convergent and hence we are done.
\end{proof}

We proceed to prove Proposition \ref{prop:quotient-are-complete}:

\begin{proof}
    Let $[x_{n}]$ be an absolutely summable in $X/M$, that is, 
    \begin{align*}
	\sum_{n=1}^{\infty} \norm{[x_{n}]} < \infty
    \end{align*}
    We want to show that 
    \begin{align*}
	\sum_{k=1}^{n} [x_{k}] \to [x] 
    \end{align*}
    for some $x\in X$ as $n\to \infty$.

    By the definition of the norm in the quotient space, for each $k\in \N$, we can find $m_{k} \in M$ such that
    \begin{align*}
	\norm{x_{k} + m_{k}} < \norm{[x_{k}]} + \frac{1}{2^{k}}
    \end{align*}
    From the above, we conclude that $\sum_{k=1}^{\infty} (x_{k} + m_{k})$ is summable. Hence,  $\sum_{k=1}^{\infty} (x_{k} + m_{k})$ converges to some $x\in M$. By the continuity of the projection, we are done (see below).
\end{proof}

\begin{proposition}
    Projection map is continuous.
    \label{prop:proj-is-cont}
\end{proposition}

\subsection{Continuous Linear Maps on Normed Linear Maps}

\begin{example}
    Consider the linear map $L: c_{00} \to \R$ given by
    \begin{align*}
	\left( x_{j} \right) \mapsto x_{1} + 2 x_{2} + \ldots + nx_{n} + \ldots
    \end{align*}
    This map is not continuous because we can consider the sequence $\left\{ e_{j}/j \right\}$. Note that $e_{j} /j \to 0$ but $T\left( e_{j} \right)/j = 1$ for every $j\in \N$.
\end{example}

\begin{proposition}
    Let $T: X\to Y$ be a linear map. The following are equivalent:
    \begin{enumerate}
	\item $T$ is continuous map.
	\item $T$ is continuous at $x=0$.
	\item $T\left( B\left( 0,1 \right) \right)$ is bounded.
	\item $\norm{Tx} \le M \norm{x}$ for every $x\in X$ .
    \end{enumerate}
    \label{prop:equiv-lin-maps}
\end{proposition}

\begin{proposition}
    Every linear map on a finite dimensional normed linear space is continuous.
    \label{prop:lin-map-on-fin-dim-cont}
\end{proposition}

\begin{example}
    $T: \ell ^{2} \left( \N \right) \to \ell ^{2} \left( \N \right)$
    \begin{align*}
	\left( x_{1},x_{2}, \ldots \right) \mapsto \left( \lambda_{1} x_{1}, \lambda _{2} x_{2}, \ldots \right)
    \end{align*}
    Continuity iff $\left\{ \lambda_{i} \right\}$ is bounded.

\end{example}
